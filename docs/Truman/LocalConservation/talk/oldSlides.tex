 %===============================================================================
 % NEW SLIDE
 %===============================================================================
 \begin{frame}
 \frametitle{Motivation}
 Why DPG?
 \begin{itemize}
 \item Robust for singularly perturbed problems.
 \item Stable in the preasymptotic regime.
 \item Perfect for adaptive mesh refinement.
 \end{itemize}
 \bigskip
 
 Why local conservation?
 \begin{itemize}
 \item Local conservation is a priority for the CFD community.
 \item For a system of hyperbolic conservation laws, the Lax-Wendroff theorem
 guarantees that a convergent, locally conservative numerical method converges
 to the correct weak solution.
 \end{itemize}
 \bigskip
 
 Why convection-diffusion?
 \begin{itemize}
 \item Model problem for Navier-Stokes.
 \end{itemize}
 \end{frame}
 \begin{comment}
 Hopefully Nate and Jesse's talks explained some of the attractions of
 developing the discontinuous Petrov-Galkerin method for fluid problems, but
 here are a couple key points.
 DPG has proved robust in the face of singularly perturbed problems which holds
 promise for high Reynold's number flows.
 You do not need a domain expert to craft well designed meshes for each new
 problem. We are mathematically guaranteed to remain stable under very coarse
 meshes while adaptively refining toward a solution.
 
 But if we are concerned with realistic flows, why are we worried about local
 conservation in the convection-diffusion equation. First of all, in order for
 DPG to be accepted by the CFD community, we need to pacify some of their
 concerns about local conservation - a very weighty issue among CFD
 practitioners. There are also some attractive reasons, mathematically to
 develop a locally conservative method, such as the Lax-Wendroff theorem which
 uses local conservation as one of two conditions to guarantee that a numerical
 solution converges to the weak solution of a system of hyperbolic conservation
 laws. But before we jump in and start trying to enforce local conservation in
 our solutions to the Navier-Stokes equations, we are starting with something a
 little simpler, the convection-diffusion equation.
 \end{comment}
 
 
 %===============================================================================
 % The Abstract Problem and Minimization of the Residual
 %===============================================================================
 %===============================================================================
 % NEW SLIDE
 %===============================================================================
 \begin{frame}
 \frametitle{The Abstract Problem and Minimization of the Residual}
 Take $U,V$ Hilbert.\\
 \vspace{5mm}
 We seek $u \in U$ such that
 \[
 b(u,v) = l(v) \quad \forall v \in V,
 \]
 where $b$ and $l$ are linear in $v$.  Define $B$ by $Bu = b(u,\cdot) \in V'$; $Bu$ is a linear functional on the test space $V$.\\
 \vspace{5mm}
 We seek to minimize the residual in the discrete space $U_{h} \subset U$:
 \begin{align*}
 u_{h} = \underset{w_{h} \in U_{h}} \argmin \,\, \frac{1}{2} \norm{Bw_{h}-l}_{V'}^{2}.
 \end{align*}
 \end{frame}
 \begin{comment}
 Let's first develop the abstract theory of DPG before applying it to the
 convection diffusion equation and then modifying it to enforce local
 conservation. These next couple slides should look familiar, I borrowed them
 from Nate.
 
 Start with two Hilbert spaces. Our problem seeks a solution u in Hilbert space
 U that satisfies this bilinear form for all test functions in V. Now define
 operator B: U -> V' through the bilinear form. Our problem can then be
 thought of as minimizing a residual in the dual space of V.
 \end{comment}
 
 
 %===============================================================================
 % NEW SLIDE
 %===============================================================================
 \begin{frame}
 \frametitle{The Abstract Problem and Minimization of the Residual}
 \begin{align*}
 u_{h} = \underset{w_{h} \in U_{h}} \argmin \,\, \frac{1}{2} \norm{Bw_{h}-l}_{V'}^{2}.
 \end{align*}
 
 % PATTER: ``not especially easy to work with'': non-constructive; involves taking a supremum
 Now, the dual space $V'$ is not especially easy to work with; we would prefer to work with $V$ itself.  Recalling that the Riesz operator $R_{V} : V \rightarrow V'$ defined by 
 \begin{align*}
 \langle R_{V}v, \delta v\rangle=(v,\delta v)_{V}, \quad \forall \delta v \in V,
 \end{align*}
 is an \emph{isometry}---$\norm{R_{V}v}_{V'} = \norm{v}_{V}$---we can rewrite the term we want
 to minimize as a norm in $V$:
 
 \begin{align*}
 \frac{1}{2} \norm{Bw_{h}-l}_{V'}^{2} &= \frac{1}{2} \norm{R_{V}^{-1}\left(Bw_{h}-l\right)}_{V}^{2}\\
                                                               &= \frac{1}{2} \left(R_{V}^{-1}\left(Bw_{h}-l\right), R_{V}^{-1}\left(Bw_{h}-l\right) \right)_{V}.
 \end{align*}
 \end{frame}
 \begin{comment}
 Working in the dual space is not especially conducive to developing a
 numerical method, but through the Riesz representation theorem, we can
 associate every member of the dual space with a member of V through the Riesz
 map which is defined by the choice of norm on V. This is an isometry, meaning
 that norms are preserved after the map.
 
 Now, using the inverse Riesz map, we can reframe our problem as an inner
 product on the V space.
 \end{comment}
 
 
 %===============================================================================
 % NEW SLIDE
 %===============================================================================
 \begin{frame}
 \frametitle{The Abstract Problem and Minimization of the Residual}
 We seek to minimize
 \begin{align*}
 \frac{1}{2} \left(R_{V}^{-1}\left(Bw_{h}-l\right), R_{V}^{-1}\left(Bw_{h}-l\right) \right)_{V}.
 \end{align*}
 The first-order optimality condition requires that the G\^ateaux derivative be equal to zero for minimizer $u_{h}$; assuming $B$ is linear, we have
 \begin{align*}
 \left(R_{V}^{-1}\left(Bu_{h}-l\right), R_{V}^{-1}B \delta u_{h} \right)_{V} = 0, \quad \forall \delta u_{h} \in U_{h}.
 \end{align*}
 By the definition of $R_{V}$, this is equivalent to
 \begin{align*} % PATTER: use the phrase ``duality pairing''
 \langle Bu_{h} - l, R_{V}^{-1}B \delta u_{h} \rangle = 0 \quad \forall \delta u_{h} \in U_{h}.
 \end{align*}
 \end{frame}
 \begin{comment}
 So the goal is to minimize this inner product. Obviously, the Gateaux
 derivative will be zero at the minimizer. And by definition of the Riesz
 map, we can turn this inner product into a duality pairing.
 \end{comment}
 
 
 %===============================================================================
 % NEW SLIDE
 %===============================================================================
 \begin{frame}
 \frametitle{The Abstract Problem and Minimization of the Residual}
 We have:
 \begin{align*}
 \langle Bu_{h} - l, R_{V}^{-1}B \delta u_{h} \rangle = 0 \quad \forall \delta u_{h} \in U_{h}.
 \end{align*}
 Now, if we identify $v_{\delta u_{h}} = R_{V}^{-1}B \delta u_{h}$ as a test function, we can rewrite this as
 \begin{align*}
 b(u_{h},v_{\delta u_{h}}) &= l(v_{\delta u_{h}}).
 \end{align*}
 
 Thus, the discrete solution that minimizes the residual is exactly attained by
 testing the original equation with appropriate test functions.  We call these
 \pecosbold{optimal test functions}.\footnote{\FootSize \bibentry{DPG2}}
 \end{frame}
 \begin{comment}
 We are getting close to a recognizable bilinear form again. If we now identify
 our test function as the inverse Riesz map applied to B applied to delta uh,
 we get a bilinear form with a very specific test function that minimizes the
 residual. We call these optimal test functions.
 \end{comment}


 %===============================================================================
 % NEW SLIDE
 %===============================================================================
 \begin{frame}
 \frametitle{DPG for Convection-Diffusion}
 Combining the two equations, we arrive at the bilinear form.
 \bigskip
 
 Find $\mathbf{u}=\{u,\bsigma,\hat u,\hat f\}\in\mathbf{U}$
 such that
 \[
 b(\mathbf{u},\mathbf{v}) =
 l(\mathbf{v})\quad\forall\mathbf{v}=\{v,\btau\}\in\mathbf{V}
 \]
 where $b$ and $l$ are linear in $\mathbf{v}$ and
 \[
 b(\mathbf{u},\mathbf{v}) =
 \int_K-(\bbeta u-\bsigma)\cdot\nabla v+\frac{1}{\epsilon}\bsigma\cdot\btau+u\nabla\cdot\btau
 +\int_{\partial K}\hat f v-\hat u\btau\cdot n\,,
 \]
 \[
 l(\mathbf{v})=\int_K gv\,,
 \]
 \[
 \begin{array}{lll}
 u\in L^2(K) & \hat u\in H^1(K)|_{\partial K} & v\in H^1(K)\\
 \bsigma\in \mathbf{L}^2(K) & \hat f\in H(div,K)|_{\partial K} & \btau\in H(div,K)\\
 \end{array}\,.
 \]
 Define $B$ by $B\mathbf{u}=b(\mathbf{u},\cdot)\in\mathbf{V}'$.
 \end{frame}

