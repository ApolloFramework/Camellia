\documentclass[letterpaper]{article}
\usepackage[margin=1.0in]{geometry}
\usepackage{authblk}
\usepackage{amsmath, amssymb, amsthm, mathtools}
\usepackage{graphicx}
\usepackage{subfigure}

\def\btau{\boldsymbol\tau}
\def\bsigma{\boldsymbol\sigma}
\def\bbeta{\boldsymbol\beta}
\def\blambda{\boldsymbol\lambda}

\newcommand{\bs}[1]{\boldsymbol{#1}}
\DeclareMathOperator{\diag}{diag}
\DeclareMathOperator*{\argmin}{\arg\,\min}

\newcommand{\equaldef}{\stackrel{\mathrm{def}}{=}}

\newcommand{\tablab}[1]{\label{tab:#1}}
\newcommand{\tabref}[1]{Table~\ref{tab:#1}}

\newcommand{\theolab}[1]{\label{theo:#1}}
\newcommand{\theoref}[1]{\ref{theo:#1}}
\newcommand{\eqnlab}[1]{\label{eq:#1}}
\newcommand{\eqnref}[1]{\eqref{eq:#1}}
\newcommand{\seclab}[1]{\label{sec:#1}}
\newcommand{\secref}[1]{\ref{sec:#1}}
\newcommand{\lemlab}[1]{\label{lem:#1}}
\newcommand{\lemref}[1]{\ref{lem:#1}}

\newcommand{\mb}[1]{\mathbf{#1}}
\newcommand{\mbb}[1]{\mathbb{#1}}
\newcommand{\mc}[1]{\mathcal{#1}}
\newcommand{\nor}[1]{\left\| #1 \right\|}
\newcommand{\snor}[1]{\left| #1 \right|}
\newcommand{\LRp}[1]{\left( #1 \right)}
\newcommand{\LRs}[1]{\left[ #1 \right]}
\newcommand{\LRa}[1]{\left\langle #1 \right\rangle}
\newcommand{\LRc}[1]{\left\{ #1 \right\}}
\newcommand{\tanbui}[2]{\textcolor{blue}{\sout{#1}} \textcolor{red}{#2}}
\newcommand{\Grad} {\ensuremath{\nabla}}
\newcommand{\Div} {\ensuremath{\nabla\cdot}}
\newcommand{\Nel} {\ensuremath{{N^\text{el}}}}
\newcommand{\jump}[1] {\ensuremath{\LRs{\![#1]\!}}}
\newcommand{\uh}{\widehat{u}}
\newcommand{\fnh}{\widehat{f}_n}
\renewcommand{\L}{L^2\LRp{\Omega}}
\newcommand{\pO}{\partial\Omega}
\newcommand{\Gh}{\Gamma_h}
\newcommand{\Gm}{\Gamma_{-}}
\newcommand{\Gp}{\Gamma_{+}}
\newcommand{\Go}{\Gamma_0}
\newcommand{\Oh}{\Omega_h}

\newcommand{\eval}[2][\right]{\relax
  \ifx#1\right\relax \left.\fi#2#1\rvert}

\def\etal{{\it et al.~}}

\newcommand{\vect}[1]{\ensuremath\boldsymbol{#1}}
\newcommand{\tensor}[1]{\underline{\vect{#1}}}
\newcommand{\del}{\Delta}
\newcommand{\grad}{\nabla}
\newcommand{\curl}{\grad \times}
\renewcommand{\div}{\grad \cdot}
\newcommand{\ip}[1]{\left\langle #1 \right\rangle}
\newcommand{\eip}[1]{a\left( #1 \right)}
\newcommand{\pd}[2]{\frac{\partial#1}{\partial#2}}
\newcommand{\pdd}[2]{\frac{\partial^2#1}{\partial#2^2}}

\newcommand{\circone}{\ding{192}}
\newcommand{\circtwo}{\ding{193}}
\newcommand{\circthree}{\ding{194}}
\newcommand{\circfour}{\ding{195}}
\newcommand{\circfive}{\ding{196}}

\def\arr#1#2#3#4{\left[
\begin{array}{cc}
#1 & #2\\
#3 & #4\\
\end{array}
\right]}
\def\vecttwo#1#2{\left[
\begin{array}{c}
#1\\
#2\\
\end{array}
\right]}
\def\vectthree#1#2#3{\left[
\begin{array}{c}
#1\\
#2\\
#3\\
\end{array}
\right]}
\def\vectfour#1#2#3#4{\left[
\begin{array}{c}
#1\\
#2\\
#3\\
#4\\
\end{array}
\right]}

\newtheorem{proposition}{Proposition}
\newtheorem{corollary}{Corollary}
\newtheorem{theorem}{Theorem}
\newtheorem{lemma}{Lemma}

\newcommand{\G} {\Gamma}
\newcommand{\Gin} {\Gamma_{in}}
\newcommand{\Gout} {\Gamma_{out}}

\title{Locally Conservative Discontinuous Petrov-Galerkin Finite Elements for
Fluid Problems}
\author{Truman Ellis} 
\author{Leszek Demkowicz}
\author{Jesse Chan}
\affil{Institute for Computational Engineering and Sciences,\\
The University of Texas at Austin, \\
Austin, TX 78712}
\date{}

\begin{document}
\maketitle

\begin{abstract}
\end{abstract}

\section{Intoduction}
Verteeg and Malalasekera, in \emph{An Introduction to Computational Fluid
Dynamics: The Finite Volume Method}\cite[p. 110-113]{IntroCFD} cite three
characteristics that they consider essential to any numerical discretization
of convection-diffusion type problems: conservativeness, boundedness, and
transportiveness.

Perot\cite{Perot2011} argues
\begin{quote}
Accuracy, stability, and consistency are the mathematical concepts that are
typically used to analyze numerical methods for partial differential equations
(PDEs). These important tools quantify how well the mathematics of a PDE is
represented, but they fail to say anything about how well the physics of the
system is represented by a particular numerical method. In practice, physical
fidelity of a numerical solution can be just as important (perhaps even more
important to a physicist) as these more traditional mathematical concepts. A
numerical solution that violates the underlying physics (destroying mass or
entropy, for example) is in many respects just as flawed as an unstable
solution.
\end{quote}

The discontinuous Petrov-Galerkin finite element method has been described as
least squares finite elements with a twist. The key difference is that least
square methods seek to minimize the residual of the solution in some Hilbert
space norm, while DPG seeks the minimization in a dual norm through the
inverse Riesz map. Exact mass conservation has been an issue that has plagued
least squares finite elements for a long time. Several approaches have been
used to try to adress this. Chang and Nelson\cite{ChangNelson1997} developed
the 'restricted LSFEM'\cite{ChangNelson1997} be augmenting the least squares
equations with a Lagrange multiplier explicitly enforcing mass conservation
element-wise. Our conservative formulation of DPG takes a similar approach and
both methods share similar negative of transforming a minimization method to a
saddle-point problem.

The discontinuous Petrov-Galerkin finite element method has shown a lot of
promise for convection-diffusion type problems including robustness in the
face of singularly perturbed problems.

\section{DPG is a Minimum Residual Method}
We now proceed with a abstract derivation of the standard Discontinuous
Petrov-Galerkin method. Suppose we have two Hilbert spaces, $U$ and $V$, the
trial and test spaces, respectively. And suppose we are trying to solve a
well-posed variational problem $b(u,v)=l(v)$. We can rewrite this in operator
form $Bu=l$, where $B:U\rightarrow V'$, where $V'$ is the dual space to $V$.
Then, for a discrete subspace $U_h\subset U$, we seek to find $u_h\in U_h$
that minimizes the error residual:
\begin{equation}
u_h=\argmin_{u_h\in U_h}\frac{1}{2}\nor{Bu_h-l}^2_{V'}\,.
\label{minresidual}
\end{equation}
Recalling that the Riesz operator $R_V:V\rightarrow V'$ is an isometry defined
by
\[
\LRa{R_Vv,\delta v}=\LRp{v,\delta v}_V,\quad\forall\delta v\in V,
\]
we can use the Riesz inverse to minimize on $V$ rather than its dual:
\begin{equation}
\frac{1}{2}\nor{Bu_h-l}^2_{V'}=\frac{1}{2}\nor{R_V^{-1}(Bu_h-l)}_V
=\frac{1}{2}\LRp{R_V^{-1}(Bu_h-l),R_V^{-1}(Bu_h-l)}_V\,.
\label{eq:rieszapplied}
\end{equation}
The first order optimality condition for \eqnref{rieszapplied} requires
the G\^ateaux derivative to be zero in all directions $\delta u \in
U_h$, i.e.,
\[
\left(R_V^{-1}(Bu_h-l),R_V^{-1}B\delta u\right)_V = 0, \quad \forall \delta u \in U. 
\]
By definition of the Riesz operator, this is equivalent to
\begin{equation}
\LRa{Bu_h-l,R_V^{-1}B\delta u_h}=0\quad\forall\delta u_h\in U_h\,.
\label{eq:DPGbilinearform}
\end{equation}
Now, we can identify $v_{\delta u_h}\coloneqq R_V^{-1}B\delta u_h$ as the
optimal test function for trial function $u_h$ and we can rewrite
\eqnref{DPGbilinearform} as
\begin{equation}
b(u_h,v_{\delta u_h})=l(v_{\delta u_h}).
\label{eq:DPGmethod}
\end{equation}
The DPG method then is to solve \eqnref{DPGmethod} with optimal test functions
$v_{\delta u_h}\in V$ that solve the auxiliary problem
\begin{equation}
\LRp{v_{\delta u_h},\delta v}_V=\LRa{R_Vv_{\delta u_h},\delta v}
=\LRa{B\delta u_h,\delta v}=b(\delta u_h,\delta v)\,,\quad\forall\delta v\in V.
\label{eq:optimaltestproblem}
\end{equation}
Using a continuous test basis would result in a global solve for every optimal
test function. Therefore DPG uses a discontinuous test basis which makes each
solve element-local and much more computationally tractable. Of course,
\eqnref{optimaltestproblem} still requires the inversion of the
infinite-dimensional Riesz map, but in approximating the $V$ by finite
dimensional $V_h$ which is of a higher polynomial degree than $U_h$ (hence
``enriched space'') works well in practice.

No assumptions have been made so far on the definition of the inner product on
$V$. In fact, proper choice of $\LRp{\cdot,\cdot}_V$ can make the difference
between a solid DPG method and one that suffers from robustness issues.

\section{DPG Applied to Convection-Diffusion}
Now that we have briefly outline the abstract DPG method, let us put it into
practice with the Convection-Diffusion equation. The strong form of the
steady Convection-Diffusion equation reads
\[
\div(\bs\beta u)-\epsilon\del u=g\,,
\]
where $u$ is the property of interest, $\bs\beta$ is the convection vector,
and $g$ is the source term. Let us write this as an equivalent system of first
order equations:
\begin{align*}
\div(\bs\beta u-\bs\sigma)&=g\\
\frac{1}{\epsilon}\bs\sigma-\grad u&=\bs0\,.
\end{align*}
If we then multiply the top equation by some scalar test function $v$ and the
bottom equation by some vector-valued test function $\tau$, we can integrate by
parts over each element $K$:
\begin{equation}
\label{eq:preultraweak}
\begin{aligned}
-(\bbeta u-\bsigma,\nabla v)_K+((\bbeta
u-\bsigma)\cdot\mathbf{n},v)_{\partial K}&=(g,v)_K\\
\frac{1}{\epsilon}(\bsigma,\btau)_K+(u,\nabla\cdot\btau)_K
-(u,\tau_n)_{\partial K}&=0\,.
\end{aligned}
\end{equation}
The discontinuous Petrov-Galerkin method refers to the fact that we are using
discontinuous optimal test functions that come from a space differing from the
trial space. It does not specify our choice of trial space. Nevertheless, many
considerations of DPG in the literature \cite{} associate DPG with the
so-called ``ultra-weak formulation.'' We will follow the same derivation for
the convection-diffusion equation, but we emphasize that other formulations
are available. Thus, we seek field variables $u\in L^2(K)$ and
$\bsigma\in\mb{L^2}(K)$. Mathematically, this leaves their traces on element
boundaries undefined, and in a manner similar to the hybridized discontinuous
Galerkin method, we define new unknowns for trace $\hat u$ and flux $\hat f$.
Applying these definitions to \eqnref{preultraweak} and adding the two
equations together, we arrive at our desired bilinear form,
\begin{equation}
\label{eq:confusionBF}
-(\bbeta u-\bsigma,\nabla v)_K+(\hat f,v)_{\partial K}
+ \frac{1}{\epsilon}(\bsigma,\btau)_K
+(u,\nabla\cdot\btau)_K
-(\hat u,\tau_n)_{\partial K}=(g,v)_K\,.
\end{equation}
% Function Spaces

Within each element, we perform a Bubnov-Galerkin solve for the optimal test
functions. %Problem description
As mentioned earlier the choice of test norm on $V$ can have profound
influence on the robustness of our method.

\bibliographystyle{plain}
\bibliography{../DPG}
\end{document}


