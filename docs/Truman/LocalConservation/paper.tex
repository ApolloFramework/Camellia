\documentclass[letterpaper]{article}
\usepackage[margin=1.0in]{geometry}
\usepackage{authblk}
\usepackage{amsmath, amssymb, amsthm, mathtools}
\usepackage{graphicx}
\usepackage{subfigure}

\def\btau{\boldsymbol\tau}
\def\bsigma{\boldsymbol\sigma}
\def\bbeta{\boldsymbol\beta}
\def\blambda{\boldsymbol\lambda}

\newcommand{\bs}[1]{\boldsymbol{#1}}
\DeclareMathOperator{\diag}{diag}
\DeclareMathOperator*{\argmin}{\arg\,\min}

\newcommand{\equaldef}{\stackrel{\mathrm{def}}{=}}

\newcommand{\tablab}[1]{\label{tab:#1}}
\newcommand{\tabref}[1]{Table~\ref{tab:#1}}

\newcommand{\theolab}[1]{\label{theo:#1}}
\newcommand{\theoref}[1]{\ref{theo:#1}}
\newcommand{\eqnlab}[1]{\label{eq:#1}}
\newcommand{\eqnref}[1]{\eqref{eq:#1}}
\newcommand{\seclab}[1]{\label{sec:#1}}
\newcommand{\secref}[1]{\ref{sec:#1}}
\newcommand{\lemlab}[1]{\label{lem:#1}}
\newcommand{\lemref}[1]{\ref{lem:#1}}

\newcommand{\mb}[1]{\mathbf{#1}}
\newcommand{\mbb}[1]{\mathbb{#1}}
\newcommand{\mc}[1]{\mathcal{#1}}
\newcommand{\norm}[1]{\left\| #1 \right\|}
\newcommand{\snorm}[1]{\left| #1 \right|}
\newcommand{\LRp}[1]{\left( #1 \right)}
\newcommand{\LRs}[1]{\left[ #1 \right]}
\newcommand{\LRa}[1]{\left\langle #1 \right\rangle}
\newcommand{\LRc}[1]{\left\{ #1 \right\}}
\newcommand{\tanbui}[2]{\textcolor{blue}{\sout{#1}} \textcolor{red}{#2}}
\newcommand{\Grad} {\ensuremath{\nabla}}
\newcommand{\Div} {\ensuremath{\nabla\cdot}}
\newcommand{\Nel} {\ensuremath{{N^\text{el}}}}
\newcommand{\jump}[1] {\ensuremath{\LRs{\![#1]\!}}}
\newcommand{\uh}{\widehat{u}}
\newcommand{\fnh}{\widehat{f}_n}
\renewcommand{\L}{L^2\LRp{\Omega}}
\newcommand{\pO}{\partial\Omega}
\newcommand{\Gh}{\Gamma_h}
\newcommand{\Gm}{\Gamma_{-}}
\newcommand{\Gp}{\Gamma_{+}}
\newcommand{\Go}{\Gamma_0}
\newcommand{\Oh}{\Omega_h}
\newcommand{\ptl}{{\partial}}
\newcommand{\bfsig}{\mbox{\boldmath $\sigma$}}
\newcommand{\bfn}{\mbox{\boldmath $n$}}
\newcommand{\bfH}{\mbox{\boldmath $H$}}
\newcommand{\HdivK}{\bfH(\text{div},K)}
\newcommand{\HOneK}{H^{-1}(K)}
\newcommand{\HOneOmegah}{H^{-1}(\Omega_h)}
\newcommand{\HdivOmegah}{\bfH(\text{div},\Omega_h)}
\newcommand{\vdeltau}{v_{\delta\bs u_h}}
\newcommand{\taudeltau}{\btau_{\delta\bs u_h}}

\newcommand{\eval}[2][\right]{\relax
  \ifx#1\right\relax \left.\fi#2#1\rvert}

\def\etal{{\it et al.~}}

\newcommand{\vect}[1]{\ensuremath\boldsymbol{#1}}
\newcommand{\tensor}[1]{\underline{\vect{#1}}}
\newcommand{\del}{\Delta}
\newcommand{\grad}{\nabla}
\newcommand{\curl}{\grad \times}
\renewcommand{\div}{\grad \cdot}
\newcommand{\ip}[1]{\left\langle #1 \right\rangle}
\newcommand{\eip}[1]{a\left( #1 \right)}
\newcommand{\pd}[2]{\frac{\partial#1}{\partial#2}}
\newcommand{\pdd}[2]{\frac{\partial^2#1}{\partial#2^2}}

\newcommand{\circone}{\ding{192}}
\newcommand{\circtwo}{\ding{193}}
\newcommand{\circthree}{\ding{194}}
\newcommand{\circfour}{\ding{195}}
\newcommand{\circfive}{\ding{196}}

\def\arr#1#2#3#4{\left[
\begin{array}{cc}
#1 & #2\\
#3 & #4\\
\end{array}
\right]}
\def\vecttwo#1#2{\left[
\begin{array}{c}
#1\\
#2\\
\end{array}
\right]}
\def\vectthree#1#2#3{\left[
\begin{array}{c}
#1\\
#2\\
#3\\
\end{array}
\right]}
\def\vectfour#1#2#3#4{\left[
\begin{array}{c}
#1\\
#2\\
#3\\
#4\\
\end{array}
\right]}

\renewcommand{\arraystretch}{2.0}

\newtheorem{proposition}{Proposition}
\newtheorem{corollary}{Corollary}
\newtheorem{theorem}{Theorem}
\newtheorem{lemma}{Lemma}

\newcommand{\G} {\Gamma}
\newcommand{\Gin} {\Gamma_{in}}
\newcommand{\Gout} {\Gamma_{out}}

\title{Locally Conservative Discontinuous Petrov-Galerkin Finite Elements for
Fluid Problems}
\author{Truman Ellis} 
\author{Leszek Demkowicz}
\author{Jesse Chan}
\affil{Institute for Computational Engineering and Sciences,\\
The University of Texas at Austin, \\
Austin, TX 78712}
\date{}

\begin{document}
\maketitle

\begin{abstract}
\end{abstract}

\section{Intoduction}
The discontinuous Petrov-Galerkin (DPG) methd with optimal test functions has
been under active development for convection-diffusion type systems
\cite{DPG1, DPG2, DPG3, DPG5, DemkowiczHeuer, ChanHeuerThanhDemkowicz2012,
MoroNguyenPeraire11}. In this paper, we develop a locally conservative
formulation of DPG for convection-diffusion type equations and supplement this
with numerical results. 

\subsection{Importance of Local Conservation}
Locally conservative methods hold a special place for numerical analysts in
the field of fluid dynamics. 
Perot\cite{Perot2011} argues
\begin{quote}
Accuracy, stability, and consistency are the mathematical concepts that are
typically used to analyze numerical methods for partial differential equations
(PDEs). These important tools quantify how well the mathematics of a PDE is
represented, but they fail to say anything about how well the physics of the
system is represented by a particular numerical method. In practice, physical
fidelity of a numerical solution can be just as important (perhaps even more
important to a physicist) as these more traditional mathematical concepts. A
numerical solution that violates the underlying physics (destroying mass or
entropy, for example) is in many respects just as flawed as an unstable
solution.
\end{quote}

The discontinuous Petrov-Galerkin finite element method has been described as
least squares finite elements with a twist. The key difference is that least
square methods seek to minimize the residual of the solution in some Hilbert
space norm, while DPG seeks the minimization in a dual norm through the
inverse Riesz map. Exact mass conservation has been an issue that has plagued
least squares finite elements for a long time. Several approaches have been
used to try to adress this. Chang and Nelson\cite{ChangNelson1997} developed
the 'restricted LSFEM'\cite{ChangNelson1997} by augmenting the least squares
equations with a Lagrange multiplier explicitly enforcing mass conservation
element-wise. Our conservative formulation of DPG takes a similar approach and
both methods share similar negative of transforming a minimization method to a
saddle-point problem.

\subsection{DPG is a Minimum Residual Method}
Roberts \etal presents a brief history and derivation of DPG with optimal test functions in
\cite{DPGStokes}. We follow his derivation of the standard DPG method as a
minumum residual method. Let $U$ be the trial Hilbert space and $V$ the test
Hilbert space for a well-posed variational problem $b(u,v)=l(v)$. In operator
form this is $Bu=l$, where $B:U\rightarrow V'$. We seek to minimize the
residual for the discrete space $U_h\subset U$:
\begin{equation}
u_h=\argmin_{u_h\in U_h}\frac{1}{2}\norm{Bu_h-l}^2_{V'}\,.
\label{minresidual}
\end{equation}
Recalling that the Riesz operator $R_V:V\rightarrow V'$ is an isometry defined
by
\[
\LRa{R_Vv,\delta v}=\LRp{v,\delta v}_V,\quad\forall\delta v\in V,
\]
we can use the Riesz inverse to minimize on $V$ rather than its dual:
\begin{equation}
\frac{1}{2}\norm{Bu_h-l}^2_{V'}=\frac{1}{2}\norm{R_V^{-1}(Bu_h-l)}^2_V
=\frac{1}{2}\LRp{R_V^{-1}(Bu_h-l),R_V^{-1}(Bu_h-l)}_V\,.
\label{eq:rieszapplied}
\end{equation}
The first order optimality condition for \eqnref{rieszapplied} requires
the G\^ateaux derivative to be zero in all directions $\delta u \in
U_h$, i.e.,
\[
\left(R_V^{-1}(Bu_h-l),R_V^{-1}B\delta u\right)_V = 0, \quad \forall \delta u \in U. 
\]
By definition of the Riesz operator, this is equivalent to
\begin{equation}
\LRa{Bu_h-l,R_V^{-1}B\delta u_h}=0\quad\forall\delta u_h\in U_h\,.
\label{eq:DPGbilinearform}
\end{equation}
Now, we can identify $v_{\delta u_h}\coloneqq R_V^{-1}B\delta u_h$ as the
optimal test function for trial function $u_h$ and we can rewrite
\eqnref{DPGbilinearform} as
\begin{equation}
b(u_h,v_{\delta u_h})=l(v_{\delta u_h}).
\label{eq:DPGmethod}
\end{equation}
The DPG method then is to solve \eqnref{DPGmethod} with optimal test functions
$v_{\delta u_h}\in V$ that solve the auxiliary problem
\begin{equation}
\LRp{v_{\delta u_h},\delta v}_V=\LRa{R_Vv_{\delta u_h},\delta v}
=\LRa{B\delta u_h,\delta v}=b(\delta u_h,\delta v)\,,\quad\forall\delta v\in V.
\label{eq:optimaltestproblem}
\end{equation}
Using a continuous test basis would result in a global solve for every optimal
test function. Therefore DPG uses a discontinuous test basis which makes each
solve element-local and much more computationally tractable. Of course,
\eqnref{optimaltestproblem} still requires the inversion of the
infinite-dimensional Riesz map, but approximating $V$ by finite
dimensional $V_h$ which is of a higher polynomial degree than $U_h$ (hence
``enriched space'') works well in practice.

No assumptions have been made so far on the definition of the inner product on
$V$. In fact, proper choice of $\LRp{\cdot,\cdot}_V$ can make the difference
between a solid DPG method and one that suffers from robustness issues.

\section{Analysis}
We now proceed to develop a locally conservative formulation of DPG for
convection-diffusion type problems, but there are a few terms that we need to
define first. If $\Omega$ is our problem domain, then we can partition it into
finite elements $K$ such that
\[
\overline{\Omega} = \bigcup_K  \bar{K},\: \quad K \text { open},
\]
with corresponding {\em skeleton} $\Gamma_h$ and {\em interior
  skeleton} $\Gamma_h^0$,
\[
\Gamma_h := \bigcup_K \partial K\qquad \Gamma_h^0 := \Gamma_h - \Gamma.
\]
We define broken Sobolev spaces element-wise:
\[
\begin{array}{rl}
H^1(\Omega_h) & := \prod_K H^1(K), \\[8pt]
\bfH(\text{div},\Omega_h) & := \prod_K \bfH(\text{div},K).
\end{array}
\]
We also need the trace spaces:
\[
\begin{array}{rl}
H^\frac{1}{2}(\Gamma_h) & := \left\{ \hat{v} = \{\hat{v}_K \} \in \prod_K H^{1/2}(\ptl K) \: :
\: \exists v \in H^1(\Omega) : v\vert_{\ptl K} = \hat{v}_K \right\}, \\[8pt]
H^{-\frac{1}{2}}(\Gamma_h) & := \left\{ {\hat{\sigma}}_n = \{ {\hat{\sigma}}_{Kn} \}\in \prod_K H^{-1/2}(\ptl K) \: : \: \exists \bfsig \in \bfH(\text{div},\Omega)
: {\hat{\sigma}}_{Kn} = (\bfsig \cdot \bfn)\vert_{\ptl K} \right\},
\end{array}
\]
which are developed more precisely in \cite{DPGStokes}.

\subsection{Element Conservative Convection-Diffusion}
Now that we have briefly outlined the abstract DPG method, let us apply it to
the convection-diffusion equation. The strong form of the steady
convection-diffusion problem with homogeneous Dirichlet boundary conditions reads
\[
\left\{
\begin{array}[c]{rrl}
\div(\bs\beta u)-\epsilon\del u & =f & \text{in }\Omega\\
u & =0 & \text{on }\Gamma\,,
\end{array}
\right.
\]
where $u$ is the property of interest, $\bs\beta$ is the convection vector,
and $f$ is the source term. Nonhomogeneous Dirichlet and Neumann boundary
conditions are straightforward but would add technicality to the following
discussion. Let us write this as an equivalent system of first
order equations:
\begin{align*}
\div(\bs\beta u-\bs\sigma)&=f\\
\frac{1}{\epsilon}\bs\sigma-\grad u&=\bs0\,.
\end{align*}
If we then multiply the top equation by some scalar test function $v$ and the
bottom equation by some vector-valued test function $\tau$, we can integrate by
parts over each element $K$:
\begin{equation}
\label{eq:preultraweak}
\begin{aligned}
-(\bbeta u-\bsigma,\nabla v)_K+((\bbeta
u-\bsigma)\cdot\mathbf{n},v)_{\partial K}&=(f,v)_K\\
\frac{1}{\epsilon}(\bsigma,\btau)_K+(u,\nabla\cdot\btau)_K
-(u,\tau_n)_{\partial K}&=0\,.
\end{aligned}
\end{equation}
The discontinuous Petrov-Galerkin method refers to the fact that we are using
discontinuous optimal test functions that come from a space differing from the
trial space. It does not specify our choice of trial space. Nevertheless, many
considerations of DPG in the literature \cite{} associate DPG with the
so-called ``ultra-weak formulation.'' We will follow the same derivation for
the convection-diffusion equation, but we emphasize that other formulations
are available. Thus, we seek field variables $u\in L^2(K)$ and
$\bsigma\in\mb{L^2}(K)$. Mathematically, this leaves their traces on element
boundaries undefined, and in a manner similar to the hybridized discontinuous
Galerkin method, we define new unknowns for trace $\hat u$ and flux $\hat t$.
Applying these definitions to \eqnref{preultraweak} and adding the two
equations together, we arrive at our desired variational problem. 

Find
$\bs u:=(u,\bsigma,\hat u,\hat t)
\in\bs U:=L^2(\Omega_h)\times \bs L^2(\Omega_h)\times H^{1/2}(\Gamma_h)\times H^{-1/2}(\Gamma_h)$ 
such that
\begin{equation}
\label{eq:variationalFormulation}
-(\bbeta u-\bsigma,\nabla v)_K+(\hat t,v)_{\partial K}
+ \frac{1}{\epsilon}(\bsigma,\btau)_K
+(u,\nabla\cdot\btau)_K
-(\hat u,\tau_n)_{\partial K}=(f,v)_K \\
\end{equation}
for all $\bs v:=(v,\btau)\in
\bs V:=H^1(\Omega_h)\times\bfH(\text{div},\Omega_h)$. Thus, our bilinear form
is $b(\bs u,\bs v)=l(\bs v)$.

Let $\bs U_h:=U_h\times\bs S_h\times\hat U\times\hat F\subset L^2(\Omega_h)\times\bs
L^2(\Omega_h)\times H^{\frac{1}{2}}(\Gamma_h)\times H^{-\frac{1}{2}}(\Gamma_h)$
be a finite-dimensional subspace, and let $\bs u_h:=(u_h.\bsigma_h,\hat
u_h\hat t_h)\in\bs U_h$ be a group variable. The element conservative DPG scheme is
derived from the Lagrangian:
\begin{equation}
L(\bs u_h,\lambda_k)=\frac{1}{2}\norm{R_V^{-1}(b(\bs
u_h,\cdot)-(f,\cdot))}^2_{\bs V}-\sum_K\lambda_K(b(\bs u_h,1_K)-l(1_K))\,.
\label{eq:lagrangian}
\end{equation}

Taking the G\^ateaux derivative as before, we arrive at the following system
of equations:
\begin{equation}
\left\{
\begin{array}[c]{rll}
b(\bs u_h,T(\delta\bs u_h))-\sum_K\lambda_K b(\bs u_h,1_K)
&=(f,T(\delta\bs u_h)) & \forall\delta\bs u_h\in\bs U_h\\
b(\bs u_h,1_K) &=(f,1_K) & \forall K\,,
\end{array}
\right.
\label{eq:conservativeSystem}
\end{equation}
where $T:=R_V^{-1}B:\bs U_h\rightarrow\bs V$ is the trial-to-test operator.

Since the trial-to-test operator selectively produces test functions in
$H^1(\Omega_h)$ and $\bs H(\text{div},\Omega_h)$, denote the result
$v_{\delta\bs u_h}$ when $T(\delta u_h)\in H^1(\Omega_h)$ and
$\btau_{\delta\bs u_h}$ when $T(\delta u_h)\in \bs H(\text{div},\Omega_h)$.
Then, putting \ref{eq:conservativeSystem} into more concrete terms for
convection-diffusion, we get:
\begin{equation}
\left\{
\begin{array}[c]{rll}
-(\bbeta u-\bsigma,\nabla \vdeltau)+\langle\hat t,\vdeltau\rangle
+ \frac{1}{\epsilon}(\bsigma,\taudeltau)
+(u,\nabla\cdot\taudeltau)
-\langle\hat u,\taudeltau\cdot\bs n\rangle\\
-\sum_K\lambda_K (\delta\hat t,1_K)
&=(f,\vdeltau) & \forall\delta\bs u_h\in\bs U_h\\
b(\bs u_h,1_K) &=(f,1_K) & \forall K\,.
\end{array}
\right.
\label{eq:conservativeConfusionSystem}
\end{equation}

The optimal test functions are determined by solving local problems determined
by the choice of test norm. For illustrative purposes we choose the robust
test norm developed by Chan \etal\cite{ChanHeuerThanhDemkowicz2012}. With this
choice, we get our local problem:

Find $\vdeltau\in\HOneK,\,\taudeltau\in\HdivK$ such that:
\begin{align}
(\nabla\cdot\taudeltau,\nabla\cdot\delta\btau)_K
&+\min\left\{\frac{1}{\epsilon},\frac{1}{|K|}\right\}(\taudeltau,\delta\btau)_K
+\epsilon(\nabla\vdeltau,\nabla \delta v)_K
+(\bbeta\cdot\nabla\vdeltau,\bbeta\cdot\nabla \delta v)_K\\
&+\alpha\min\left\{\sqrt{\frac{\epsilon}{|K|}},1\right\}(\vdeltau,\delta v)_K
\quad\quad\forall\delta v\in\HOneK,\,\delta\btau\in\HdivK\,,
\label{eq:localSolve}
\end{align}
where, typically, $\alpha=1$.

We can decompose the discrete flux space $\hat F_h$ into element conserving
fluxes and some algebraic complement:
\begin{equation}
\hat F_h=\hat F_h^c\oplus\hat F_h^{nc}\,,\quad\hat 
F_h^{c}:=\{\hat t:\langle\hat t,1_K\rangle=0\quad\forall K\}\,.
\label{eq:decomposition}
\end{equation}
This may be accomplished by employing for fluxes in $\hat F_h^{c}$ traces of
curls, $\nabla\times\phi$. Restricting ourselves to $\delta\hat t\in\hat
F_h^{c}$ in \eqref{eq:conservativeConfusionSystem}$_1$, we obtain a system  of
equations that can be solved for $u_h$, $\bsigma_h$, $\hat u$, and $\hat
t\in\hat F_h^c$. The remaining fluxes, $\hat t\in\hat F_h^{nc}$ are determined
from \eqref{eq:conservativeConfusionSystem}$_2$. Finally, testing with $\hat
t\in\hat F_h^{nc}$ in \eqref{eq:conservativeConfusionSystem}$_1$, we obtain a
system of equations that can be solved for Lagrange multipliers $\lambda_K$.
Additionally, we can pass in local problem \eqref{eq:localSolve} with
$\alpha\rightarrow0$. The fact that the test functions will be determined then
up to a constant does not matter, for $\delta\hat t\in\hat F_h^c$, equation
\eqref{eq:conservativeConfusionSystem}$_1$ is orthogonal to constants. 
It is convenient to be able to take $\alpha\rightarrow0$ as we will see in
some later numerical experiments.

\subsection{Proof of Convergence}
We cannot directly use Brezzi's theorem, but we can reproduce his argument.
First of all, variational formulation \eqref{eq:variationalFormulation} is
equivalent to:

Find $\bs u\in\bs U$ such that
\begin{equation}
b(\bs u,T(\delta\bs u))=(f,T(\delta\bs u))\quad\forall\bs v\in\bs V\,.
\label{eq:bilinearForm}
\end{equation}
This is because the trial-to-test operator $T=R_V^{-1}B$ is an isomorphism.

Restricting ourselves to $\delta\hat t_h\in\hat F_h^c$, and subtracting
\eqref{eq:bilinearForm} from \eqref{eq:conservativeConfusionSystem}, we have
the Galerkin orthogonality condition:
\begin{align}
\label{eq:galerkinOrthogonality}
\frac{1}{\epsilon}(\bsigma_h-\bsigma,\taudeltau)
&+(u_h-u,\nabla\cdot\taudeltau)
-\langle\hat u_h-\hat u,\taudeltau\cdot\bs n\rangle
\nonumber\\
&-(\bbeta(u_h-u)-(\bsigma_h-\bsigma),\nabla\vdeltau)
+\langle\hat t_h-\hat t,\vdeltau\rangle
\\
&\hspace{25ex}\forall(\delta u_h,\delta\bsigma_h,\delta\hat u_h,\delta\hat t_h)
\in U_h\times\bs S_h\times\hat U_h\times\hat F_h^c\,.
\nonumber
\end{align}

We begin with an equivalent of Brezzi's \emph{discrete inf-sup in kernel
condition}.
\begin{lemma}
\label{lemma:infsuplemma}
The following condition holds:
\begin{align}
\sup_{\stackrel{
(\delta u_h,\delta\bsigma_h,\delta\hat u,\delta\hat t)\in}
{U_h\times\bs S_h\times\hat U_h\times\hat F_h^c}
}
\frac{\snorm{
b((u_h,\bsigma_h,\hat u_h,\hat t_h), (\vdeltau,\taudeltau))
% \frac{1}{\epsilon}(\bsigma_h,\taudeltau)
% +(u_h,\nabla \cdot\taudeltau)
% -\langle\hat u_h,\taudeltau\cdot\bs n\rangle
% -(\bbeta u_h-\bsigma_h,\nabla\vdeltau)
% +\langle\hat t_h,\vdeltau\rangle
}}
{\norm{(\vdeltau,\taudeltau)}}
\ge\alpha\norm{(u_h,\bsigma_h,\hat u_h,\hat t_h)}\quad\nonumber\\
\forall(u_h,\bsigma_h,\hat u_h,\hat t_h)
\in U_h\times\bs S_h\times\hat U_h\times\hat F_h^c\,.
\label{eq:infsup}
\end{align}

\end{lemma}
\begin{proof}
By the very construction of the trial-to-test operator, the supremum on the
left is equal to
\begin{equation}
\norm{T(u_h,\bsigma_h,\hat u_h,\hat t_h)}_{V}
=\norm{R_V^{-1}B(u_h,\bsigma_h,\hat u_h, \hat t_h)}_{V}
=\norm{B(u_h,\bsigma_h,\hat u_h, \hat t_h)}_{V'}
\label{eq:infsupproof}
\end{equation}
The condition follows now from the fact that operator $B$ is bounded below,
i.e. bilinear form \linebreak$b((u_h,\bsigma_h,\hat u_h, \hat t_h),(v,\btau))$ satisfies
the inf-sup condition.
\end{proof}

Assume additionally that flux decomposition \eqref{eq:decomposition} is
orthogonal. We begin by decomposing $\hat t_h$,
\begin{equation}
\hat t_h=\hat t_h^{c}+\hat t_h^{nc}\,.
\label{eq:fluxdecomposition}
\end{equation}
Next, we use the triangle inequality,
\begin{align}
\label{eq:triangle}
\norm{(u-u_h,\bsigma-\bsigma_h,\hat u-\hat u_h, \hat t-\hat t_h)}&=
\norm{(u-u_h,\bsigma-\bsigma_h,\hat u-\hat u_h, \hat t-(\hat t_h^c+\hat t_h^{nc}))}\nonumber\\
&\le\norm{(u-w_h,\bsigma-\bs s_h,\hat u-\hat w_h, \hat t-(\hat r_h^c+\hat t_h^{nc}))}\\
&\hspace{3ex}+\norm{(w_h-u_h,\bs s_h-\bsigma_h,\hat w_h-\hat u_h, \hat r_h^c-\hat t_h^c)}\nonumber
\end{align}
where $w_h\in U_h$, $\bs s_h\in\bs S_h$, $\hat w_h\in\hat U_h$, and $\hat
r_h^{c}\in\hat F_h$ are arbitrary. The inf-sup in kernel condition now
implies:
\begin{align}
\label{eq:estimate}
&\norm{(w_h-u_h,\bs s_h-\bsigma_h,\hat w_h-\hat u_h, \hat r_h^c-\hat t_h^c)}\\
&\hspace{8ex}\le\frac{1}{\alpha}
\sup_{\stackrel{
(\delta u_h,\delta\bsigma_h,\delta\hat u,\delta\hat t)\in}
{U_h\times\bs S_h\times\hat U_h\times\hat F_h^c}
}
\frac{\snorm{
b( (w_h-u_h,\bs s_h-\bsigma_h,\hat w_h-\hat u_h,\hat r_h^c-\hat t_h^c),
(\vdeltau,\taudeltau))
}}
{\norm{(\vdeltau,\taudeltau)}}\nonumber\\
&\hspace{8ex}\le\frac{1}{\alpha}
\sup_{\stackrel{
(\delta u_h,\delta\bsigma_h,\delta\hat u,\delta\hat t)\in}
{U_h\times\bs S_h\times\hat U_h\times\hat F_h^c}
}
\frac{\snorm{
b( (w_h-u_h,\bs s_h-\bsigma_h,\hat w_h-\hat u_h,\hat r_h^c+\hat t_h^{nc}-\hat t),
(\vdeltau,\taudeltau))
}}
{\norm{(\vdeltau,\taudeltau)}}\nonumber\\
&\hspace{8ex}\le\frac{1}{\alpha}
\norm{(w_h-u_h,\bs s_h-\bsigma_h,\hat w_h-\hat u_h,\hat t-(\hat r_h^c+\hat
t_h^{nc}))}
\end{align}
% \frac{1}{\epsilon}(\bs s_h-\bsigma_h,\taudeltau)
% +(w_h-u_h,\nabla \cdot\taudeltau)
% -\langle\hat w_h-\hat u_h,\taudeltau\cdot\bs n\rangle
% -(\bbeta (w_h-u_h)-(\bs s_h-\bsigma_h),\nabla\vdeltau)
% +\langle\hat r_h^c-\hat t_h^c,\vdeltau\rangle
where the second step follows from the Galerkin orthogonality condition
\eqref{eq:galerkinOrthogonality}, and the last step is a consequence of the
minimum energy extension norm for fluxes. Cobining with \eqref{eq:triangle},
we get the final stability result:
\begin{equation}
\norm{(u-u_h,\bsigma-\bsigma_h,\hat u-\hat u_h,\hat t-\hat t_h)}
\leq\left(1+\frac{1}{\alpha}\right)
\inf_{\stackrel{
(w_h,\bs s_h,\hat w,\hat r_h^c)\in}
{U_h\times\bs S_h\times\hat U_h\times\hat F_h^c}
}
\norm{(u-w_h,\bsigma-\bs s_h,\hat u-\hat w,\hat t-(\hat r_h^c+\hat t_h^{nc}))}
\,.
\label{eq:stability}
\end{equation}

\section{Numerical Experiments}

% A cursory look at \eqnref{confusionBF} tells us which function spaces to look
% in for each variable:
% \[
% \begin{array}{lll}
% u\in L^2(\Omega_h) \quad& \hat u\in H^{1/2}(\Gamma_h) \quad& v\in H^1(\Omega_h)\\
% \bsigma\in \mathbf{L}^2(\Omega_h) \quad& \hat t\in H^{-1/2}(\Gamma_h) \quad&
% \btau\in \bs{H}(\bs{div},\Omega_h)\,.\\
% \end{array}
% \]

% All that is left to pin down this problem is a definition of our test norm so
% we can invert the Riesz operator and calculate our optimal test functions.
% Within each element, we perform a Bubnov-Galerkin solve for the optimal test
% functions. Define finite-dimensional subspaces
% $\mb{U}_h\subset\mb{U}:=L^2(\Omega_h)\times\mb{L}^2(\Omega_h)\times
% H^{1/2}(\Gamma_h)\times H^{-1/2}(\Gamma_h)$ the trial space and 
% $\mb{V}_h\subset\mb{V}:=H^1(\Omega_h)\times H(div,\Omega_h)$ the ``enriched''
% test space.
% For each $\mathbf{u}=\{u,\bsigma,\hat u,\hat t\}\in\mathbf{U}_h$, find
% $\mathbf{v}_{\mathbf{u}}=\{v_\mathbf{u},\btau_\mathbf{u}\}\in\mathbf{V_h}$ such that
% \[
% (\mathbf{v_u},\mathbf{w})_\mathbf{V}=b(\mathbf{u},\mathbf{w})\quad\forall\mathbf{w}\in\mathbf{V}
% \]
% As mentioned earlier the choice of test norm on $V$ can have profound
% influence on the robustness of our method. Unfortunately, the structure of the
% optimal test norm makes it non-localizable. For many problems, it suffices to
% use the so-call quasi-optimal test norm which is based on the adjoint of the
% $B$ operator, but for convection-diffusion type equations, the adjoint
% develops boundary layers which make solving for the optimal test functions
% much more difficult for small diffusion. In an earlier consideration of DPG
% for convection-diffusion problems, Chan \etal developed the more robust
% test norm\cite{ChanHeuerThanhDemkowicz2012}, 
% \begin{equation}
% \label{eq:robustNorm}
% \norm{(v,\btau)}^2_{\mathbf{V},\Omega_h}=
% \norm{\nabla\cdot\btau}^2+\norm{\min\left\{\frac{1}{\sqrt{\epsilon}},\frac{1}{\sqrt{|K|}}\right\}\btau}^2
% +\epsilon\norm{\nabla v}^2+\norm{\bbeta\cdot\nabla v}^2
% +\norm{\min\left\{\sqrt{\frac{\epsilon}{|K|}},1\right\}v^2}\,.
% \end{equation}
% Unfortunately, this test norm also has a few issues. For one, some of the
% assumptions that went into its development break down as the flow field
% degenerates to zero. In such cases the final $L^2$ term on $v$ can register
% higher error levels and trigger unnecessary refinements in smooth regions, see
% ref{}.
%TODO: have Jesse write about this

% \subsection{Locally Conservative Formulation}
% A simple control volume analysis will tell us that a locally conservative
% method must enforce that
% \begin{equation}
% \int_{\partial K}\hat t=\int_K f\,,\quad\forall K\in\Omega_h\,,
% \label{eq:localconservation}
% \end{equation}
% which is equivalent to having the set
% $\mathbf{v}_K:=\{v,\btau\}=\{1_K,\boldsymbol0\}$ for $K=1,\dots,N$ ($N$ is the
% number of mesh elements)
% in the test space, where each $1_K$ has value one on element $K$ and zero
% elsewhere.  In fact, if we insert this test function into
% \eqnref{confusionBF}, all of the $\tau$ and $\grad v$ terms vanish and we are
% left exactly with this condition.
% Numerical experiments imply that local conservation occurs in one dimension,
% but the standard DPG method is not exactly locally conservative for higher
% dimensional problems.
% 
% Following Moro \etal\cite{MoroNguyenPeraire11}, we can explicitly augment our
% test space with constants through the use of Lagrange multipliers. Going back
% to \eqnref{rieszapplied}, we can define our Lagrange function,
% \begin{equation}
% \label{eq:lagrangian}
% L(u_h,\blambda) = \frac{1}{2}\norm{R_V^{-1}(Bu_h-l)}_V^2
% -\sum_K\lambda_K\underbrace{\langle Bu_h-l,\mathbf{v}_K\rangle}_
% {\langle\hat t, 1_K\rangle_{\partial K}-\langle g,1_K\rangle_K}\,,
% \end{equation}
% where $\blambda=\{\lambda_1,\cdots,\lambda_N\}$.
% We then proceed as before and find the critical points of \eqnref{lagrangian},
% \begin{equation}
% \label{eq:modifiedBF}
% \frac{\partial L(u_h,\blambda)}{\partial u_h}=b(u_h,R_V^{-1}B\delta u_h)
% -l(R_V^{-1}B\delta u_h)
% -\sum_K\lambda_K b(\delta
% u_h,\mathbf{v}_K)=0\,,\quad\forall\delta u_h\in U_h
% \end{equation}
% \begin{equation}
% \label{eq:constraint}
% \frac{\partial
% L(u_h,\blambda)}{\partial\lambda_K}=-b(u_h,\mathbf{v}_K)+l(\mathbf{v}_K)=0\,,\quad\forall
% K\,.
% \end{equation}
% Equation \eqnref{modifiedBF} is just \eqnref{DPGmethod} with the extra
% Lagrange terms. As usual, the second equation just enforces the constraint. As
% a consequence, we now explicitly have constants in our test space and should
% enforce local conservation to machine precision. The negative side is that we
% have added an additional unknown associated with every mesh element and turned
% our well-behaved minimization into a saddle point problem.
% 
% This change has further consequences to how we compute our optimal test
% functions. Since constants are now explicitly represented in the test space,
% we only need to search for optimal test functions in the orthogonal complement
% of constants. The final term in \eqnref{robustNorm} is somewhat troublesome,
% but becomes unnecessary when solving in the orthogonal complement of
% constants. Instead, we can replace it with a much nicer zero mean term. Thus,
% \eqnref{robustNorm} becomes
% \begin{equation}
% \norm{(v,\btau)}^2_{\mathbf{V},\Omega_h}=
% \norm{\nabla\cdot\btau}^2+\norm{\min\left\{\frac{1}{\sqrt{\epsilon}},\frac{1}{\sqrt{|K|}}\right\}\btau}^2\\
% +\epsilon\norm{\nabla v}^2+\norm{\bbeta\cdot\nabla
% v}^2+\left(\int_Kv\right)^2\,.
% \end{equation}

\bibliographystyle{plain}
\bibliography{../DPG}
\end{document}
