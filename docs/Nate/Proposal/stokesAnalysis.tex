\section{Analysis of the Velocity-Gradient-Pressure Stokes Formulation}
\label{sec:StokesAnalysis}

The purpose of this appendix is to present a general theory for the DPG
method for linear PDEs and immediately specialize it to the Stokes
problem. The presented theory summarizes results obtained in
\cite{DPG6,BramwellDemkowiczGopalakrishnanQiu11,DemkowiczGopalakrishnanMugaZitelli12}
for particular boundary-value problems and specializes the
general theory for Friedrichs systems, presented in
\cite{Bui-ThanhDemkowiczGhattas11b}, for cases in which the traces of the graph spaces
are available.  As most of the technical results here are classical,
we have chosen a somewhat informal format of presentation.


\subsection{Notation and Definitions\label{section:notation}}


Let $\Omega$ denote a bounded Lipschitz domain in $\doubleIR^n, n=2,3$,\: with boundary
$\Gamma = \ptl \Omega$. We shall use the following standard energy spaces:
\[
\begin{array}{rl}
H^1(\Omega) & := \{ u \in L^2(\Omega) \: : \: \bfnab u \in \bfL^2(\Omega) \}, \\[8pt]
\bfH(\text{div},\Omega) & := \{ \bfsig \in \bfL^2(\Omega) \: : \: \text{div} \bfsig \in L^2(\Omega) \}, 
\end{array}
\]
with the corresponding trace spaces on $\Gamma$:
\[
\begin{array}{rl}
H^{1/2}(\Gamma) & := \{ 
{\hat{u} =} u\vert_{\Gamma}, \: u \in H^1(\Omega) \},
\\[8pt] H^{-1/2}(\Gamma) & := \{ {\hat{\sigma}_{n}} = (\bfsig \cdot
\bfn)\vert_{\Gamma} ,\: \bfsig \in \bfH(\text{div},\Omega) \},
\end{array}
\] 
where $\bfn$ denotes the outward normal unit vector to the boundary
$\Gamma$.  The definition of trace space $H^{1/2}(\Gamma)$ is
classical but far from trivial, see e.g.  \cite[pp. 96]{McLean}.  The
assumption on the regularity of the domain (being Lipschitz) is
essential; domains with ``cracks'' require a special and
non-classical treatment. When necessary, we will formalize the use of
the trace operator by identifying it with a separate symbol, 
\[
\text{tr} \: : \: H^1(\Omega) \ni u \to \text{tr } u = {\hat{u} =} u\vert_{\Gamma}
\in H^{1/2}(\Gamma).  
\] 
The space $H^{-1/2}(\Gamma)$ is the topological
dual of $H^{1/2}(\Gamma)$ and a second trace operator corresponding to it,
\[ \text{tr} \: : \: \bfH(\text{div},\Omega) \ni \bfsig \to \text{tr
} \bfsig = {\hat{\sigma}_n} = (\bfsig \cdot \bfn)\vert_{\Gamma} \in
H^{-1/2}(\Gamma),
\] 
is usually defined by the {\em generalized Green
  Formula}; see e.g. \cite[pp. 61]{Showalter} or
\cite[pp. 530]{FAbook}. Note that, unless otherwise stated, in this
paper we use the same trace notation ``tr'' for functions in both
$H^1(\Omega)$ and $\bfH(\text{div},\Omega)$.

We shall also use group variables consisting of multiple copies of functions from $H^1(\Omega),\bfH(\text{div},\Omega)$, and $H^{1/2}(\Gamma)$ or distributions from $H^{-1/2}(\Gamma)$. We will then switch to boldface notation:
\[
\begin{array}{rl}
\bfH^1(\Omega) & = H^1(\Omega) \times \ldots \times H^1(\Omega),\\[8pt]
\bfH^{1/2}(\Gamma) & = H^{1/2}(\Gamma) \times \ldots \times H^{1/2}(\Gamma), \\[8pt]
\bfH^{-1/2}(\Gamma) & = H^{-1/2}(\Gamma) \times \ldots \times H^{-1/2}(\Gamma),
\end{array}
\]
etc. In the case of tensors, the definitions will be applied row-wise:
$$
\bfsig = (\sigma_{ij}) \in \bfH(\text{\bf div},\Omega) \quad \Longleftrightarrow \quad
(\sigma_{i1},\ldots,\sigma_{in}) \in \bfH(\text{div},\Omega),\: i=1,\ldots,n.
$$

\paragraph*{Broken energy spaces.}

Let $\Omega$ be partitioned into finite elements $K$ such that
\[
  \overline{\Omega} = \bigcup_K  \bar{K},\: \quad K \text { open},
\]
with corresponding {\em skeleton} $\Gamma_h$ and {\em interior
  skeleton} $\Gamma_h^0$,
\[
\Gamma_h := \bigcup_K \ptl K\qquad \Gamma_h^0 := \Gamma_h - \Gamma.
\]
The usual regularity assumptions for the elements can essentially be relaxed. 
The elements may be general polygons in 2D, or polyhedra\footnote{Possibly curvilinear polyhedra.}
 in 3D (with triangular and quadrilateral
faces). Meshes may be {\em irregular}, i.e. with hanging nodes (see e.g. \cite[pp. 211]{hpbook}).
Also, at this point, we do not make any shape regularity assumptions.
By {\em broken} energy spaces we simply mean standard energy spaces defined element-wise:
\[
\begin{array}{rl}
H^1(\Omega_h) & := \prod_K H^1(K), \\[8pt]
\bfH(\text{div},\Omega_h) & := \prod_K \bfH(\text{div},K).
\end{array}
\]
With broken energy spaces, integration by parts is performed element-wise. 
For $\bfsig \in \bfH(\text{div},\Omega_h)$ and $v \in H^1(\Omega)$, we have
\[
\begin{array}{rl}
(\text{div}_h \bfsig, v)_{\Omega_h}  := &\ds \sum_K (\text{div} \bfsig, v)_K \\[8pt]
 = &\ds \sum_K \left( -(\bfsig, \bfnab v)_K 
+ {\langle\hat{\sigma}_n,\hat{v}\rangle _{\ptl K}} \right) \\[8pt]
 = &\ds - (\bfsig, \bfnab v) + {\underbrace{\sum_K 
\langle\hat{\sigma}_n,\hat{v}\rangle_{\ptl K}}
_{=: \langle\hat{\sigma}_n,\hat{v}\rangle_{\Gamma_h}}.}
\end{array}
\]
Here $(\cdot,\cdot)$ and $(\cdot,\cdot)_K$ denote the $L^2$-product over
the whole domain and element $K$, resp., and $\langle\cdot,\cdot\rangle_{\ptl K}$
stands for the duality pairing between $H^{-1/2}(\ptl K)$ and
$H^{1/2}(\ptl K)$.


Integration by parts leads naturally to the concept of the trace space over the skeleton $\Gamma_h$,
\[
H^{1/2}(\Gamma_h) := 
{
\left\{ \hat{v} = \{\hat{v}_K \} \in \prod_K H^{1/2}(\ptl K) \: :
\: \exists v \in H^1(\Omega) : v\vert_{\ptl K} = \hat{v}_K \right\}.}
\]
This is not a trivial definition. First of all, to be more precise, by
${v}\vert_{\ptl K}$ we mean the trace (for element $K)$
of the restriction of ${v}$ to $K$. Secondly,
$H^{1/2}(\Gamma_h)$ is a {\em closed} subspace of $\prod_K
H^{1/2}(\ptl K)$, as we shall show momentarily. For convenience, we
assume that all trace spaces are endowed with {\em minimum-energy
  extension norms}, i.e.,
\[
 \Vert {\hat{u}}\Vert_{H^{1/2}(\ptl K)} :=
\inf_{\substack{
E{\hat{u}} \in H^1(K)\\ 
E{\hat{u}}\vert_{\ptl K} = {\hat{u}}}}
\Vert E{\hat{u}} \Vert_{H^1(K)},
\]
etc. Let 
${\hat{u}}^n = \{ {\hat{u}}^n_K \} $ be a sequence of functions in
$H^{1/2}(\Gamma_h)$ converging in the product space to a limit ${\hat{u}} =
\{{\hat{u}}_K \} $. For each element $K$, ${\hat{u}}^n_K$ 
is the trace of restriction
${u}^n\vert_K$ for some ${u}^n \in H^1(\Omega)$.  By the definition of
norms, ${u}^n\vert_K \stackrel{n \to \infty}{\longrightarrow} 
{u}_K$ in $H^1(K)$, for
each element $K$. The delicate question is whether we can claim that
the union ${u}$ of ${u}_K$ is in $H^1(\Omega)$. But this follows from the
definition of distributional derivatives. Indeed, given a test function $\phi
\in \mathcal{D}(\Omega)$, we have for each $n$,
\[
 \int_\Omega {u}^n \frac{\ptl \phi}{\ptl x_i} = - \int_{\Omega}
\frac{\ptl {u}^n}{\ptl x_i} \phi 
\] 
or 
\[ \sum_K \int_K {u}^n \frac{\ptl
  \phi}{\ptl x_i} = - \sum_K \int_K \frac{\ptl {u}^n}{\ptl x_i} \phi.
\]
Passing to the limit with $n \to \infty$, we get 
\[ \int_\Omega {u}
\frac{\ptl \phi}{\ptl x_i} = - \sum_K \int_K \frac{\ptl {u}_K}{\ptl x_i}
\phi, 
\] 
which proves that the union of element-wise derivatives
$\frac{\ptl {u}_K}{\ptl x_i}$, a function in $L^2(\Omega)$, {\em is} the
distributional derivative of ${u}$. Consequently, ${u} \in H^1(\Omega)$.

Notice that we have not attempted to extend the classical definition
of the trace space $H^{1/2}(\Gamma)$ for Lipschitz boundary $\Gamma$
to a non-Lipschitz skeleton $\Gamma_h$.\footnote{The definition of a
  Lipschitz domain includes the assumption that the domain is on one
  side of its boundary, see \cite[pp. 89]{McLean}.}  This is not
impossible but much more technical (like for domains with cracks).

A similar construction holds for globally conforming $\bfsig \in \bfH(\text{div},\Omega)$
but broken $v \in H^1(\Omega_h)$:
\[
\begin{array}{ll}
(\bfsig, \bfnab_h v)_{\Omega_h} & := \ds \sum_K (\bfsig, \bfnab v)_K \\[8pt]
& \ds = \sum_K \left( - (\text{div} \bfsig, v)_K + \langle{\hat{\sigma}}_n,{\hat{v}}\rangle_{\ptl K} \right) \\[8pt]
& \ds = -(\text{div} \bfsig,v) + \underbrace{\sum_K \langle{\hat{\sigma}}_n,{\hat{v}}\rangle_{\ptl K}}_{=: \langle{\hat{\sigma}}_n, {\hat{v}}\rangle_{\Gamma_h}}.
\end{array}
\]
In this case, we are led to the definition of the trace space $H^{-1/2}(\Gamma_h)$:
\[
H^{-1/2}(\Gamma_h) := \left\{ {\hat{\sigma}}_n = \{ {\hat{\sigma}}_{Kn} \}\in \prod_K H^{-1/2}(\ptl K) \: : \: \exists \bfsig \in \bfH(\text{div},\Omega)
: {\hat{\sigma}}_{Kn} = (\bfsig \cdot \bfn)\vert_{\ptl K} \right\}.
\]
We equip both trace spaces over the mesh skeleton with minimum energy extension norms
\[
\begin{array}{rl}
\Vert {\hat{v}} \Vert_{H^{1/2}(\Gamma_h)} := \ds &\inf_{\substack{{u} \in H^1(\Omega)\\ {u}\vert_{\Gamma_h} = u}} 
\Vert {u} \Vert_{H^1(\Omega)} \quad \text{ and} \\[24pt]
\Vert {\hat{\sigma}}_n \Vert_{H^{-1/2}(\Gamma_h)} := \ds &\inf_{\substack{\text{\bfsig} \in \text{\bfH}(\text{div},\Omega)\\
(\text{\bfsig} \cdot \text{\bfn})\vert_{\Gamma_h} = {\hat{\sigma}}_n}}
\Vert \bfsig \Vert_{\text{$\bfH$}(\text{div},\Omega)}.
\end{array}
\]
We will also need the space of traces on the internal skeleton
\[
\tilde{H}^{1/2}(\Gamma_h) := \left\{ {\hat{v}} = \{{\hat{v}}_K \} \in  \prod_K H^{1/2}(\ptl K) \: : \: \exists {v} \in H^1_0(\Omega) : {v}\vert_{\ptl K} = {\hat{v}}_K \right\},
\]
which we likewise equip with the minimum energy extension norm.



To summarize, we have defined the term $\langle{\hat{\sigma}}_n,{\hat{v}}\rangle_{\Gamma_h}$ when
one of the variables is a trace over the whole skeleton and
the other is the trace of a function from the broken energy
space. Also notice that, for sufficiently regular functions, 
$\langle{\hat{\sigma}}_n,{\hat{v}}\rangle_{\Gamma_h}$ represents either the $L^2(\Gamma_h)$-product
of trace of a conforming $\bfsig$ and inter-element jumps of $v$, or the product
of jumps in $\sigma_n$ and the trace of a globally conforming $v$. This
can be seen by switching from the summation over elements to the
summation over element faces (edges in 2D).


\subsection{Strong and Ultra-Weak Formulations}




\paragraph*{Integration by parts.}

Let $u$ now represent a group variable consisting of functions defined
on the domain $\Omega$, and $A$ be a linear differential operator
corresponding to a system of first order PDEs. We start with an abstract
integration by parts formula,
\be
(Au,v) = (u,A^\ast v) + \langle C u, v\rangle
\label{eq:integration_by_parts_classical}.
\ee
Here $(\cdot,\cdot)$ denotes the $L^2(\Omega)$-inner product, $A^\ast$ is the formal
adjoint operator, $C$ is  a  boundary operator and at this point $\langle\cdot,\cdot\rangle$ 
stands for just the $L^2(\Gamma)$-inner
product on boundary $\Gamma = \ptl \Omega$. Obviously, the formula holds under
appropriate regularity assumptions, e.g. $u,v \in C^1(\overline{\Omega})$, if all
derivatives are understood in the classical sense.

If we assume $u,v \in L^2(\Omega)$ and interpret the derivatives in a distributional sense,
we arrive naturally at the graph energy spaces
\[
\begin{array}{rl}
\ds H_A(\Omega) & \ds := \{ u \in L^2(\Omega) \: : \: A u \in L^2(\Omega) \} \quad \text{ and }\\[8pt]
\ds H_{A^\ast}(\Omega) & \ds := \{ u \in L^2(\Omega) \: : \: A^\ast u \in L^2(\Omega) \}.
\end{array}
\]
{ {\bf Assumption 1:} We take operators $A$ and $A^\ast$ to be surjections; i.e. given $f \in L^2(\Omega)$, we
  can always find $u \in H_A(\Omega)$ and $v \in H_{A^\ast}(\Omega)$
  such that $Au = f$ and $A^\ast v = f$.  Roughly speaking, this
  corresponds to an assumption  that neither $A$ nor $A^\ast$ are, in a sense,
  degenerate.\footnote{Ivo Babu\v{s}ka, private communication.}}

With $u$ and $v$ coming from the energy spaces, the domain integrals $(Au,v)$ and $(u, A^\ast v)$
are well-defined.  We now assume that the graph spaces admit
trace operators
\[
\begin{array}{rlll}
\text{tr}_A &\ds : \: H_A(\Omega) & \twoheadrightarrow  \widehat{H}_A(\Gamma) \quad \text{ and }\\[8pt]
\text{tr}_{A^\ast} &\ds : \: H_{A^\ast}(\Omega) & \twoheadrightarrow  \widehat{H}_{A^\ast}(\Gamma).
\end{array}
\]
The double arrowheads indicate that the trace operators are surjective. We equip the trace
spaces with the minimum energy extension norms
\[
\Vert \hat{u} \Vert_{\widehat{H}_A(\Gamma)} = \inf_{\substack{u \in H_A(\Omega)\\ \text{tr}_A u = \hat{u} }} \Vert u \Vert_{H_A(\Omega)} \quad \text{ and } \quad 
\Vert \hat{v} \Vert_{\widehat{H}_{A^*}(\Gamma)} = \inf_{\substack{v \in H_{A^*}(\Omega)\\ \text{tr}_{A^*} v = \hat{v} }} \Vert v \Vert_{H_{A^*}(\Omega)}.
\]

We now generalize the classical integration by parts 
formula~(\ref{eq:integration_by_parts_classical}) to a more general, distributional case:
\[
(Au,v) = (u,A^\ast v) + c(\text{tr}_A u, \text{tr}_{A^\ast} v),
\label{eq:integration_by_parts}
\]
with $u \in H_A(\Omega), v \in H_{A^\ast}(\Omega)$, and 
\[
c(\hat{u},\hat{v}), \quad \hat{u} \in \widehat{H}_A(\Gamma), \hat{v} \in \widehat{H}_{A^\ast}(\Gamma) 
\]
being a duality
pairing,\footnote{This notion extends the definition of the ``usual'' duality pairing between a space and its
dual.} i.e. a {\em definite} continuous bilinear (sesquilinear) form. Recall that form $c(\hat{u},\hat{v})$
is definite if
\begin{align*}
\left(c(\hat{u},\hat{v})\right. &= \left.0\quad \forall \hat{v} \right) \implies \hat{u} = 0 \quad \text{ and} \\[8pt]
\left(c(\hat{u},\hat{v})\right. &= \left.0\quad \forall \hat{u} \right) \implies \hat{v} = 0.
\end{align*}
Equivalently, the corresponding boundary operator
\[
C \: : \: \widehat{H}_A(\Gamma) \to (\widehat{H}_{A^\ast}(\Gamma))^\prime,\quad
\langle C\hat{u},\hat{v} \rangle = c(\hat{u},\hat{v})
\]
and its adjoint $C'$ are injective and
therefore both $C$
and $C'$ are isomorphisms.\footnote{$\mathcal{R}(A) = \mathcal{N}(C')^\perp$.} Above and in what follows, $\langle \cdot,\cdot\rangle$
denotes the usual duality pairing between a space and its dual.



\paragraph*{Integration by parts for the Stokes problem.}

We verify (and illustrate) our general assumptions for the Stokes problem.
Multiplying equations~(\ref{NVR:eq:StokesFirstIntro}-\ref{NVR:eq:StokesLastIntro}) with test functions $\bfv,q,\bftau$, integrating by parts
and summing up the equations, we obtain
\begin{align*}
(- \text{\bf div} ( \bfsig - p \bfI ),\bfv)+
(\text{div} \bfu,q)+
(\bfsig  - \bfnab \bfu,\bftau)
=& \quad ( \bfsig - p \bfI, \bfnab \bfv) + \langle  (- \bfsig + p \bfI) \bfn, \bfv\rangle\\[8pt] 
&+ (\bfu, - \bfnab q) + \langle  \bfu \cdot \bfn,q\rangle\\[8pt]
&+ (\bfsig,\bftau) + (\bfu, \text{\bf div} \bftau) + \langle \bfu, - \bftau \bfn\rangle\\[8pt] 
=& \quad (\bfu, {\bf div} (\bftau - q \bfI)) + (p, - \text{div} \bfv) + (\bfsig, \bftau + \bfnab \bfv) \\[8pt]
&+ \langle  (- \bfsig + p \bfI) \bfn, \bfv\rangle + \langle \bfu, (-\bftau + q\bfI) \bfn\rangle.
\end{align*}
Thus, comparing with the abstract theory, we have
\[
\begin{array}{rl}
u & = (\bfu,p,\bfsig), \\[8pt]
v & = (\bfv,q,\bftau), \\[8pt]
Au & = (- \text{\bf div} ( \bfsig - p \bfI ),\text{div} \bfu,\bfsig  - \bfnab \bfu), \\[8pt]
A^\ast v & = (\text{\bf div} (\bftau - q \bfI) ,- \text{div} \bfv,\bftau + \bfnab \bfv ).
\end{array}
\]
The operator is not (formally) self-adjoint but the corresponding energy graph spaces are identical,
$H_A(\Omega) = H_{A^\ast}(\Omega)$, where
\[
H_A(\Omega) = \{ (\bfu,p,\bfsig) \: : \: \bfsig - p\bfI \in
\bfH(\text{\bf div},\Omega), \bfu \in \bfH^1(\Omega) \}.
\]
The traces are also the same; that is, $\text{tr}_A = \text{tr}_{A^\ast}$, where 
\[
\text{tr}_A : H_A(\Omega) \ni (\bfu,p,\bfsig) \to ((-\bfsig + p \bfI)
\bfn, \bfu) \in \bfH^{-1/2}(\Gamma) \times \bfH^{1/2}(\Gamma).
\]
The
boundary term, being the sum of standard duality pairings for
respective components of the traces, is definite. 

\begin{remark}
The Stokes problem can be framed into a general abstract case
discussed in \cite[Section 6.6]{FAbook}.  The boundary term is
identified as a {\em concominant} of the corresponding
traces. Moreover, the Stokes equation can be cast into a Friedrichs'
system \cite{ErnGuermond08}, and hence can be studied using the
unified DPG framework \cite{Bui-ThanhDemkowiczGhattas11b} which uses
boundary operator and graph spaces. Here, inspired by our previous work
\cite{Bui-ThanhDemkowiczGhattas11b}, we develop an abstract
DPG theory using the trace operators and graph spaces, assuming the existence of
traces, and study the Stokes equation using this framework.
\end{remark}


\paragraph*{Strong formulation with homogeneous BC.}
We now return to our abstract setting and assume that the boundary operator $C$ can be split into two operators $C_1$ and $C_2$ such that
\[
\begin{array}{ll}
\langle C u,v\rangle &= \langle C_1 u,v\rangle + \langle C_2 u,v\rangle \\[5pt]
& = \langle C_1 u,v\rangle + \langle u, C_2^\prime v\rangle;\\[5pt]
\end{array}
\]
we also take $C_{1}$ and $C_{2}$ to be ``reasonable'' in the sense that both have closed range. It should be pointed out that this is analogous to boundary operator splitting due to Friedrichs \cite{Friedrichs58, ErnGuermondCaplain07}.

We are interested in solving a non-homogeneous boundary-value problem
\be
\left\{
\begin{array}{rll}
A u & = f \quad & \mbox{in } \Omega,\\
C_1 u & = f_D & \mbox{on } \Gamma,
\end{array}
\right.
\label{eq:nonhom_BVP}
\ee
with $f \in L^2(\Omega)$ and $ f_D \in \mathcal{R}(C_1)$.


We begin with the homogeneous BC case:
\be
\left\{
\begin{array}{rll}
A u & = f \quad & \mbox{in } \Omega,\\
C_1 u & = 0 & \mbox{on } \Gamma.
\end{array}
\right.
\label{eq:hom_BVP}
\ee
Introducing the spaces
\[
\begin{array}{ll}
U := \{ u \in H_A(\Omega) : & C_1 \: \text{tr}_A u = 0 \} \quad \text{ and } \\[8pt]
V := \{ v \in H_{A^\ast}(\Omega) : & C_2^\prime \: \text{tr}_{A^\ast} v = 0 \},
\end{array}
\]
we see that, if we restrict operators $A$ and $A^\ast$ to $U$ and $V$, the
boundary term vanishes. However, for $A$ and $A^\ast$ to be
$L^2$-adjoint, we have to make an additional technical
assumption.\footnote{The domain of the adjoint operator has to be maximal in the
  sense that it includes {\em all} $v$ for which the boundary term
  vanishes.}\\
{{\bf Assumption 2:} 
\be \left( \langle u,
  C_2^\prime v\rangle = 0 \quad \forall u \: : C_1 u = 0 \right) \implies
  C_2^\prime v = 0.
\label{eq:adjoint_assumption}
\ee } 
The following lemma allows us to use this condition to establish a decomposition of the trace space.
\begin{lemma}
Assume that $C: X \to
Y$ is an isomorphism from Hilbert space $X$ onto a Hilbert space
$Y$. Assume $C$ has been split into $C_1$ and $C_2$ that satisfy
condition~(\ref{eq:adjoint_assumption}).  Each of following conditions is then equivalent to~(\ref{eq:adjoint_assumption}).
\[
\begin{array}{c}
\mathcal{N}(C_1)^\perp \cap \mathcal{R}(C_2^\prime) = \{ 0 \},  \\[8pt]
\mathcal{N}(C_1)^\perp \cap \mathcal{N}(C_2)^\perp = \{ 0 \},  \\[8pt]
\mathcal{N}(C_2) \cap \mathcal{N}(C_1)  = \{ 0 \},  \\[8pt]
X = \mathcal{N}(C_2) \oplus \mathcal{N}(C_1).
\end{array}
%\label{eq:adjoint_assumption_equivalent}
\]
\end{lemma}
\begin{proof} Elementary with an application of the Closed Range Theorem that we recall below. \end{proof}
Condition~(\ref{eq:adjoint_assumption}) thus decomposes the trace
space $\widehat{H}_A(\Gamma)$ into the direct sum of the nullspaces of
operators $C_1$ and $C_2$; \be \widehat{H}_A(\Gamma) = \widehat{H}_A^1(\Gamma)
\oplus \widehat{H}_A^2(\Gamma),
\label{eq:trace_split}
\ee where $\widehat{H}_A^1(\Gamma) = \mathcal{N}(C_2)$ and
$\widehat{H}_A^2(\Gamma) = \mathcal{N}(C_1)$. In other words, for each
$\hat{u} \in \widehat{H}_A(\Gamma)$, there exist unique $ \hat{u}_1 \in
\widehat{H}_A^1(\Gamma)$ and $ \hat{u}_2 \in \widehat{H}_A^2(\Gamma)$ such that
\[
\hat{u} = \hat{u}_1 + \hat{u}_2,
\]
which is analogous to the condition introduced by Friedrichs \cite{Friedrichs58} on his boundary
operator $M$, subsequently generalized in \cite{ErnGuermondCaplain07}, and first used
in the DPG context in \cite{Bui-ThanhDemkowiczGhattas11b}.

Having reduced the problem with homogeneous BCs to the classical theory of
$L^2$-adjoint operators, we now recall the Banach Closed Range
Theorem. Let $T : X \to Y$ be a linear, continuous operator from a
Hilbert space $X$ into a Hilbert space $Y$. Let $\check{T}$ be the
corresponding operator defined on the quotient space
$X/\mathcal{N}(T)$ or, equivalently, the restriction of $T$ to the
$X$-orthogonal complement of null space of operator $T$,
$\mathcal{N}(T)^\perp \subset X$. Let $\check{T}^\ast$ denote the
analogous operator for the adjoint $T^\ast$.  The following
conditions are then equivalent to each other.  
\[
\begin{array}{c}
T \text{ has a closed range }, \\[8pt]
T^\ast \text{ has a closed range }, \\[8pt]
\check{T} \text{ is bounded below, i.e. } 
\Vert T u \Vert \ge \gamma \Vert u \Vert \quad \forall u \in \mathcal{N}(T)^\perp,\\[8pt]
\check{T}^\ast \text{ is bounded below, i.e. } 
\Vert T^\ast v \Vert \ge \gamma \Vert v \Vert \quad \forall v \in \mathcal{N}(T^\ast)^\perp.
\end{array}
%\label{eq:closed_range_conditions}
\]
Note that the (maximal) constant $\gamma$ is the same for $T$ and $T^\ast$.

{{\bf Assumption 3:} The operator $A\vert_U$, restricted to the
  $L^2$-orthogonal complement $\mathcal{N}(A)^\perp$, is bounded
  below.}

The homogeneous problem~(\ref{eq:hom_BVP}) and its adjoint counterpart
are thus well-posed. More precisely, for each data function $f$ which is
$L^2$-orthogonal to the null space of the adjoint operator, a solution
exists, is unique in the orthogonal complement of the null space
of the operator (equivalently, in the quotient space), and
depends continuously on $f$.  The inverse of the maximal constant $\gamma$ is precisely the norm of the inverse
operator from $\mathcal{R}(A)$ into $\mathcal{N}(A)^\perp$ (which is equal to the norm of the solution operator from $\mathcal{R}(A^{\ast})$ into $\mathcal{N}(A^{\ast})^\perp$).


\paragraph*{Strong formulation with homogeneous BC for the Stokes problem.}
We have 
\[
C_1 u = C_1(\bfu,p,\bfsig) = \text{tr } \bfu \quad \text{and} \quad  
C_2^\prime v = C_2^\prime(\bfv,q,\bftau) = \text{tr } \bfv.
\]
Condition~(\ref{eq:adjoint_assumption}) is easily satisfied.
We have $U=V$, where
\[
U = \{ (\bfu,p,\bfsig) \in (\bfL^2(\Omega) \times L^2(\Omega) \times
\bfL^2(\Omega)) \: : \: \bfsig - p\bfI \in \bfH(\text{\bf
  div},\Omega), \bfu \in \bfH^1_0(\Omega) \}.
\]
Both $A$ and $A^\ast$ have non-trivial null space consisting
of constant pressures.  To ensure uniqueness, we have to
restrict ourselves to pressures $p$ and $q$ with zero average;
\[ p,q \in
L^2_{0} := \left\{ q \in L^2(\Omega) \; : \: \int_\Omega q = 0 \right\}.
\]
The proof that $A$ and $A^\ast$ are bounded below involves the
Ladyzenskaya inf-sup condition; for the reader's convenience we reproduce
the classical reasoning in Appendix \ref{sec:boundedness_below}.


\paragraph*{Strong formulation with non-homogeneous BC.}


We are now ready to tackle the case with a non-homogeneous boundary
condition. We have
\[ (Au,v) - \langle C_1 u,v\rangle = (u,A^\ast v) +
\langle u,C_2^\prime v\rangle \quad u \in H_A(\Omega), v \in H_{A^\ast}(\Omega) .
\]
We are interested in the operators 
\[
\begin{array}{rrl}
H_A(\Omega) \ni u \to & (Au,C_1 u) \in & L^2(\Omega) \times (\widehat{H}_{A^\ast}(\Gamma))^\prime \quad \text{ and } \\[8pt]
H_{A^\ast}(\Omega) \ni v \to & (A^\ast v,C_2^\prime v) \in & L^2(\Omega) \times (\widehat{H}_{A}(\Gamma))^\prime.
\end{array}
\label{eq:bounded_below_nhom}
\]
We have the following classical result.
\begin{theorem}
Assume
that data $f \in L^2(\Omega)$ and 
$f_D \in \mathcal{R}(C_1)$ satisfy the compatibility condition
\[
(f,v) - \langle f_D,v\rangle = 0 \quad \forall v : A^\ast v = 0, C_2^\prime v = 0.
\]
Problem~(\ref{eq:nonhom_BVP}) has a unique solution $u$ in $\mathcal{N}(A)^\perp$ that depends
continuously upon the data; i.e. there exists a constant $\tilde{\gamma}>0$, independent of the data, such that
\[
\tilde{\gamma} \Vert u \Vert_{H_A(\Omega)} \leq \left( \Vert f \Vert^2 + \Vert  f_D \Vert^2 \right)^{1/2}.
\]
The analogous result holds for the adjoint operator $(A^\ast, C_2^\prime)$.
%% \tanbui{Same result holds for adjoint operator $A^\ast$}{Here we mean the pair $(A^\ast v,C_2^\prime v)$, should we say ``Similar results hold for the adjoint equation given by $(A^\ast v,C_2^\prime v)$ since the compatibility condition and the operators are different?}.
\label{theorem:strong_nonhom_formulation}
\end{theorem}
\begin{proof}
Let $\bar{C}_1$ denote the restriction of $C_1$ to $\widehat{H}_A^1(\Gamma)$. Since $\bar{C}_1$ is then
injective and has closed range,
it admits a continuous inverse:
\[
\norm{ \hat{u}_1 }_{\widehat{H}_A(\Gamma)} = \norm{ \bar{C}_1^{-1} f_D }_{\widehat{H}_A(\Gamma)} \leq \frac{1}{\delta} \norm{ f_D }_{(\widehat{H}_{A^\ast}(\Gamma))^\prime}.
\]
Let $\hat{u} = \LRp{\hat{u}_1,0}$ and let $\tilde{\hat{u}}$ be the
minimum-energy extension of $\hat{u}$ in $H_A(\Omega)$.  We seek a solution $u$
of the form 
\[ u = u_0 + \tilde{\hat{u}},
\]
where $u_0 \in
\mathcal{N}(A\vert_U)^\perp$ solves the homogeneous BVP~(\ref{eq:hom_BVP})
with the modified right-hand side $f - A \tilde{\hat{u}}$. The load $f - A
\tilde{\hat{u}}$ satisfies the compatibility condition for the
homogeneous case. Indeed, 
\[
\begin{array}{rl}
(f - A \tilde{\hat{u}}, v) & = (f,v) - (A \tilde{\hat{u}},v) \\[8pt]
& = (f,v) - \left( (\tilde{\hat{u}},A^\ast v) - \langle f_D,v\rangle - \langle \tilde{\hat{u}},C_2^\prime v\rangle \right) \\[8pt]
& = (f,v) - \langle f_D,v\rangle = 0,
\end{array}
\]
for each $v$ such that $A^\ast v = 0$ and $C_2^\prime v = 0$.

We now have
\[
\begin{array}{ll}
\Vert u \Vert^2_{H_A} & = \Vert u_0 + \tilde{\hat{u}} \Vert^2_{H_A}\\[8pt]
& \le 2\LRp{\Vert u_0 \Vert^2_{H_A} + \Vert  \tilde{\hat{u}} \Vert^2_{H_A}}.
\end{array}
\]
Now, observe that
\[ \Vert \tilde{\hat{u}} \Vert_{H_A(\Omega)} =
\Vert \hat{u} \Vert_{\widehat{H}_A(\Gamma)} \leq \frac{1}{\delta} \Vert f_D
\Vert_{(\widehat{H}_{A^\ast}(\Gamma))^\prime}
\] 
and that
\[
\begin{array}{ll}
\Vert u_0 \Vert^2_{H_A} & = \Vert u_0 \Vert^2 + \Vert A u_0 \Vert^2 \\[8pt]
&\leq \ds \left(\frac{1}{\gamma^2} + 1 \right) \Vert A u_0 \Vert^2 \\[12pt]
& \ds \leq 2 \left(\frac{1}{\gamma^2} + 1 \right) \left[ \Vert f \Vert^2 + \frac{1}{\delta^2} \Vert f_D \Vert^2 \right].
\end{array}
\]
Combining these results ends the proof with
\[
\frac{1}{\tilde{\gamma}} \le 2\max\LRc{\sqrt{\frac{1}{\gamma^2} + 1}, \frac{1}{\delta}\sqrt{\frac{1}{\gamma^2} + \frac{3}{2}}}.
\]

\end{proof}

%% \begin{corollary}
%% Operators~(\ref{eq:bounded_below_nhom})
%% are bounded below with common constant 
%% \[
%% \tilde{\gamma} \geq (2 (\frac{1}{\gamma^2} + 1 )\max \{ 1, \frac{1}{\delta^2}\} )^{-1/2}.
%% \]
%% \end{corollary}

\paragraph*{Strong formulation with non-homogeneous BCs for the Stokes problem.}
The ranges of operators $C_1$ and $C'_2$ coincide exactly with
$\bfH^{1/2}(\Gamma)$.  The strong formulation for the non-homogeneous
Stokes problem is well-posed, provided the data $g$ and $\bfu_D$ satisfy the
compatibility condition
\[ 
\int_\Omega g = \int_{\Gamma} \bfu_D \cdot
\bfn.
\] 
The analogous conclusion holds for the adjoint operator.


\paragraph{Ultra-weak (variational) formulation.}


We are now ready to formulate the {\em ultra-weak variational
  formulation} for problem~(\ref{eq:nonhom_BVP}).  The steps are as
follows.
\begin{enumerate}
  \item Integrate by parts:
\[
(u,A^\ast v) + \langle C \text{tr}_A u, v\rangle = (f,v).
\]
  \item Split the boundary operator $C$ into $C_1$ and $C_2$ according to the decomposition in ~(\ref{eq:trace_split}):
\[
(u,A^\ast v) + \langle C_1 (\text{tr}_A u)_1, v\rangle   +  \langle C_2 (\text{tr}_A u)_2, v\rangle = (f,v). 
\]
  \item Apply the BC, moving the known term $C_1 (\text{tr}_A u)_1 = f_{D} $ to the right-hand side:
\[
(u,A^\ast v) + \langle (\text{tr}_A u)_2, C_2^\prime v\rangle = (f,v) - \langle f_D,v\rangle.
\]
\item Declare $\hat{u}_2 = (\text{tr}_A u)_2$ to be an independent unknown.  The problem then becomes
\be
\left\{
\begin{array}{ll}
\text{Find } u \in L^2(\Omega), \hat{u}_2 \in \widehat{H}_A^2(\Gamma) \text{ such that} \\[8pt]
(u,A^\ast v) + \langle \hat{u}_2, C_2^\prime v\rangle = (f,v) - \langle f_D,v\rangle \quad \forall v \in H_{A^\ast}(\Omega).
\end{array}
\right.
\label{eq:ultraweak_formulation}
\ee
\end{enumerate}
The bilinear form
\be
b((u,\hat{u}_2),v) := (u,A^\ast v) + \langle \hat{u}_2, C_2^\prime v\rangle
= (u,A^\ast v) + c(\hat{u}_2, v)
\label{eq:bilinear_form_ultraweak_formulation}
\ee
generates two associated operators $B$ and $B^{\prime}$, where
\[
b((u,\hat{u}_2),v) = \langle  B(u,\hat{u}_2),v\rangle = \langle (u,\hat{u}_2),B^\prime v\rangle.
\]
Operator $B'$ corresponds to the strong setting for the adjoint $A^\ast$ with non-homogeneous
BCs;
\[
B' v = (A^\ast v, C_2^\prime v) \in L^2(\Omega) \times
(\widehat{H}_A(\Gamma))^\prime.
\]
In order to determine the null-space of operator $B$, assume that
\[
 b((u,\hat{u}_2),v) = 0 \quad \forall v \in H_{A^\ast}(\Omega).
\]
Testing first with $v \in \mathcal{D}(\Omega)$, we deduce that $A u =
0$. Integrating the first term by parts, and testing with arbitrary
$v$, we learn that $\hat{u}_2 = u$ on $\Gamma$ and
  $C_1 u = 0$.

\begin{theorem}
Problem~(\ref{eq:ultraweak_formulation}) is well-posed. In particular,
for each $f$ and $f_D$ which satisfy the compatibility condition
\be
(f,v) - \langle f_D,v\rangle =0 \quad \forall v \in \mathcal{N}(A^\ast\vert_V),
\label{eq:compatibility_condition_ultraweak}
\ee a solution of (\ref{eq:ultraweak_formulation}) exists which is
unique up to $u \in \mathcal{N}(A\vert_U)$ and corresponding
$\hat{u}_2 \in \widehat{H}_A^2(\Gamma)$ such that $u_2 - \hat{u}_2 \in
{\mathcal{N}(C_1)}$ where $u_2$ is the corresponding component of trace
of $u$ in $\widehat{H}_A^2(\Gamma)$.

The inf-sup constant for bilinear
form~(\ref{eq:bilinear_form_ultraweak_formulation}) is equal to the
inf-sup constant of the adjoint operator  $(A^\ast, C_2^\prime)$ from
Theorem~\ref{theorem:strong_nonhom_formulation}.
\end{theorem}
\begin{proof}
We observe that the conjugate
$B^\prime$ of operator $B$ corresponding to the bilinear
form~(\ref{eq:bilinear_form_ultraweak_formulation}) coincides with the
strong form of operator $(A^\ast, C_2^\prime)$. The result is then a direct consequence of Theorem \ref{theorem:strong_nonhom_formulation}.
\end{proof}



\paragraph{Ultra-weak formulation for the Stokes problem.} 
The Stokes formulation is now just a matter of interpretation. The solution consists of $u = (\bfu,p,\bfsig)$
and unknown traction
\[
\hat{\bft} = (-\bfsig + p \bfI) \bfn
\]
Remember that only for a sufficiently regular\footnote{That is, $u \in H_A(\Omega)$.}
 solution $u$ will $\hat{\bft}$ coincide with
the trace $(-\bfsig + p \bfI) \bfn$. In the ultra-weak formulation, the traction $\hat{\bft}$
appears as an {\em independent} unknown. 
With homogeneous incompressibility constraint, the variational problem reads as follows.
\[
\left\{
\begin{array}{ll}
\text{Find }\bfu \in \bfL^2(\Omega), p \in L^2(\Omega), \bfsig \in \bfL^2(\Omega),
\hat{\bft} \in H^{-1/2}(\Gamma) \text{ such that} \\[8pt]
(\bfu, {\bf div} (\bftau - q \bfI)) + (p, - \text{div} q) + (\bfsig, \bftau + \bfnab \bfv) 
+ \langle \hat{\bft},\bfv\rangle = (\bff,\bfv) - \langle \bfu_D,(-\bftau + q\bfI) \bfn\rangle \\[8pt]
\hfill \forall (\bfv,q,\bftau) \text{ such that } 
\bftau - q\bfI \in \bfH(\text{\bf div},\Omega), \bfv \in \bfH^1(\Omega).
\end{array}
\right.
\]
The load is specified by a body force $\bff \in \bfL^2(\Omega)$ and a velocity $\bfu_D \in \bfH^{1/2}(\Gamma)$
on the boundary with vanishing normal component:
\[
\int_{\Gamma} \bfu_D \cdot \bfn = 0.
\]
The solution is determined up to a constant pressure $p_{0}$
{and corresponding constant traction $\hat{\vect{t}}_{0} = p_{0} \bfn$.}


Notice that there are no boundary conditions imposed on the test functions. This is important
from a practical point of view.

{
\begin{remark}
{\bf Strong versus weak imposition of boundary conditions.}
In classical variational formulations for second order PDEs, we distinguish between {\em strong}
(Dirichlet) BCs and weak (Neumann) BCs. Dirichlet BCs are accounted for by coming up
with a finite-energy lift of the BC data, and looking for a solution to the 
problem with homogeneous BCs and a modified ``load vector'' that includes the action
of the bilinear form on the lift \cite[p. 34]{hpbook}. In practice, the Dirichlet data is first projected
(interpolated) into the trace of FE space and then lifted with FE shape functions.
By contrast to the Dirichlet BC case, Neumann BCs only contribute to the load vector. 
The terms ``strong'' and ``weak'' refer to the fact that, with Dirichlet data in the
FE space, Dirichlet BCs are enforced pointwise, whereas Neumann BCs, in general, are satisfied
only in the limit.
}

{ In an ultra-weak variational formulation, each BC may
  be specified either in a ``strong'' or ``weak'' way. The formulation
  discussed above corresponds to a weak imposition of the BC. Data
  $f_D$ contributes to the load vector and is accounted for on the
  element level, in the integration for the load vector. An alternate,
  strong imposition of the same BC, starts with finding a trace lift
  $\hat{u}_0$ of the BC,
$$
C_1 \hat{u}_0 = f_D.
$$
Notice that the lift may have a non-zero $\widehat{H}_A^2$-component but the final trace will
be equal to the sum of the lift and an unknown component $\hat{u}_2 \in \widehat{H}_A^2$,
$$
\hat{u} = \hat{u}_0 + \hat{u}_2.
$$
The term $\langle f_D,v \rangle$ on the right-hand side of~(\ref{eq:bilinear_form_ultraweak_formulation})
is simply replaced with $c(u_0,v)$. 
The rest of the formulation remains unchanged.
The difference between the two formulations becomes
more visible in context of discontinuous test functions discussed next.
\label{remark:strong_vs_weak_BC}
\end{remark}
}


\subsection{DPG formulation}\label{sec:AnalysisDPGFormulation}

The essence of the DPG formulation lies in extending the concept of the ultra-weak variational
formulation to broken test spaces. We begin by partitioning domain $\Omega$ into finite elements $K$
and integrating by parts on each element:
\[
(Au,v)_{K} = (u,A^\ast v)_{K} + c_{\ptl K}(u,v) \quad u \in H_A(K), v \in H_{A^\ast}(K).
\]
Next, we sum over all elements to obtain
\[
\underbrace{\sum_K (Au,v)_{K}}_{ = (Au,v)}
 = \underbrace{\sum_K (u,A^\ast v)_{K}}_{=: (u,A_h^\ast v)_{h}}
 + \underbrace{\sum_K c_{\ptl K}(u,v)}_{=: c_h(u,v)}
 \quad u \in H_A(\Omega), v \in H_{A^\ast}(\Omega_h).
\]
Here we take $u$ to be globally conforming but allow $v$ to come from the broken graph space
\[
H_{A^\ast}(\Omega_h) := \left\{ v \in L^2(\Omega) \: : \: A^\ast v\vert_K \in L^2(K) \: \forall K\right\}.
\]
The index $h$ on the domain
indicates that the formal adjoint operator is to be
understood element-wise. The boundary term $c_{h}$ now extends to the whole skeleton $\Gamma_h = \cup_K \ptl K$.
For the internal skeleton $\Gamma_h^0 = \Gamma_h - \Gamma$, this term represents the action of traces $\hat{u}$
on the jumps of traces $\hat{v}$. As with $H^{1/2}(\Gamma_h)$ and $H^{-1/2}(\Gamma_h)$,
we introduce a general, abstract space of traces on the skeleton,
\[
\widehat{H}_A(\Gamma_h) := \left\{
\hat{u} = \{ \hat{u}_K \} \in \prod_K \widehat{H}_A(\ptl K) \: : \: \exists u \in H_A(\Omega) : 
\text{tr}_A u\vert_K =  \hat{u}_K \right\},
\]
and the corresponding subspace of traces that vanish on $\Gamma = \ptl \Omega$,
\[
\hat{\tilde{H}}_A(\Gamma_h) := \left\{
\hat{u} = \{ \hat{u}_K \} \in \prod_K \widehat{H}_A(\ptl K) \: : \: \exists u \in \tilde{H}_A(\Omega)
 : 
\text{tr}_A u\vert_K =  \hat{u}_K \right\},
\]
where
\[
\tilde{H}_A(\Omega) = \left\{ u \in H_A(\Omega) \: : \: \text{tr }u = 0 \text{ on } \Gamma \right\}.
\]
As usual, we equip the trace space with the minimum energy extension norm.

Any function $u \in H_A(\Omega)$ can be decomposed into an extension of its trace to $\Gamma$
and a component that vanishes on $\Gamma$:
\[
u = E(\text{tr } u ) + \tilde{u},\quad \tilde{u} \in \tilde{H}_A(\Omega).
\]
If the extension $E(\text{tr } u )$ is the minimum-energy extension, the decomposition above
is $H_A$-orthogonal. This implies a corresponding decomposition for traces $\hat{u}
\in \widehat{H}_A(\Gamma_h)$:
\[
\hat{u} = 
{\hat{E}}\hat{u}_0 + \hat{\tilde{u}},\quad \hat{\tilde{u}} \in \hat{\tilde{H}}_A(\Gamma_h).
\]
Here {
$\hat{u}_0 \in \widehat{H}_A(\Gamma)$ is the restriction of $\hat{u}$ to $\Gamma$ and  
$\hat{E}\hat{u}_0 \in \widehat{H}_A(\Gamma_h)$ is any extension of $\hat{u}_0$ back to the whole
skeleton $\Gamma_h$.}
Again, if we use the minimum-energy extension, the decomposition
is $\widehat{H}_A(\Gamma_h)$-orthogonal. We have
\be
\widehat{H}_A(\Gamma_h) = {\hat{E}} \widehat{H}_A(\Gamma) \oplus \hat{\tilde{H}}_A(\Gamma_h).
\label{eq:trace_decomposition}
\ee








By construction, we have a generalization of the trace operator to the whole skeleton,
\[
\text{tr} : H_A(\Omega) \twoheadrightarrow \widehat{H}_A(\Gamma_h).
\]
The skeleton term
$c_h(\hat{u},\hat{v})$ is well-defined for 
$\hat{u} \in \widehat{H}_A(\Gamma_h)$ and $\hat{v} = \{\hat{v}_K \} \in \prod_K H_{A^\ast}(\ptl K)$.
We also have the condition
\[
\begin{array}{c}
\left((c_h(\hat{u},\hat{v}) = 0 \: \forall \hat{u} \in \hat{\tilde{H}}_A(\Gamma_h) \right)
\Longleftrightarrow v \in H_{A^\ast}(\Omega). \\[8pt]
\end{array}
\]
Indeed, if we restrict ourselves to a globally conforming test function, the skeleton term reduces
to a term that involves only the domain boundary $\Gamma$, where the trace $\hat{u}$ vanishes. The converse follows
from the definition of distributional derivatives. 
Indeed, for any test function $\phi \in \mathcal{D}(\Omega)$,
we have
\[
c_h (\text{tr} \phi, v) = (A \phi,v) - (\phi,A^\ast_h v)_h = 0,
\]
which proves that the union of element-wise values $A^\ast_h v$ (which lives in $L^2(\Omega)$) is
equal to $A^\ast v$ in the sense of distributions.

We now use the decomposition of traces~(\ref{eq:trace_decomposition})
to set up the boundary operators. 
{
Recall that condition~(\ref{eq:adjoint_assumption}) decomposed the
trace space $\widehat{H}_A(\Gamma)$
into  the direct sum of the nullspaces of operators $C_2$ and $C_1$:
\[
\widehat{H}_A(\Gamma) = \widehat{H}_A^1(\Gamma) \oplus \widehat{H}_A^2(\Gamma),
\quad \hat{u} = \hat{u}_1 + \hat{u}_2.
%\label{eq:trace_split_h}
\]
The first term is known from the boundary condition; the second remains as an additional
unknown.
We have
\be
\begin{array}{ll}
c_h(\hat{u},\hat{v}) & = c_h(\hat{E}\hat{u}_0,\hat{v}) + c_h(\hat{\tilde{u}},\hat{v}) \\[8pt]
& 
= c_h(\hat{E} \hat{u}_0^1, \hat{v}) + c_h(\hat{E} \hat{u}_0^2, \hat{v}) 
+  c_h(\hat{\tilde{u}},\hat{v}).
\end{array}
\label{eq:split_ch}
\ee
For conforming test functions $\hat{v} \in \widehat{H}_{A^\ast}(\Gamma_h)$, 
the bilinear form on the skeleton reduces to the
bilinear form on the domain boundary,
$$
c_h(\hat{u},\hat{v}) = c(\hat{u},\hat{v}) = c(\hat{u}_0^1,\hat{v}) + c(\hat{u}_0^2,\hat{v}) 
= \langle f_D, \hat{v} \rangle + c(\hat{u}_0^2,\hat{v}).
$$
In particular, this is independent of the choice of lift $\hat{E}$.
To impose the BC strongly, we need to find a trace lift of the BC data $f_D$,
$$
C_1 \hat{u}_0 = f_D,
$$
move the term with the lift to the right-hand side, and look for the unknown component $\hat{u}_2$
of the trace on $\Gamma$. The final formulation reads as follows.
\[
\left\{
\begin{array}{ll}
u \in L^2(\Omega), \hat{u} \in \hat{E} \widehat{H}_A^2(\Gamma) \oplus \hat{\tilde{H}}_A(\Gamma_h)  \\[8pt]
(u,A^\ast_h v)_h +  c_h(\hat{u},v) = (f,v) - c_h(\hat{E}\hat{u}_0,v)
\quad \forall v \in H_{A^\ast}(\Omega_h)
\end{array}
\right.
%\label{eq:abstract_DPG_formulation_strong}
\]
However, if we decide to enforce the BC in a weak way, we need to replace the first
term on the right-hand side of ~(\ref{eq:split_ch}) with an extension of
known BC data $\langle f,v\rangle$ to discontinuous test functions.
This is always possible as the term $c_h(\hat{E} \hat{u}_0,v)$ provides an example of such an extension.
}

{
The final abstract DPG formulation\footnote{That is, the ultra-weak variational formulation with broken test functions.}
is then
\be
\left\{
\begin{array}{ll}
u \in L^2(\Omega), \hat{u} \in \hat{E} \widehat{H}_A^2(\Gamma) \oplus \hat{\tilde{H}}_A(\Gamma_h)  \\[8pt]
(u,A^\ast_h v)_h +  c_h(\hat{u},v) = (f,v) - \langle f_D, v\rangle_{\Gamma_h} 
\quad \forall v \in H_{A^\ast}(\Omega_h).
\end{array}
\right.
\label{eq:abstract_DPG_formulation}
\ee
}


The bilinear\footnote{Sesquilinear for complex-valued problems.} form corresponding to the formulation 
\[
b((u,\hat{u}),v) := (u,A^\ast_h v)_h  + c_h(\hat{u},v)
\]
generates operators $B$ and $B'$, where
\[
 b((u,\hat{u}),v) = \langle B(u,\hat{u}),v\rangle 
= \langle (u,\hat{u}),B'v\rangle.
\]
The null space of conjugate operator $B'$ coincides with the null space of $A^\ast\vert_V$. Indeed, let
\[
b((u,\hat{u}),v) = 0 \quad \forall (u,\hat{u}).
\label{eq:null_space_conjugate_operator}
\]
Taking arbitrary
$\hat{u} \in \hat{\tilde{H}}_A(\Gamma_h)$, we conclude that $v$ must be globally conforming, 
so the bilinear form in~(\ref{eq:null_space_conjugate_operator}) reduces to 
the bilinear form~(\ref{eq:bilinear_form_ultraweak_formulation}).

The null space of DPG operator $B$ consists of all $(u,\hat{u})$ such that
\[
 b((u,\hat{u}),v) = 0 \quad \forall v \in H_{A^\ast}(\Omega_h).
\]
As with the ultra-weak variational formulation, we first test with
$v \in \mathcal{D}(\Omega)$ to conclude that $A u = 0$. Integrating the first term by parts
and testing with arbitrary $v$, we conclude that $\hat{u} = u$ on $\Gamma_h$. In particular, as
$\hat{u}\vert_\Gamma \in \widehat{H}^2_A(\Gamma)$, this implies that $C_1 u = 0$ on $\Gamma$.


We are in a position to state our main abstract result.
\begin{theorem}
Problem~(\ref{eq:abstract_DPG_formulation}) is well-posed. More precisely, for any 
data $f$ and $f_{D}$ that satisfy the compatibility condition~(\ref{eq:compatibility_condition_ultraweak}), 
the problem has a solution $(u,\hat{u})$ such that $(u,\hat{u}\vert_\Gamma)$
coincides with the solution of~(\ref{eq:ultraweak_formulation}).
The bilinear form satisfies
the inf sup condition
\[
\sup_{v \in H_{A^\ast}(\Omega_h)} 
\frac{\vert (u,A^\ast_h v)_h + c_h(\hat{u},v) \vert}{\Vert v \Vert_{H_{A^\ast}(\Omega_h)}}
\geq \gamma_{DPG} 
\left( \Vert u \Vert_{L^2(\Omega)}^2 + \Vert \hat{u} \Vert_{\widehat{H}_A(\Gamma_h)}^2 \right)^{1/2}
%\sqrt{ \Vert u \Vert_{L^2(\Omega)}^2 + \Vert \hat{u} \Vert_{\widehat{H}_A(\Gamma_h)}^2 }
\]
for all $\hat{u} \in \hat{E} \widehat{H}_A^2(\Gamma) \oplus
 \hat{\tilde{H}}_A(\Gamma_h)$, and $u \in L^2(\Omega)$ {orthogonal to the null space:}
\[
\left\{ (u,\hat{u}) \: : \: u \in \mathcal{N}(A\vert_U) \text{ and } \hat{u} = u \text{ on }\Gamma_h \right\}
\]
The inf-sup constant $\gamma_{DPG}$ is mesh-independent and
$\gamma_{DPG}=$ {$O(\gamma)$ and $O(\tilde{\gamma})$ for the adjoint operator.}
\end{theorem}
\begin{proof}
We will switch the order of spaces in the inf-sup condition and prove that
\be
\sup_{\substack{u \in L^2(\Omega)\\ \hat{u} \in E \widehat{H}_A^2(\Gamma) \oplus
 \hat{\tilde{H}}_A(\Gamma_h)
}} 
\frac{\vert (u,A^\ast_h v)_h + c_h(\hat{u},v) \vert}
%{\sqrt{ \Vert u \Vert_{L^2(\Omega)}^2 + \Vert \hat{u} \Vert_{\widehat{H}_A(\Gamma_h)}^2 }
{\left( \Vert u \Vert_{L^2(\Omega)}^2 + \Vert \hat{u} \Vert_{\widehat{H}_A(\Gamma_h)}^2 \right)^{1/2}
}
\geq \gamma_{DPG} 
\Vert v \Vert_{H_{A^\ast}(\Omega_h)}
\label{eq:inverted_infsup}
\ee
for all $v$ $L^2$-orthogonal to $\mathcal{N}(A^\ast\vert_V)$.

{\bf Step 1:} Consider first a special case when $A^\ast_h v = 0$. Consider a conforming $u \in 
(\mathcal{N}(A\vert_U))^\perp \subset U$ such
that $A u = v$. Since $v \in (\mathcal{N}(A^\ast\vert_V))^\perp$, such a $u$ exists. We then have
\[
\begin{array}{ll}
\Vert v \Vert^2 & \ds = (A u, v) = \underbrace{(u,A^\ast_h v)}_{= 0} + c_h(\text{tr } u,v) \\[12pt]
& \ds \le \frac{\vert  c_h(\text{tr } u,v) \vert}{\Vert \text{tr } u \Vert } \: \Vert \text{tr } u \Vert \\[12pt]
&\ds \leq \sup_{\hat{u}} \frac{\vert  c_h(\hat{u},v) \vert}{\Vert \hat{u} \Vert } \: \Vert u \Vert_{H_A(\Omega)}
\\[12pt]
& \ds \leq \frac{1}{\gamma} \sup_{\hat{u}} \frac{\vert  c_h(\hat{u},v) \vert}{\Vert \hat{u} \Vert } \Vert v \Vert.
\end{array}
\]
Dividing both sides by $\Vert v \Vert$, we get the required inequality.

{\bf Step 2:} Now let $v$ be arbitrary. Consider a conforming $\tilde{v} \in H_{A^\ast}(\Omega)$ such that
$A^\ast \tilde{v} = \ A^\ast_h v$. By {Assumption 1}, such a function always exists and can
be interpreted as a solution to the strong adjoint problem with non-homogeneous BC data $f_{D} = C^\prime_2 \tilde{v}$:
\[
\left\{
\begin{array}{ll}
A^\ast \tilde{v}  = A^\ast_h v \\[8pt]
C^\prime_2 \tilde{v}  = f_{D}.
\end{array}
\right.
\]
To ensure uniqueness and boundedness in the $L^2$-norm, we assume that $\tilde{v}$ is $L^2$-orthogonal
to the null space $\mathcal{N}(A^\ast\vert_V)$.

Now, by construction, $A_h( v - \tilde{v}) = 0$ and 
$v - \tilde{v} \in (\mathcal{N}(A^\ast\vert_V))^\perp$
so, by the Step 1 result,
the difference $ v - \tilde{v}$ is bounded in both $L^2$ and 
$H_{A^\ast}$ norms by the supremum in~(\ref{eq:inverted_infsup}). We thus need only demonstrate that
we can control the norm of the conforming $\tilde{v}$. But, if we restrict ourselves
in~(\ref{eq:inverted_infsup}) to conforming test functions, the bilinear form collapses 
to~(\ref{eq:bilinear_form_ultraweak_formulation}).

This finishes the proof.
\end{proof}


\subsection{DPG formulation for the Stokes problem}

We begin by emphasizing the global character of decomposition of traces in~(\ref{eq:trace_decomposition}).
Velocity trace $\hat{\bfu} \in \bfH^{1/2}(\Gamma_h)$ can be decomposed into 
an extension of the velocity trace on the boundary $\Gamma$
and the trace on the internal skeleton $\Gamma_h^0$:
\[
\hat{\bfu} = \hat{E} \hat{\bfu}_0 + \hat{\tilde{\bfu}}
\]
In FE computations, traces are approximated with functions that are globally continuous
on the skeleton $\Gamma_h$.
The trace $\hat{\bfu}_0$, which is known from the boundary condition, has to be lifted to the whole skeleton.
In computations, we use FE shape functions and lift $\hat{\bfu}_0$ only into the
 layer of elements neighboring $\Gamma$.
The unknown part of velocity trace $\hat{\tilde{\bfu}} \in \tilde{\bfH}^{1/2}(\Gamma^0_h)$ 
lives on the internal skeleton only.

The unknown traction trace $\hat{\bft} \in \bfH^{-1/2}(\Gamma_h)$ lives on the whole skeleton.
On the continuous level, the decomposition of traction into 
a lift of its restriction to $\Gamma$ and the remaining component $\hat{\tilde{\bft}} 
\in \tilde{\bfH}^{-1/2}(\Gamma)$ that lives on the internal skeleton $\Gamma_h^0$ is also global.
In 2D for instance, 
for $\bft$ from a {\em standard} boundary space $\bfH^{-1/2}(\Gamma)$, the corresponding
restriction to an edge $e$ of an element $K$ adjacent to boundary $\Gamma$ lives only in $\bfH^{-1/2}(e)$
and {\em cannot} just be extended by zero to a functional in $\bfH^{-1/2}(\ptl K)$.
However, the conformity present in the definition of space $\bfH^{-1/2}(\Gamma_h)$ is so weak 
that it does
not translate into any global continuity conditions for the approximating polynomial spaces that
are discontinuous from edge to edge.

For the Stokes problem, the boundary operators represent exactly the velocity and traction components
of the solution trace;
\[
C_1 (\hat{\bft},\hat{\bfu}) = \bfu,\quad C_2 (\hat{\bft},\hat{\bfu}) = \hat{\bft}.
\]
{
The difference between the strong and weak imposition of BCs is, in our case, insignificant.
The abstract $\langle f_D,v \rangle$ term corresponds to $\langle\bfu_D,\bfr\rangle_\Gamma$,
where $\bfr$ is the traction component of the test function.
Its extension to discontinuous test functions $v$ is constructed by lifting Dirichlet data $\bfu_D$ to the
whole skeleton. The strong imposition of the BCs is essentially the same. The abstract lift $\hat{u}_0$
of $\bfu_D$ can be selected to be $(\bfu_D,{\bf 0})$ (zero traction) and, if we use the same extension
of $\bfu_D$ to the whole skeleton, the two formulations will be identical. A subtle difference
lies in the way we treat those lifts in FE computations. Assume, e.g., that we approximate
traces with quadratics.
Assume that we have a non-polynomial
data $u_D$. With the strong imposition of BCs, we first interpolate the data with quadratics
and use quadratic shape functions to lift it to the whole skeleton. The contributions
to the load vector will be computed by integrating the quadratic lifts against test functions.
With the weak imposition of the BCs, the non-polynomial $\bfu_D$ on $\Gamma$
will be integrated directly against the test functions and, in general, will yield different
values. Additionally, 
even if we lift the non-polynomial $\bfu_D$ to the whole skeleton
with the same quadratic shape functions, the lifts will differ on the internal skeleton and,
consequently, the resulting approximate traces will differ. In the limit, of course,
the difference will disappear.}



The null space of conjugate operator $B'$ coincides with the null space of adjoint $A^\ast$ with homogeneous
BC $\bfv = {\bf 0}$ on $\Gamma$ and consists of constant pressures
\[
\{ ({\bf 0},c,{\bf 0}) \: : \: c \in \doubleIR \}.
\]
The null space of operator $B$ is the same as that for the operator corresponding to the
ultra-weak formulation,
\[
\{ ((\bfu,p,\bfsig),\hat{\bft}) \: : \: \bfu = {\bf 0}, p = c , \bfsig = {\bf 0}, \hat{\bft} = c \bfn 
\text{ where } c \in \doubleIR \}.
\]
The non-trivial null spaces imply the compatibility condition for the load and non-uniqueness of the solution.
The compatibility condition for the load involves the right-hand side $g$
of the divergence equation\footnote{$g=0$ in practice.} and
the velocity trace BC data $\bfu_D$, and takes the form 
\[
\int_\Omega g = \int_{\Gamma} \bfu_D \cdot \bfn.
\]
This well-known condition can be obtained immediately by integrating the divergence equation
and using the boundary condition on $\bfu$:
\[
\int_\Omega g = \int_\Omega \text{div } \bfu = \int_\Gamma \bfu \cdot \bfn = \int_\Gamma \bfu_D \cdot \bfn.
\]
Assumption 1 thus reduces to the condition that the divergence operator is surjective, a well-known
fact.

With data satisfying the compatibility condition, the solution (pressure and tractions) is determined
up to a constant. In computations, the constant can be fixed by implementing an additional scaling
condition. We can enforce, for instance, zero average pressure in one particular element, or zero
average normal traction on a particular edge. The scaling will affect the ultimate values for pressure
and tractions, but has no effect on the velocity or on its gradient and trace.

\subsection{A summary}

We have presented a general theory on well-posedness for DPG variational formulation for an arbitrary
system of differential operators represented by an abstract operator $A$. 
We have made a number of assumptions, which we now summarize.
\begin{itemize}
  \item Operator $A$ and its formal adjoint $A^\ast$ are surjective ({Assumption 1}).
  \item Both energy graph spaces $H_A(\Omega)$ and $H_{A^\ast}(\Omega)$ admit corresponding
trace spaces $\widehat{H}_A(\ptl \Omega),\widehat{H}_{A^\ast}(\ptl \Omega)$.
  \item The boundary bilinear term $c(u,v)$ resulting from integration by parts is definite. 
  \item Boundary operator $C_1$ has been selected in such a way that {Assumption 2}
 is satisfied.
  \item With homogeneous boundary condition $C_1 u = 0$ in place, operator $A$ is bounded below
in the $L^2$-orthogonal complement of its null space ({Assumption 3}).
\end{itemize}
With these conditions satisfied, the DPG formulation is well-posed. The corresponding inf-sup constant
is mesh-independent.
Sweeping technical details under the carpet, this is the main
take-home message: {\em boundedness below of the strong operator with homogeneous BC implies
the inf-sup condition for the DPG formulation with a mesh-independent constant.}

The general theory is guiding us how to select unknown traces on the skeleton.
The energy setting involves graph norms for both operator $A$ and its formal adjoint $A^{\ast}$. The graph norm
in the test space is equivalent to the {\em optimal test norm} \cite{DPG4} with {\em mesh-independent} 
equivalence constants. The graph norm for $A$ implies the energy setting for unknown traces and
the minimum energy extension norm.

All these conditions are satisfied for the Stokes problem.

\subsection{Boundedness Below of the First-Order Stokes Operator with Homogeneous BCs}
\label{sec:boundedness_below}
Finally, for completeness we show that the operator corresponding to our first-order Stokes problem with homogeneous BCs is bounded below.  While in the above we have taken $\mu=1$, here we consider the more general case of constant $\mu > 0$.

Recall the classical strong form of the Stokes problem:

\begin{align}
- \mu \Delta \vect{u} + \NVRgrad p &= \vect{f} & \text{ in } \Omega, \label{NVR:eq:strongfirst1} \\
\NVRdiv \vect{u} &= 0 & \text{ in } \Omega, \\
\vect{u} &= \vect{u}_D & \text{ on } \partial\Omega.\label{NVR:eq:stronglast1}
\end{align}

Recall also our first-order system for the Stokes equations:
\begin{subequations}
\eqnlab{StokesFirstOrder}
\begin{align}
- \NVRdiv \NVRtensor{\sigma} + \NVRgrad p &= \vect{f} & \text{ in
} \Omega, \label{NVR:eq:strongfirst2} \\ \NVRdiv \vect{u} &=
\NVRtensor{0} & \text{ in } \Omega, \\ \frac{1}{\mu}\NVRtensor{\sigma} - \NVRgrad
\vect{u} &= 0 & \text{ in } \Omega,\\ \vect{u} &= \vect{u}_D & \text{ on
} \partial\Omega.\label{NVR:eq:stronglast2}
\end{align}
\end{subequations}

If we introduce $\A : \HA \rightarrow \L$
with group variable $u = (\vect{u},p,\NVRtensor{\sigma})$,  the Stokes equation in first order form \eqnref{StokesFirstOrder}, ignoring the boundary conditions,
can be written succinctly as
\begin{align}
\A u \NVReqdef \left[\begin{array}{c}
- \NVRdiv \NVRtensor{\sigma} + \NVRgrad p\\
\NVRdiv \vect{u}\\
\frac{1}{\mu}\NVRtensor{\sigma} - \NVRgrad \vect{u}\end{array}\right]
= \left[\begin{array}{c}
\vect{f} \\
0 \\
\NVRtensor{0}
\end{array}\right].\label{NVR:eq:operatorAdefined}
\end{align}

Considering linear operator $\A$ as defined in equation (\ref{NVR:eq:operatorAdefined}) operating on group variable $u = (\vect{u},p,\NVRtensor{\sigma})$, we seek to show that $||\A u||_{L^{2}} \geq \gamma ||u||_{\HA}$.\\

\subsubsection{$||\A u||_{\Omega} \geq \gamma ||u||_{\HA}$}
Adding right-hand sides $g$ and $\NVRtensor{h}$

\begin{align}
-\NVRdiv \NVRtensor{\sigma} + \NVRgrad p &= \vect{f} & \text{ in } \Omega, \label{NVR:eq:strongfirst3} \\
\NVRdiv \vect{u} &= g & \text{ in } \Omega, \\
\frac{1}{\mu}\NVRtensor{\sigma} - \NVRgrad \vect{u} &= \NVRtensor{h} & \text{ in } \Omega, \label{NVR:eq:strongsecondtolast3} \\
\vect{u} &= \vect{0} & \text{ on } \partial\Omega,\label{NVR:eq:stronglast3}
\end{align}
we have that $\A u = (\vect{f},g,\NVRtensor{h})$.  If we can establish bounds for the $L^{2}$ norms of each of the solution variables in terms of $L^{2}$ norms on $\vect{f},g,$ and $\NVRtensor{h}$, then we will have the required lower bound $|| \A u ||_{\Omega} \geq \gamma || u ||_{\HA}$, where $|| u ||_{\HA} \NVReqdef \left( || \A u ||_{\Omega}^{2} + || u ||_{\Omega}^{2}  \right)^{1/2}$.\\

Note that, by linearity of $\A$, it suffices to consider cases in which only one of $\vect{f},g,$ and $\NVRtensor{h}$ is non-zero.  We consider each of these cases in the following three subsections.\\

\subsubsection{$\vect{f} \neq \vect{0},g=0,\NVRtensor{h}=\NVRtensor{0}$}
In this case, we have exactly the system (\ref{NVR:eq:strongfirst2})-(\ref{NVR:eq:stronglast2}), which reduces (in a distributional sense) to (\ref{NVR:eq:strongfirst1})-(\ref{NVR:eq:stronglast1}).  Testing the first equation with the velocity $\vect{u} \in \vect{H}^{1}_{0}(\Omega)$, we have
\begin{align*}
- (\mu \Delta \vect{u}, \vect{u})_{\Omega} + (\NVRgrad p, \vect{u})_{\Omega} &= (\vect{f},\vect{u})_{\Omega}.\\
\end{align*}
Integrating by parts, we obtain:
\begin{align*}
(\mu \NVRgrad \vect{u}, \NVRgrad \vect{u})_{\Omega} - \langle \mu \NVRgrad \vect{u} \cdot \vect{n}, \vect{u} \rangle_{\partial \Omega} - (p, \NVRdiv u)_{\Omega} + \langle p, \vect{u} \cdot \vect{n} \rangle_{\partial \Omega} &= (\vect{f},\vect{u})_{\Omega}.
\end{align*}

Noting that $\vect{u} = 0$ on $\partial \Omega$ and that $\NVRdiv \vect{u} = 0 $ in $\Omega$, this reduces to:

\begin{align*}
\mu (\NVRgrad \vect{u}, \NVRgrad \vect{u})_{\Omega} = \mu ||\NVRgrad \vect{u}||^{2}_{\Omega} = \mu || \NVRtensor{ \sigma }||^{2}_{\Omega} &= (\vect{f},\vect{u})_{\Omega} \\
&\leq || f ||_{\Omega} || \vect{u} ||_{\Omega}\\
&\leq C_{P} || f ||_{\Omega} || \NVRgrad \vect{u} ||_{\Omega},
\end{align*}
where $C_{P}$ is the Poincar\'{e} constant.  Thus $ || \NVRtensor{ \sigma }||_{\Omega} \leq \frac{C_{P}}{\mu} || f ||_{\Omega}$, and $|| \vect{u} || \leq \frac{C_{P}^{2}}{\mu} || f ||_{\Omega}$.

To bound the pressure $p$, we require a result by Ladyzhenskaya \cite{Ladyzhenskaya} for $p \in L^{2}_{0} \NVReqdef \{w \in L^{2}(\Omega) : \int_{\Omega} w = 0 \}$:
\begin{align*}
\sup_{\vect{v} \in \vect{H}^{1}(\Omega)} \frac{(p, \NVRdiv \vect{v})_{\Omega}}{\norm{ \vect{v} }_{H^{1}(\Omega)} } \geq \beta \norm{p}_{\Omega}
\end{align*}
for some constant $\beta > 0$.  Testing equation (\ref{NVR:eq:strongfirst1}) with $\vect{v} \in \vect{H}^{1}_{0}(\Omega)$, we have
\begin{align*}
(\NVRgrad p, \vect{v})_{\Omega} = (\vect{f}, \vect{v})_{\Omega} + (\mu \Delta \vect{u}, \vect{v})_{\Omega}.
\end{align*}

Integrating by parts, dividing by $\norm{ \vect{v} }_{H^{1}(\Omega)}$ and taking the supremum:
\begin{align*}
\sup - \frac{(p, \NVRdiv \vect{v})}{\norm{ \vect{v} }_{H^{1}(\Omega)}} &= \sup \left\{\frac{(\vect{f}, \vect{v})}{\norm{ \vect{v} }_{H^{1}(\Omega)}} - \frac{\mu (\NVRgrad \vect{u}, \NVRgrad \vect{v})}{\norm{ \vect{v} }_{H^{1}(\Omega)}}\right\} \\
&\leq \frac{\norm{\vect{f}} \norm{\vect{v}}_{L^{2}}}{\norm{ \vect{v} }_{H^{1}(\Omega)}} + \mu \norm{\NVRgrad \vect{u}} \leq \norm{\vect{f} } + \mu \frac{C_{P}}{\mu} \norm{\vect{f}} = (1 + C_{P}) \norm{\vect{f}}.
\end{align*}
By the Ladyzhenskaya result, we then have $\norm{p} \leq \frac{1 + C_{P}}{\beta} \norm{\vect{f}}$, bounding $p$, as required.

\subsubsection{$\vect{f} = \vect{0}, g \neq 0,\NVRtensor{h}=\NVRtensor{0}$} In this case, we have
\begin{align}
- \NVRdiv \NVRtensor{\sigma} + \NVRgrad p &= \vect{0} & \text{ in } \Omega, \label{NVR:eq:gstrongfirst} \\
\NVRdiv \vect{u} &= g & \text{ in } \Omega, \\
\frac{1}{\mu} \NVRtensor{\sigma} - \NVRgrad \vect{u} &= 0 & \text{ in } \Omega, \label{NVR:eq:gstrongsigmadef}\\
\vect{u} &= \vect{0} & \text{ on } \partial\Omega.\label{NVR:eq:gstronglast}
\end{align}

We must also assume compatibility between $g$ and the boundary condition on $\vect{u}$; that is, that $g$ has zero average on $\Omega$:
\begin{align*}
\int_{\Omega} g = \int_{\partial \Omega} \vect{u} \cdot \vect{n} = 0.
\end{align*}

Now, by surjectivity of $\NVRdiv : \NVRHdiv \rightarrow L^{2}$, there exists $\vect{u}_{0}$ such that $\NVRdiv \vect{u}_{0} = g$.  By the Lax-Milgram theorem, we have $\exists ! \phi \in H^{1}_{0} : \vect{u}_{0} = \NVRgrad \phi$, so that $\Delta \phi = g$ and, assuming that $\Omega$ is convex, then by the Elliptic Regularity theorem (see \cite[p. 214]{Folland} for a proof), $\phi \in H^{2}$ and:
\begin{align*}
\norm{\phi}_{H^{2}} &\leq C \norm{g}\\ % SEE ARBOGAST/BONA, p. 223 (Elliptic Regularity, section 8.5.4, Theorem 8.13)---references [GT] or [Fo] for a proof
\implies \norm{\NVRgrad \phi}_{H^{1}} &\leq C \norm{g},
\end{align*}
for some $C > 0$ independent of $g$ and $\phi$.  That is, in fact $\vect{u}_{0} \in H^{1}$ and $\norm{ \vect{u}_{0} }_{H^{1}} \leq C \norm{g}$.

Setting $\vect{w} = \vect{u} - \vect{u}_{0}$, we have $\NVRdiv \vect{w} = 0$, and $\vect{w} = 0$ on $\partial \Omega$.  Note that we can recover $-\mu \Delta \vect{u} + \NVRgrad p = 0$ by substituting (\ref{NVR:eq:gstrongsigmadef}) into (\ref{NVR:eq:gstrongfirst}).  Testing this with $\vect{w}$ and integrating by parts, we obtain:
\begin{align*}
\mu (\NVRgrad \vect{u},\NVRgrad \vect{w})_{\Omega} &= 0\\
\mu (\NVRgrad \vect{w} + \NVRgrad \vect{u}_{0},\NVRgrad \vect{w})_{\Omega} &= 0\\
\implies \norm{\NVRgrad \vect{w}}^{2} &= -(\NVRgrad \vect{u}_{0}, \NVRgrad \vect{w}) \\
&\leq \norm{ \NVRgrad \vect{u}_{0} } \norm{ \NVRgrad \vect{w} } \\
&\leq C \norm{g} \norm{ \NVRgrad \vect{w} },
\end{align*}
so that $\norm{ \NVRgrad \vect{w} } \leq  C \norm{ g }$, and therefore as before we obtain a bound on $\norm{\NVRtensor{\sigma}} = \norm{\NVRgrad \vect{u}} \leq 2 C \norm{g}$, and again by means of the Poincar\'{e} inequality we may bound $\norm{\vect{u}} \leq C_{P} \norm{\NVRgrad \vect{u}} \leq 2 C C_{P} \norm{ g }$ as well.

The bound on $p$ is established exactly as before, except that now $\vect{f} = \vect{0}$, so that we obtain the bound $\norm{p} \leq \frac{\mu}{\beta} \norm{\NVRgrad \vect{u}} \leq \frac{2 \mu C C_{P}}{\beta} \norm{ g }$.


\subsubsection{$\vect{f} = \vect{0},g=0,\NVRtensor{h} \neq \NVRtensor{0}$}  In this case, we have
\begin{align}
- \NVRdiv \NVRtensor{\sigma} + \NVRgrad p &= \vect{0} & \text{ in } \Omega, \label{NVR:eq:hstrongfirst} \\
\NVRdiv \vect{u} &= 0 & \text{ in } \Omega, \\
\frac{1}{\mu} \NVRtensor{\sigma} - \NVRgrad \vect{u} &= \NVRtensor{h} & \text{ in } \Omega, \\
\vect{u} &= \vect{0} & \text{ on } \partial\Omega.\label{NVR:eq:hstronglast}
\end{align}
Then, $\NVRtensor{\sigma} = \mu(\NVRgrad \vect{u} + \NVRtensor{h})$, and we have:
\begin{align*}
- \NVRdiv \NVRtensor{\sigma} + \NVRgrad p = - \mu \Delta \vect{u} - \mu \NVRdiv \NVRtensor{h} + \NVRgrad p &= \vect{0},
\end{align*}
so that
\begin{align*} - \mu \Delta \vect{u} + \NVRgrad p &= - \mu  \NVRdiv \NVRtensor{h}
\end{align*}
in a distributional sense.  Testing with $\vect{u}$, and integrating the left hand side by parts, again the pressure term vanishes because $\NVRdiv \vect{u} = 0$, so that much as before, we obtain:
\begin{align*}
\mu (\NVRgrad \vect{u}, \NVRgrad \vect{u})_{\Omega} &= \mu (\NVRdiv \NVRtensor{h}, \vect{u} )_{\Omega}\\
&= - \mu(\NVRtensor{h}, \NVRgrad \vect{u})_{\Omega} \leq \mu \norm{\NVRtensor{h}}_{L^{2}(\Omega)} \norm{ \NVRgrad \vect{u} }_{L^{2}(\Omega)}.
\end{align*}
So $\norm{\NVRgrad \vect{u}} \leq \norm{\NVRtensor{h}}$, and the bounds $\norm{\vect{u} } \leq C_{P} \norm{\NVRtensor{h}}$ and $\norm{p} \leq \frac{\mu}{\beta}\left( \norm{ \NVRgrad \vect{u} } + \norm{\NVRtensor{h}} \right) \leq \frac{2 \mu}{\beta} || \NVRtensor{h} ||$ can be established in a similar fashion as above.
