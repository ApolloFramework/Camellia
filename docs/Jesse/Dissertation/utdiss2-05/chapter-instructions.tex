\chapter{Instructions for Preparing Dissertations, Theses, and Reports}
\index{Instructions for Preparing Dissertations, Theses, and Reports%
@\emph{Instructions for Preparing Dissertations, Theses, and Reports}}%

We are not going to look at the complete set of instructions contained
in \emph{Instructions for Preparation of Doctoral Dissertations and
Dissertation Abstracts} or \emph{Format For The Master's Thesis and Report}
which can be obtained from the Office of Graduate Studies (OGS)
\index{Office of Graduate Studies}%
or on their web page,
\index{Office of Graduate Studies web page}%
\url{http://www.utexas.edu/ogs}.
The doctoral Instructions I am using are dated March, 2001.
The master's Format I am using is dated May, 2001.

Here we will look at a few instructions related to the arrangement of the
dissertation, thesis, or report and a few other ``technical'' details,
providing some examples of common \LaTeX\ usage and some examples of
not-so-common \LaTeX\ usage.

The following are just a couple of tests for the ``quote''
and ``quotation'' environments. The following paragraph is a quote.
\begin{quote}
\index{quote}%
This template package is provided and licensed
``as is'' without warranty of any kind, either expressed or
implied, including, but not limited to, the implied warranties
of merchantability and fitness for a particular purpose.
\end{quote}
The following paragraph is a quotation.
\begin{quotation}
\index{quotation}%
This template package is provided and licensed
``as is'' without warranty of any kind, either expressed or
implied, including, but not limited to, the implied warranties
of merchantability and fitness for a particular purpose.
\end{quotation}

The OGS Instructions say prose quotations over four lines
should be indented on the left. The Doctoral Degree Evaluator says
the quote environment is the correct one to use.

\section{Arrangement of Dissertation}
\index{Arrangement of Dissertation@\emph{Arrangement of Dissertation}}%

Always remember that this ``fake'' dissertation 
\index{fake dissertation}%
is only intended to be a template for writing your own. Since the ultimate
responsibility of making sure your dissertation meets the Graduate School's
requirements, however, lies only with you, you \textbf{\textit{must}} get
the current \emph{Instructions for Preparation of Doctoral Dissertations and
Dissertation Abstracts} from the Office of Graduate Studies or their web
page and check everything yourself. If you don't, you may have a very
rude awakening from the Lynn Renegar, Doctoral Degree Evaluator (aka,
``The Ruler Lady'') at a most inopportune time.

Arrange your dissertation as follows (all sections are required 
unless said otherwise.

\begin{enumerate}

\item Fly Page 
\index{Fly Page}%
(blank protective page). This page is \textbf{not} counted in the numbering.
\textbf{Note:} This template does not insert a Fly Page; if you are printing
an official copy, you must manually insert a blank piece of paper on your own.
Electronic documents do not need a fly page.

\item Copyright Legend 
\index{Copyright Legend}%
(optional) - See OGS Instructions Sample Form A.
Begin counting \textbf{pretext} pages here, but \textbf{do not
place a number on this page.} 

\item Committee Certification of Approved Version.
\index{Committee Certification of Approved Version}%
See OGS Instructions Sample Form B. This page is included in the
pretext count, but there should be no page number on the page.

\item Title Page - 
\index{Title Page}%
See OGS Instructions Sample Form C.
This page is counted, but there should not be a page number on this page.

\item Dedication 
\index{Dedication}%
and/or Epigraph (optional). 
\index{Epigraph}%
Included in count, but not numbered.

\item Acknowledgments
\index{Acknowledgments}%
or Preface 
\index{Preface}%
(optional) - Begin showing \textbf{pretext} page numbers with \textbf{lower
case Roman numerals} at bottom of page.

\item Abstract 
\index{Abstract}%
(optional) - See OGS Instructions Sample Form D.

\item Table of contents -
\index{Table of contents}%
List ALL sections which follow it. There are too may different ways a table
of contents may be done for the OGS to give examples in their Instructions
booklet, but do be sure there is agreement between the major headings in
your text and their designations in the Table of Contents (fortunately \LaTeX{}
does this for you automatically). Please ask the Doctoral Degree Evaluator
for assistance if necessary.

\item List of Tables,
\index{List of Tables}%
List of Figures,
\index{List of figures}%
List of Illustrations, 
\index{List of Illustrations}% 
Nomenclature,
\index{Nomenclature}%
List of Supplemental Files (such as multimedia files)
(optional).

\item Text. 
\index{Text}%
The text should be divided into chapters, books or sections. The first page
is Arabic numeral \textbf{``1''}. All sections, \textbf{from the first page
of text through Vita,} should be numbered consecutively.

\item If you group all Tables, 
\index{Tables}%
Figures, 
\index{Figures}%
or Illustrations 
\index{Illustrations}%
in one place in your dissertation, the section should be placed 
here, immediately after the text and before any appendices 
(optional).

\item Appendix or Appendices 
\index{Appendix}%
\index{Appendices}%
(optional).

\item Glossary 
\index{Glossary}%
(optional) - this section may be placed either here or after the Table
of Contents, in the area with List of Tables, List of Figures...

\item Bibliography
\index{Bibliography}%
- consult your supervisor about which recognized style to use.

\item Index
\index{Index}%
(optional).

\item Vita -
\index{Vita}%
This should be a brief biographical sketch of the author. List in the
Table of Contents. See OGS Instructions Sample Form E.


\end{enumerate}


\section{Other Requirements}
\index{Other Requirements@\emph{Other Requirements}}%


\subsection{Margins}
\index{Margins@\emph{Margins}}%

The dissertation, after printing, should have left and top margins of
1~1/2 inches, and the right and bottom margins should be 1~1/4 inches.
These margins should be consistent throughout the dissertation - including
all pages in the appendix. \textbf{All page numbers must be \textit{at
least} one inch from the edges of the page}. Headers are rarely used in
dissertations; if you are considering using them, check with the Doctoral
Degree Evaluator first to be sure they will be accepted.


\subsection{Spacing and Page Arrangement}
\index{Spacing and Page Arrangement@\emph{Spacing and Page Arrangement}}
\index{Spacing}%
\index{Page Arrangement}%

The document should be double-spaced or space-and-a-half. 
Exceptions to double-spacing are: the Table of Contents, Lists of Tables, 
Tables, Figures, Graphs, Captions, Footnotes, Endnotes, 
Appendices, Glossary, Bibliography and Index; these may 
be single-spaced. Paragraph indentations are usually five to ten 
spaces. Prose quotations over four lines should be in 
block quote (double or single spaced, indented on the left). 
Do not use quotation marks if the quotation is indented except for
quotations within the block quote. Please refer to a style manual for
more detailed instructions.

Be sure that each new chapter or major section (i.e., Appendix, Bibliography,
Vita) begins on a new page.

\section{Master's Theses and Reports}

Always remember that this ``fake'' thesis or report 
\index{fake thesis or report}%
--- assuming you have followed the instructions in the next chapter about
how to format it as such --- is only intended to be a template for writing
your own. Since the ultimate responsibility of making sure your thesis or
report meets the Graduate School's requirements, however, lies only with
you, you \textbf{\textit{must}} get the current \emph{Format For The
Master's Thesis and Report} from the Office of Graduate Studies or their web
page and check everything yourself. If you don't, you may have a very
rude awakening from the Mike Feissli, Master's Degree Evaluator at a most
inopportune time.

That said, the formatting requirements for Dissertations and Reports and
Theses are very similar. They are, however, \textbf{\textit{not}} identical.
The primary differences are in the ordering of the title and signature
pages and where the optional index is inserted. For Master's Theses
and Reports, the Title Page must be in front of the Signature Page. For
Master's Theses and Reports, \textbf{\textit{nothing}} is permitted to
come between the bibliography and the vita; the index, if used, must be
before the bibliography. If you want to use an index, talk with Mike
Feissli before your deadline to verify that its inclusion is
acceptable. The index can be removed by commenting out one line with a
percent sign, if necessary, for producing the ``official'' copy of your
thesis or report, and then inserted for copies for your advisor and
you by removing the percent sign.

