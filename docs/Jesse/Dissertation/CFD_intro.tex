For linear problems, finite difference (FD) methods approximate derivatives based on interpolation of pointwise values of a function.  FD methods were popularized first by Lax, who introduced the concepts of the monotone scheme and numerical flux. For the conservation laws governing compressible aerodynamics, FD methods approximated the conservation law, using some numerical flux to reconstruct approximations to the derivative at a point. Finite volume (FV) methods are similar to finite difference methods, but approximate the integral version of a conservation law as opposed to the differential form. FD and FV have roughly the same computational cost/complexity; however, the advantage of FV methods over FD is that FV methods can be used on a much larger class of discretizations than FD methods, which require uniform or smooth meshes. 

For nonlinear shock problems, the solution often exhibits sharp gradients or discontinuities, around which the solution would develop spurious Gibbs-type oscillations. Several ideas were introduced to deal with oscillations in the solution near a sharp gradient or shock: artificial viscosity parameters, total variation diminishing (TVD) schemes, and slope limiters. However, each method had its drawback, either in terms of loss of accuracy, dimensional limitations, or problem-specific parameters to be tuned \cite{Shu:Lectures}. Harten, Enquist, Osher and Chakravarthy introduced the essentially non-oscillatory (ENO) scheme in 1987 \cite{ENO}, which was improved upon with the weighted essentially non-oscillatory (WENO) scheme in \cite{WENO}. WENO remains a popular choice today for both finite volume and finite difference schemes. 

Historically, finite volumes and finite difference methods have been the numerical discretizations of choice for CFD applications; the simplicity of implementation of the finite difference method allows for quick turnaround time, and the finite volume method is appealing due to its locally conservative nature and flexibility. More recently, the finite element method (FEM) has gained popularity as a discretization method for CFD applications for its stability properties and rigorous mathematical foundations. Early pioneers of the finite element method for CFD included Zienkiewicz, Oden, Karniadakis, and Hughes \cite{ChungCFDBook}.

%Well-adapted to parabolic PDEs, finite element methods (FEM) gained early acceptance in engineering mechanics problems, though FEM became popular for the hyperbolic problems of CFD later than FD and FV methods. Early pioneers of the finite element method for CFD included Zienkiewicz, Oden, Karniadakis, and Hughes \cite{ChungCFDBook}. Of special note is the Streamline Upwind Petrov-Galerkin method (SUPG) of Brooks and Hughes \cite{SUPG}, which biases the standard test function by some amount in the direction of convection. SUPG is still the most popular FEM stabilization method to date. 

