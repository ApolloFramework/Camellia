%Convection-diffusion problems are problems in which the two transport phenomena convection and diffusion are both present.  Convection is, broadly speaking, the transport of substances or information driven by fluid motion -- for example, the convection of heat via air currents.  Diffusion, in contrast, is the transport of substances or information driven by primarily by concentration gradients -- for example, the diffusion of heat in water is the flow of heat from areas of higher temperature to areas of lower temperature, independent of fluid motion.  Early numerical methods for transport were typically designed for either convection of diffusion in a mutually exclusive manner -- methods for solving a convection problem were typically unstable when solving a diffusion problem, and vice-versa.  Finite element methods in particular, while well-suited to diffusion problems, proved to be ill-suited and unstable for convection problems and convection-dominated diffusion problems.  To address this issue, stabilization terms were introduced to remedy issues with standard finite element methods in the convective regime.  In particular, the SUPG method of Brooks and Hughes \cite{SUPG} introduced the notion of residual-based stabilization to combat such oscillations, and remains one of the most successful methods for solving convection-diffusion problems in the convection-dominated regime.  

%The method of focus is the Discontinuous Petrov-Galerkin method, a method introduced in 2009 by Leszek Demkowicz and Jay Gopalakrishnan.  This method falls under the larger class of Dual Petrov-Galerkin methods, in which the minimization of a specific residual (posed in a dual norm) leads naturally to a Petrov-Galerkin formulation, where test functions are automatically tailored in a problem-specific manner.  In particular, the subject of this work is the application and mathematical analysis of DPG as a stable, automatically adaptive higher order finite element method to problems in Computational Fluid Dynamics (CFD).  

Over the last three decades, CFD simulations have become commonplace as a tool in the engineering and design of high-speed aircraft.  Experiments are often complemented by computational simulations, and CFD technologies have proved very useful in both the reduction of aircraft development cycles, and in the simulation of conditions difficult to reproduce experimentally.  Great advances have been made in the field since its introduction, especially in areas of meshing, computer architecture, and solution strategies.  Despite this, there still exist many computational limitations in existing CFD methods; in particular, reliable higher order and $hp$-adaptive methods for the Navier-Stokes equations that govern viscous compressible flow.

Solutions to the equations of viscous flow can display shocks and boundary layers, which are characterized by localized regions of rapid change and high gradients.  The use of adaptive meshes is crucial in such settings --- good resolution for such problems under uniform meshes is computationally prohibitive and impractical for most physical regimes of interest.  However, the construction of ``good" meshes is a difficult task, usually requiring a-priori knowledge of the form of the solution.  An alternative to such is the construction of automatically adaptive schemes; such methods begin with a coarse mesh and refine based on the minimization of error.  However, this task is difficult, as the convergence of numerical methods for problems in CFD is notoriously sensitive to mesh quality.  Additionally, the use of adaptivity becomes more difficult in the context of higher order and $hp$ methods \cite{BoeingHigherOrder}.  

Many of the above issues are tied to the notion of \emph{robustness}, which we define loosely for CFD applications as the degradation of the quality of numerical solutions on a coarse mesh with respect to the Reynolds number, or nondimensional viscosity. For typical physical conditions of interest for the compressible Navier-Stokes equations, the Reynolds number dictates the scale of shock and boundary layer phenomena, and can be extremely high --- on the order of $10^7$ in a unit domain.  For an under-resolved mesh, the Galerkin finite element method develops large oscillations which prevent convergence and pollute the solution.  

The issue of robustness for finite element methods was addressed early on by Brooks and Hughes in the SUPG method \cite{SUPG}, which introduced the idea of residual-based stabilization to combat such oscillations. Residual-based stabilizations can alternatively be viewed as modifying the standard finite element test space, and consequently the norm in which the finite element method converges. Demkowicz and Gopalakrishnan generalized this idea in 2009 by introducing the Discontinous Petrov-Galerkin (DPG) method with optimal test functions, where test functions are determined such that they minimize the discrete linear residual in a dual space.  Under the ultra-weak variational formulation, these test functions can be computed locally to yield a symmetric, positive-definite system.  %The DPG method was quickly used to solve the convection and convection-diffusion problems \cite{DPG1,DPG2,DPG3}, and has recently been shown to yield a robust higher order adaptive method for the convection-diffusion model problem in arbitrary dimensions \cite{DPGrobustness,DPGrobustness2}.  

The main theoretical thrust of this research is to develop a DPG method that is provably robust for singular perturbation problems in CFD, but does not suffer from discretization error in the approximation of test functions \cite{DPGrobustness, DPGrobustness2}.  Such a method is developed for the prototypical singular perturbation problem of convection-diffusion, where it is demonstrated that the method does not suffer from error in the approximation of test functions, and that the $L^2$ error is robustly bounded by the energy error in which DPG is optimal -- in other words, as the energy error decreases, the $L^2$ error of the solution is guaranteed to decrease as well.  The method is then extended to the linearized Navier-Stokes equations, and applied to the solution of the nonlinear compressible Navier-Stokes equations.  

The numerical work in this dissertation has focused on the development of a 2D compressible flow code under the Camellia library, developed and maintained by Nathan Roberts at ICES \cite{Camellia}.  In particular, we have developed a framework allowing for rapid implementation of problems and the easy application of higher order and $hp$-adaptive schemes based on a natural error representation function that stems from the DPG residual \cite{DPG2, DPG3}. 

Finally, the DPG method is applied to several convection diffusion problems which mimic difficult problems in compressible flow simulations, including problems exhibiting both boundary layers and singularities in stresses.  A viscous Burgers' equation is solved as an extension of DPG to nonlinear problems, and the effectiveness of DPG as a numerical method for compressible flow is assessed with the application of DPG to two benchmark problems in supersonic flow.  In particular, DPG is used to solve the Carter flat plate problem and the Holden compression corner problem over a range of Mach numbers and laminar Reynolds numbers using automatically adaptive schemes, beginning with very under-resolved/coarse initial meshes.  
