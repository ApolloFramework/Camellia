\subsection{Discontinuous Petrov-Galerkin methods with optimal test functions}
\seclab{optimalTest} Petrov-Galerkin methods, in which the test space
differs from the trial space, have been explored for over 30 years,
beginning with the approximate symmetrization method of Barrett and
Morton~\cite{BARRETT01101981}. The idea was continued with the SUPG
method of Hughes, and the characteristic Petrov-Galerkin approach of
Demkowicz and Oden~\cite{Demkowicz1986188}, which introduced the
idea of tailoring the test space to change the norm in which a finite
element method would converge.

The idea of optimal test functions was introduced by Demkowicz and
Gopalakrishnan in \cite{DPG2}.  Conceptually, these optimal test
functions are the natural result of the minimization of a residual
corresponding to the operator form of a variational equation. The
connection between stabilization and least squares/minimum residual
methods has been observed previously \cite{GLS}. However, the method in \cite{DPG2} distinguishes itself
by measuring the residual of the natural \textit{operator form of the equation}, which is posed in the dual space, and measured with the dual norm, as we now discuss. 	

Throughout this dissertation, we assume that the trial space $U$ and test space $V$ are real Hilbert spaces, and denote $U'$ and $V'$ as the respective topological dual spaces. Let $U_h \subset U$ and $V_h\subset V$ be finite dimensional subsets. We are interested in the following problem  
\begin{equation}
\eqnlab{variationEq}
\left\{
  \begin{array}{l}
    \text{Given } l \in V', \text{ find } u_h \in U_h  \text{ such that} \\ 
    b(u_h,v_h) = l(v_h), \quad \forall v_h\in V_h,
  \end{array}
  \right.
\end{equation}
where $b\LRp{\cdot,\cdot}: U \times V \to \mbb{R}$ is a continuous
bilinear form.  $U_h$ is chosen to be some trial space of approximating functions, but $V_h$ is as of yet unspecified. 

Throughout the paper, we suppose the variational
problem \eqnref{variationEq} to be well-posed. In that case, we can
identify a unique operator $B:U\rightarrow V'$ such that
\[
\langle Bu,v\rangle_V \coloneqq b(u,v), \quad u\in U, v\in V
\]
with $\LRa{\cdot, \cdot}_V$ denoting the duality pairing between $V'$ and $V$, to obtain the operator form of the continuous variational problem
\begin{equation}
\eqnlab{dualEq}
Bu = l \quad \text{in } V'.
\end{equation}
In other words, we can represent the continuous form of our variational equation
\eqnref{variationEq} equivalently as the operator equation \eqnref{dualEq} with values in the
dual space $V'$.  This motivates us to consider the conditions under which the solution to \eqnref{variationEq} is the solution to the minimum residual problem in $V'$ 
\[
u_h = \underset{u_h\in U_h}{\arg\min}\, J(u_h),
\]
where $J(w)$ is defined for $w\in U$ as 
\[
J(w) = \frac{1}{2}\|Bw-l\|_{V'}^2 \coloneqq\frac{1}{2} \sup_{v\in V\setminus\{0\}} \frac{| b(w,v)-l(v)|^2}{\nor{v}_V^2}.
\]
For convenience in writing, we will abuse the notation $\sup_{v \in V}$ to denote $\sup_{v\in V\setminus\{0\}}$ for the remainder of the paper.

Let us define $R_V: V \to V'$ as the Riesz map, which identifies
elements of $V$ with elements of $V'$ by 
\[
\langle R_V v,\delta
v\rangle_V \coloneqq(v, \delta v)_V, \quad \forall \delta v \in V.
\]
Here, $(\cdot, \cdot)_V$ denotes the
inner product in $V$. As $R_V$ and its inverse, $R_V^{-1}$, are both
isometries, e.g.\ $\|f\|_{V'} = \|R_V^{-1} f\|_V, \forall f \in V'$, we
have
\begin{equation}
\eqnlab{minimization}
\min_{u_h\in U_h} J(u_h) = \frac{1}{2}\left\|Bu_h-l\right\|_{V'}^2 =  \frac{1}{2}\left\|R_V^{-1}(Bu_h-l)\right\|_V^2.
\end{equation}
The first order optimality condition for \eqnref{minimization} requires
the G\^ateaux derivative to be zero in all directions $\delta u \in
U_h$, i\.e\.,
\begin{align*}
\left(R_V^{-1}(Bu_h-l),R_V^{-1}B\delta u\right)_V = 0, \quad \forall \delta u \in U. 
\end{align*}
We define, for a given $\delta u \in U$, the corresponding {\em optimal test function} $v_{\delta u}$
\begin{equation}
\eqnlab{optv}
v_{\delta u} \coloneqq R_V^{-1}B\delta u \quad  \text{in } V.
\end{equation} 
The optimality condition then becomes
\[
 \langle Bu_h-l, v_{\delta u}\rangle_V = 0, \quad \forall \delta u \in U
\]
which is exactly the standard variational equation in
 \eqnref{variationEq} with $v_{\delta u}$ as the test functions. We can define the optimal test space $V_{\rm opt} \coloneqq \{v_{\delta u} \text{ s.t. } \delta u\in U\}$. Thus, the solution of the variational problem \eqnref{variationEq} with test space $V_h = V_{\rm opt}$ minimizes the residual in the dual norm $\nor{Bu_h-l}_{V'}$. This is the key idea behind the concept of optimal test functions. 

Since $U_h \subset U$ is spanned by a finite number of basis functions $\LRc{\varphi_i}_{i=1}^N$, \eqnref{optv} allows us to compute (for each basis function) a corresponding optimal test function $v_{\varphi_i}$. The collection $\LRc{v_{\varphi_i}}_{i = 1}^N$ of optimal test functions then forms a basis for the optimal test space.  In order to express optimal test functions defined in \eqnref{optv} in a more familiar form, we take  $\delta u = \varphi$, a generic basis function in $U_h$, and rewrite \eqnref{optv} as
\[
R_Vv_{\varphi} = B\varphi, \quad \text{in } V',
\]
which is, by definition, equivalent to
\[
\LRp{v_\varphi,\delta v}_V = \LRa{R_Vv_\varphi,\delta v}_{V}=
\LRa{B\varphi, \delta v}_V = b\LRp{\varphi,\delta v}, \quad
\forall \delta v \in V.
\]
%% Now, let us define the \textit{trial-to-test} operator $T =
%% R_V^{-1}B$, which, by \eqnref{optv}, maps a trial function $\delta u$
%% to its corresponding \textit{optimal} test function $v = T\delta u$.
%% On the other hand,
As a result, optimal test functions can be determined by solving the auxiliary
variational problem
\begin{equation}
\eqnlab{optvVar}
\left(v_\varphi,\delta v\right)_V = b(\varphi,\delta v), \quad \forall
\delta v \in V.
\end{equation}
However, in general, for standard $H^1$ and $H({\rm div})$-conforming finite element methods, test functions are continuous over the entire domain, and hence solving variational problem \eqnref{optvVar} for each optimal test function requires a global operation over the entire mesh, rendering the method impractical. A breakthrough came through the development of discontinous Galerkin (DG) methods, for which basis functions are discontinuous over elements. In particular, the use of discontinuous test functions $\delta v$ and a \textit{localizable} norm $\nor{\cdot}_V$\footnote{A localizable norm $\nor{v}_{V(\Oh)}$ can be written in the form 
$$\nor{v}_{V(\Oh)}^2 = \sum_{K\in\Oh} \nor{v}_{V(K)}^2,$$ where $\nor{v}_{V(K)}$ is a norm over the element $K$.} reduces the problem of determining global optimal test functions in \eqnref{optvVar} to local problems that can be solved in an element-by-element fashion.

We note that solving \eqnref{optvVar} on each element exactly is infeasible since it amounts to inverting the Riesz map $R_V$ exactly. Instead, optimal test functions are approximated using the standard Bubnov-Galerkin method on an ``enriched" subspace $\tilde{V} \subset V$ such that $\dim(\tilde{V}) > \dim(U_h)$ elementwise \cite{DPG1, DPG2}. In this paper, we assume the error in approximating the optimal test functions is negligible, and refer to the work in \cite{practicalDPG} for estimating the effects of approximation error on the performance of DPG.

It is now well known that the DPG method delivers the best approximation error in the ``energy norm" --- that is \cite{Bui-ThanhDemkowiczGhattas11a, DPG2,DPG4} 
\begin{equation}
\eqnlab{optimalError}
\nor{u-u_h}_{U,E} = \inf_{w\in U_h} \nor{u-w}_{U,E},
\end{equation}
where the energy norm $\|\cdot \|_{U,E}$ is defined for a function $\varphi \in U$ as
\begin{equation}
\eqnlab{energyNorm} \|\varphi\|_{U,E} \coloneqq \sup_{v\in V}
\frac{b(\varphi,v)}{\|v\|_V} = \sup_{\nor{v}_V = 1} b(\varphi,v) =
\sup_{\nor{v}_V = 1} \LRa{B\varphi,v}_V = \nor{B\varphi}_{V'} =
\nor{v_\varphi}_V,
\end{equation}
where the last equality holds due to the isometry of the Riesz map
$R_V$ (or directly from \eqnref{optvVar} by taking the supremum). An
additional consequence of adopting such an energy norm is that,
without knowing the exact solution, the energy error $\|u-u_h\|_{U,E}  = \nor{Bu - Bu_h}_{V'} = \nor{R_V^{-1}\LRp{l-Bu_h}}_{V}$ can
be determined by computing the \textit{error representation function} $e \coloneqq R_V^{-1}\LRp{l-Bu_h}$ through
\[
\left(e,\delta v\right)_V = b(u-u_h,\delta v) = l\LRp{\delta v} - b(u_h,\delta v)
\]
and measuring its norm $\|e\|_V$.  This is simply a consequence of the least-squares nature of DPG; the energy error is simply the norm of the residual in $V'$.  Under the assumption of a localizable norm on $V$, we can compute the squared norm over the entire domain $\nor{e}_{V(\Oh)}^2$ as the sum of individual element contributions $\sum_{K\in \Oh} \nor{e}_{V(K)}^2$.  We define $e_K^2 \coloneqq \nor{e}_{V(K)}^2$ as a local error indicator with which we can drive adaptive mesh refinement.  

Practically speaking, this implies that the DPG method is discretely stable on any mesh. In particular, DPG is unconditionally stable for higher order adaptive meshes, where discrete stability is often an issue. 

\subsection{Duality between trial and test norms (energy norm pairings)}
%\subsection{Abstract energy norm pairings}
\seclab{energyPair}
A clear property of the energy norm defined by \eqnref{energyNorm} is that the trial norm $\nor{\cdot}_{U,E}$ is induced by a given test norm. However, the reverse relationship holds as well; for any trial norm, the test norm that induces such a norm is recoverable through duality. We have a result, Lemma 2.5 in \cite{Bui-ThanhDemkowiczGhattas11a}: assuming, for simplicity, that the bilinear form $b(u,v)$ is definite\footnote{By definite, we mean that \begin{align*}b(u,v) &= 0, \forall v\in V \Rightarrow u = 0\\ b(u,v)&=0, \forall u\in U \Rightarrow v = 0,\end{align*}  which imply injectivity of the bilinear operator $B$ and its transpose $B'$, defined such that $\LRa{Bu,v} = \LRa{u,B'v}$.  These conditions imply solvability of the variational problem.}, given any norm $\nor{\cdot}_{U}$ on the trial space $U$, for $\varphi \in U$, we can represent $\nor{\varphi}_{U}$ via
\[
\nor{\varphi}_{U} = \sup_{v \in V}\frac{b\LRp{w,v}}{\nor{v}_{V,U}}.
\]
where $\nor{v}_{V,U}$ is defined through
\[
\nor{v}_{V,U} = \sup_{w \in U}\frac{b\LRp{w,v}}{\nor{w}_{U}}.
\]
%In other words, the norm $\nor{\cdot}_V$ in $V$ can be recovered using the energy norm $\nor{\cdot}_{U,E}$ in $U$, and vice versa. 

In particular, given two arbitrary norms $\nor{\cdot}_{U,1}$ and $\nor{\cdot}_{U,2}$ in $U$
such that $\nor{\cdot}_{U,1} \le c \nor{\cdot}_{U,2}$ for some constant
$c$, they generate two norms $\nor{\cdot}_{V,U,1}$ and
$\nor{\cdot}_{V,U,2}$ in $V$ defined by
\[
\nor{v}_{V,U,1} \coloneqq \sup_{w \in U}\frac{b\LRp{w,v}}{\nor{w}_{U,1}}, \quad
\text{and }\nor{v}_{V,U,2} \coloneqq \sup_{w \in U}\frac{b\LRp{w,v}}{\nor{w}_{U,2}},
\]
such that $\nor{\cdot}_{V,U,1}$ and $\nor{\cdot}_{V,U,2}$ induce
$\nor{\cdot}_{U,1}$ and $\nor{\cdot}_{U,2}$ as energy
norms in $U$, respectively. That is,
\[
\|\varphi\|_{U,1} = \sup_{v\in V}
\frac{b(\varphi,v)}{\|v\|_{V,U,1}}, \quad \text{and }\|\varphi\|_{U,2} = \sup_{v\in V}
\frac{b(\varphi,v)}{\|v\|_{V,U,2}}.
\]

A question that remains to be addressed is to establish the relationship
between $\nor{\cdot}_{V,U,1}$ and $\nor{\cdot}_{V,U,2}$, given that
$\nor{\cdot}_{U,1} \le c \nor{\cdot}_{U,2}$. But this is
straightforward since we have
\begin{align*}
 \| v \|_{V,U,2} = \sup_{u \in U} \frac{b\left(w,v\right)}{\left\|
  w \right\|_{U,2}} \le c\sup_{w \in U} \frac{b\left(w,v\right)}{\left\| w
  \right\|_{U,1}} = c\| v \|_{V,U,1}.
\end{align*}
Consequently, a stronger energy norm in $U$ will generate a weaker
norm in $V$ and vice versa. In other words, to show that an
energy norm $\nor{\cdot}_{U,1}$ is weaker than another energy norm
$\nor{\cdot}_{U,2}$ in $U$, one simply needs to show the reverse inequality on the
corresponding norms in $V$, that is, $\nor{\cdot}_{V,U,1}$ is stronger
than $\nor{\cdot}_{V,U,2}$.

%From now on, unless otherwise stated, we will refer to  $\nor{\cdot}_{V,U,1}$, $\nor{\cdot}_{V,U,2}$, and $\nor{\cdot}_{V}$ as the test norms which induce $\nor{\cdot}_{U,1}$, $\nor{\cdot}_{U,2}$, and $\nor{\cdot}_{U,E}$ and vice versa, in the sense discussed above. 
From now on, unless otherwise stated, we will refer to $\nor{\cdot}_{V,U}$ as the test norm that induces a given norm $\nor{\cdot}_U$. Likewise, we will refer $\nor{\cdot}_{U,V}$ as the trial norm induced by a given test norm $\nor{\cdot}_V$. In this paper, for simplicity of exposition, we shall call a pair of norms in $U$ and $V$ that induce each other as an {\em energy norm pairing}.

\subsection{Discontinuous Petrov-Galerkin methods with the ultra-weak formulation}
\seclab{abstractUweak}

The naming of the discontinuous Petrov-Galerkin method refers to the fact that the method is a Petrov-Galerkin method, and that the test functions are specified to be discontinuous across element boundaries. There is no specification of the regularity of the trial space, and we stress that the idea of DPG is not inherently tied to a single variational formulation \cite{Bui-ThanhDemkowiczGhattas11a}. Additionally, Cohen, Dahmen and Welper simultaneously extend the minimum residual concept behind DPG to general variational settings and avoid the use of discontinuous test functions by formulating the minimum residual method as a saddle-point problem \cite{DahmenVariationalStabilization}.  

In most of the DPG literature, however, the discontinuous Petrov-Galerkin method refers to the combination of the concept of locally computable optimal test functions in Section \secref{optimalTest} with the so-called ``ultra-weak formulation" \cite{DPG1,DPG2,DPG3,DPG4,DPGElas,DBLP:journals/procedia/NiemiCC11}. Unlike the previous two sections in which we studied the general equation \eqnref{variationEq} given by abstract bilinear and linear forms, we now consider a concrete instance of \eqnref{variationEq} resulting from an ultra-weak formulation for an abstract first-order system of PDEs $Au = f$. Additionally, from this section onwards, we will refer to DPG as the pairing of the ultra-weak variational formulation with the concept of locally computable optimal test functions. 

We begin by partitioning the domain of interest $\Omega$ into $\Nel$ non-overlapping elements $K_j, j = 1,\hdots,\Nel$ such that $\Oh = \cup_{j=1}^\Nel K_j$ and $\overline{\Omega} = \overline{\Omega}_h$. Here, $h$ is defined as $h= \max_{j\in \LRc{1,\hdots,\Nel}}\text{diam}\LRp{K_j}$.  We denote the mesh ``skeleton" by $\Gh = \cup_{j=1}^\Nel \partial K_j$; the set of all faces/edges $e$, each of which comes with a normal vector ${n}_e$. The internal skeleton is then defined as $\Gamma^0_h = \Gh \setminus \partial \Omega$. If a face/edge $e \in \Gh$ is the intersection of $\partial K_i$ and $\partial K_j$, $i \ne j$, we define the following jumps:
\[
\jump{v} = \text{sgn} \LRp{{n}^-}v^- + \text{sgn} \LRp{{n}^+}v^+, \quad
\jump{\tau \cdot n} = {n}^-\cdot \tau^- + {n}^+\cdot\tau^+,
\]
where
\[
\text{sgn}\LRp{{n}^{\pm}} =
\left\{
\begin{array}{ll}
1 & \text{if } {n}^{\pm} = {n}_e \\
-1 & \text{if } {n}^\pm = -{n}_e
\end{array}
\right..
\]
For $e$ belonging to the domain boundary $\partial \Omega$, we define
\[
\jump{v} = v, \quad
\jump{\tau \cdot n} = {n}_e\cdot \tau.
\]
Note that we allow arbitrariness in assigning ``-'' and ``+'' quantities to the adjacent elements $K_i$ and $K_j$.
%For the rest of the paper, we will use the same notation for both a function and its trace (if it is well-defined) when there is no ambiguity.

The ultra-weak formulation for $Au = f$ on $\Oh$, ignoring boundary
conditions for now, reads
\begin{equation}
\eqnlab{uweak}
b\left(\left(u, \widehat{u}\right),v\right) \coloneqq \langle \widehat{u}, \jump{v}
\rangle_{\Gh} - (u,A_h^*v)_{\Oh}= \LRp{f,v}_{\Oh},
\end{equation}
where we have denoted $\LRa{\cdot,\cdot}_{\Gh}$ as the duality
pairing on $\Gh$, $\LRp{\cdot,\cdot}_{\Oh}$ the $L^2$-inner
product over $\Oh$, and $A_h^*$ the formal adjoint resulting from
element-wise integration by parts.  Occasionally, for simplicity in
writing, we will ignore the subscripts in the duality pairing and
$L^2$-inner product if they are $\Gh$ and $\Oh$. Both the
inner product and formal adjoint are understood to be taken
element-wise. Using the ultra-weak formulation, the regularity
requirement on solution variable $u$ is relaxed, that is, $u$ is now
square integrable for the ultra-weak formulation \eqnref{uweak} to be
meaningful, instead of being (weakly) differentiable.  The trade-off
is that $u$ does not admit a trace on $\Gh$ even though it did
originally. Consequently, we need to introduce an additional new
``trace'' variable $\widehat{u}$ in \eqnref{uweak}, that is defined only on
$\Gh$.

The energy setting is now clear; namely,
\[
u\in L^2\LRp{\Oh} \equiv L^2(\Omega), \quad v\in V=D(A^*_h), \quad
\widehat{u}\in \gamma(D(A)),
\]
where $D(A_h^*)$ denotes the broken graph space corresponding to $A_h^*$,
and $\gamma(D(A))$ the trace space (assumed to exist) of the graph space of
operator $A$. The first discussion of the well-posedness of DPG with the ultra-weak formulation can be found in \cite{analysisDPG}, where the proof is presented for the Poisson and convection-diffusion equations. A more comprehensive discussion of the abstract setting for DPG with the ultra-weak formulation using the graph space, as well as a more general proof of well-posedness, can be consulted in \cite{Bui-ThanhDemkowiczGhattas11b}. 

%\subsection{Optimal and quasi-optimal test norms}
\subsection{A canonical energy norm pairing for ultra-weak formulation}

From the discussion in Section~\secref{energyPair} of energy norm and test norm pairings, we know that specifying either a test norm or trial norm is sufficient to define an energy pairing. %In this section, we derive and discuss two important energy norm pairings, the first of which begins by specifying the canonical norm in $U$ and inducing a test norm on $V$. The second pairing begins instead by specifying the canonical norm on $V$ under the ultra-weak formulation \eqnref{uweak} and inducing an energy norm on the trial space $U$.
In this section, we derive and discuss an important energy norm pairing which specifies the canonical norm in $U$ and induces a test norm on $V$. %The second pairing begins instead by specifying the canonical norm on $V$ under the ultra-weak formulation \eqnref{uweak} and inducing an energy norm on the trial space $U$.

We begin first with the canonical norm in $U$. Since $\uh \in \gamma\LRp{D\LRp{A}}$, the standard norm for $\uh$ is
the so-called minimum energy extension norm defined as
\begin{equation}
\eqnlab{MEnorm}
\|\widehat{u}\| = \inf_{w\in D\LRp{A},
  \left.w\right|_{\Gh}=\widehat{u}} \|w\|_{D\LRp{A}}.
\end{equation}
The canonical norm for the group variable $\LRp{u,\uh}$ is then given by
\[
\|\left(u,\widehat{u}\right)\|_U^2 = \|u\|^2_{\L} + \|\widehat{u}\|^2.
\]
Applying the Cauchy-Schwarz inequality, we arrive at
\[
b\LRp{\LRp{u,\uh},v} \le \nor{\LRp{u,\uh}}_U \nor{v}_{V,U},
\]
where
\[
\nor{v}_{V,U}^2 = \|A_h^*v\|_{\L}^2
+\left(\sup_{\widehat{u} \in \gamma\LRp{D\LRp{A}}} \frac{\LRa{ \widehat{u},
  \jump{v} }_{\Gh}}{\|\widehat{u}\|}\right)^2.
\]

%On the other hand, since $v \in D\LRp{A^*_h}$, the canonical norm for
%$v$ is the broken graph norm: 
%\[
%\nor{v}_V^2 =  \|A_h^*v\|_{\L}^2 + \nor{v}_{\L}^2.
%\]
%Using the Cauchy-Schwarz inequality again, we obtain
%\[
%b\LRp{\LRp{u,\uh},v} \le \nor{\LRp{u,\uh}}_{U,V} \nor{v}_{V},
%\]
%where
%\begin{equation}
%\eqnlab{inducedQuasi}
%\nor{\LRp{u,\uh}}_{U,V}^2 = \nor{u}_{\L}^2
%+\sup_{v \in D\LRp{A_h^*}} \frac{\LRa{\widehat{u},
%  \jump{v}}_{\Gh}^2}{\|v\|_V^2},
%\end{equation}

Using the framework developed in \cite{Bui-ThanhDemkowiczGhattas11a}, one can show that $\LRp{\nor{\LRp{u,\uh}}_U,  \nor{v}_{V,U}}$ is an energy norm pairing in the sense discussed in Section \secref{energyPair}. That is, the canonical norm $\nor{\LRp{u,\uh}}_U$ in $U$ induces (generates) the norm $\nor{v}_{V,U}$ in $V$.  

The canonical norm $\nor{\LRp{u,\uh}}_U$ in $U$ provides an optimal balance between the standard norms on the field $u$ and the flux $\uh$ \cite{DPG4}. As a result, if the induced norm $\nor{v}_{V,U}$ (namely, the optimal test norm) is used to compute  optimal test functions in \eqnref{optvVar}, the finite element error in the canonical norm is the best in the sense of \eqnref{optimalError}. Unfortunately, the optimal test norm is non-localizable due to the presence of the jump term $\jump{v}$. Since the jump terms couple elements together, the evaluation of the jump terms requires contributions from all the elements in the mesh. Consequently, solving for an optimal test function amounts to inverting the Riesz map over the entire mesh $\Oh$, making the optimal test norm impractical.

On the other hand, since $v \in D\LRp{A^*_h}$, we can use as a test norm for $v$ the broken graph norm: 
\[
\nor{v}_V^2 =  \|A_h^*v\|_{\L}^2 + \nor{v}_{\L}^2.
\]
This norm is a localization of $\nor{v}_{V,U}$ to allow for the solution of optimal test functions on an element-by-element basis, and is considered to be the canonical norm on $V$.  In the DPG literature \cite{DPG4}, $\nor{v}_{V,U}$ is known as the {\em optimal test norm}, while $\nor{v}_{V}$ is known as the {\em quasi-optimal} or {\em graph test norm}.

%Using the framework developed in \cite{Bui-ThanhDemkowiczGhattas11a}, one can show that both pairs $\LRp{\nor{\LRp{u,\uh}}_U,  \nor{v}_{V,U}}$ and $\LRp{\nor{\LRp{u,\uh}}_{U,V}, \nor{v}_{V}}$ are energy norm pairings in the sense discussed in Section \secref{energyPair}. That is, the canonical norm $\nor{\LRp{u,\uh}}_U$ in $U$ induces (generates) the norm $\nor{v}_{V,U}$ in $V$, while the canonical norm $\nor{v}_V$ in $V$ induces (generates) the energy norm $\nor{\LRp{u,\uh}}_{U,V}$ in $U$. In the DPG literature \cite{DPG4}, $\nor{v}_{V,U}$ is known as the {\em optimal test norm}, while $\nor{v}_{V}$ is known as the {\em quasi-optimal test norm}.

%\begin{figure}[!h]
%\centering
%\begin{tabular}{l c c}
%Trial norm & & Test norm \\
%\hline
%$\boxed{\|u\|^2_{\L} + \|\widehat{u}\|^2}$ & $\Longrightarrow$  & $\|A_h^*v\|_{\L}^2
%+\left(\sup_{\widehat{u}} \frac{\LRa{ \widehat{u},
%  \jump{v} }_{\Gh}}{\|\widehat{u}\|}\right)^2$ \\
%%\hline
%%Quasi-optimal trial norm & & Canonical test norm \\
%%\hline
%$\nor{u}_{\L}^2+\sup_{v } \left(\frac{\LRa{\widehat{u},
%  \jump{v}}_{\Gh}}{\|v\|_V}\right)^2$ &  $\Longleftarrow$ & $\boxed{\|A_h^*v\|_{\L}^2 + \nor{v}_{\L}^2}$
%\end{tabular}
%\caption{A summary of the derivation of test/trial norm pairings; we begin with the boxed norm on either the trial or test space, and induce the norm on the other space through duality. The optimal \textit{test} norm is naturally derived by beginning with the canonical norm on the trial space, while the quasi-optimal \textit{trial} norm is derived from beginning with the canonical norm on the test space.}
%\end{figure}

%The canonical norm $\nor{\LRp{u,\uh}}_U$ in $U$ provides an optimal balance between the standard norms on the field $u$ and the flux $\uh$ \cite{DPG4}. As a result, if the induced norm $\nor{v}_{V,U}$ (namely, the optimal test norm) is used to compute  optimal test functions in \eqnref{optvVar}, the finite element error in the canonical norm is the best in the sense of \eqnref{optimalError}. 
%
%Unfortunately, the optimal test norm is non-localizable due to the presence of the jump term $\jump{v}$. Since the jump terms couple elements together, the evaluation of the jump terms requires contributions from all the elements in the mesh. Consequently, solving for an optimal test function amounts to inverting the Riesz map over the entire mesh $\Oh$, making the optimal test norm impractical.

%On the other hand, the quasi-optimal test norm $\nor{v}_V$, namely the canonical norm in $V$, \textit{is} localizable, and hence practical. However, it's worth noting the difference between the induced energy norm $\nor{\LRp{u,\uh}}_{U,V}$ and the canonical norm in $U$; under the induced norm $\nor{\LRp{u,\uh}}_{U,V}$ there is no natural interpretation for the norm in which the error in the flux variable $\uh$ is measured. 

Using variants of the quasi-optimal test norm, numerical results show that the DPG method appears to provide a ``pollution-free" method without phase error for the Helmholtz equation \cite{DPG4}, and analysis of the pollution-free nature of DPG is currently under investigation. Similar results have also been obtained in the context of elasticity \cite{DPGElas} and the linear Stokes equations \cite{Camellia}. On the theoretical side, the quasi-optimal test norm has been shown to yield a well-posed DPG methodology for the Poisson and convection-diffusion equations \cite{analysisDPG}. More recently, this theory has been generalized to show the well-posedness of DPG for the large class of PDEs of Friedrichs' type \cite{Bui-ThanhDemkowiczGhattas11b}.  

