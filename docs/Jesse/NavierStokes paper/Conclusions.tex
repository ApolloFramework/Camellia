\section{Conclusions and future work}

The goal of this work has been to explore the behavior of the Discontinuous Petrov-Galerkin (DPG) method as a method for the discretization and solution of convection-dominated diffusion problems, and to produce both theory and numerical results for model problems in this area.  We begin by introducing the DPG method for linear problems; the concept of problem-dependent optimal test functions is derived through equivalence with the minimization of a specific residual, and discontinuous test functions are introduced in order to localize the determination of such optimal test functions to a single element.  

The DPG framework is then extended to nonlinear problems, and equivalence is shown between the DPG method and a Gauss-Newton minimization scheme for the nonlinear residual.  An anisotropic refinement scheme is implemented to more effectively capture lower-dimensional behavior of solutions of convection-diffusion problems, such as boundary layers.  We extrapolate the DPG method to a nonlinear viscous Burgers' equation, and conclude by applying the DPG method to systems of equations and to the solution of two model problems in viscous compressible flow.  

In particular, we demonstrate for both the flat plate and compression ramp problem in supersonic/hypersonic in compressible flow that the DPG method is able to begin from a highly underresolved meshes (two elements for the Carter plate problem, and 12 elements for the Holden ramp problem), and resolve physical features of the solution through automatic adaptivity, without the aid of artificial diffusivity or shock capturing terms.  We believe this indicates both the robustness of the method on coarse grids and the effectiveness of the DPG error indicator for adaptive refinement.  

As is the case with any research, much work remains to be done.  We have chosen the classical variables in which to cast the compressible Navier-Stokes equations; however, investigation of alternative sets of variables may have merit, as different choices of variables yield differing linearizations with their own advantages (for example, all derivatives in time are linear with respect to the momentum variables, and the entropy variables of Hughes both symmetrize the Navier-Stokes equations and yield solutions obeying second law of thermodynamics for standard $H^1$ formulations \cite{Hughes1986223}).  

We also hope to investigate artificial viscosity methods as regularization for problems in viscous compressible flow.  We present an analysis of the 1D Burgers' equation demonstrating that the exact solutions under Newton linearization contain large oscillations.  While these oscillations are not the result of the stability of the spatial discretization, their presence can cause density and temperature to become non-physically negative, which can stall the convergence of the nonlinear solver.  Our goal in incorporating artificial viscosity would not be to provide additional stabilization mechanisms for the discrete spatial discretization, but to provide regularization with which to suppress the presence of large oscillations in the linearized solution.  

Finally, though the method is inf-sup stable for arbitrary meshes, our experiments have focused on meshes of uniform $p$.  We hope to implement a true $hp$-adaptive DPG method in the future.  