\documentclass[12pt,letterpaper,oneside]{article}
\usepackage{layout}
\usepackage{fullpage}

\usepackage{array}
\usepackage{amsmath,amssymb,amsfonts,mathrsfs,amsthm}
\usepackage[utf8]{inputenc}
\usepackage{listings}
\usepackage{mathtools}
\usepackage{dsfont}
\usepackage{pdfpages}
\usepackage[textsize=footnotesize,color=green]{todonotes}
\usepackage{algorithm, algorithmic}
\usepackage{bm}
\usepackage{tikz}
\usepackage[normalem]{ulem}

\usepackage{graphicx}
\usepackage{subfigure}
\usepackage{color}
\usepackage{undertilde}
\usepackage[colorlinks = true, filecolor = red, urlcolor = blue, linkcolor = black]{hyperref}
\usepackage{pdflscape}
\usepackage{pifont}
\usepackage{geometry} 
\geometry{letterpaper} 
\usepackage{setspace}
%\usepackage{fullpage}
\setlength\textwidth{6in}
\setlength\textheight{8in}
\setlength\oddsidemargin{0.25in} % LaTeX adds a default 1in to this!
\setlength\evensidemargin{0.25in}
\setlength\topmargin{-0.0in} % LaTeX adds a default 1in to this!
\setlength\headsep{0in}
\setlength\headheight{0in}
\setlength\footskip{1in}

\renewcommand{\topfraction}{0.85}
\renewcommand{\textfraction}{0.1}
\renewcommand{\floatpagefraction}{0.75}

\newcommand{\vect}[1]{\ensuremath\boldsymbol{#1}}
\newcommand{\tensor}[1]{\underline{\vect{#1}}}
\newcommand{\del}{\triangle}
\newcommand{\grad}{\nabla}
\newcommand{\curl}{\grad \times}
\renewcommand{\div}{\grad \cdot}
\newcommand{\ip}[1]{\left\langle #1 \right\rangle}
\newcommand{\eip}[1]{a\left( #1 \right)}
\newcommand{\pd}[2]{\frac{\partial#1}{\partial#2}}
\newcommand{\pdd}[2]{\frac{\partial^2#1}{\partial#2^2}}

\newcommand{\circone}{\ding{192}}
\newcommand{\circtwo}{\ding{193}}
\newcommand{\circthree}{\ding{194}}
\newcommand{\circfour}{\ding{195}}
\newcommand{\circfive}{\ding{196}}

\newcommand{\Reyn}{\rm Re}

\newcommand{\bs}[1]{\boldsymbol{#1}}
\DeclareMathOperator{\diag}{diag}

\newcommand{\equaldef}{\stackrel{\mathrm{def}}{=}}

\newcommand{\tablab}[1]{\label{tab:#1}}
\newcommand{\tabref}[1]{Table~\ref{tab:#1}}

\newcommand{\theolab}[1]{\label{theo:#1}}
\newcommand{\theoref}[1]{\ref{theo:#1}}
\newcommand{\eqnlab}[1]{\label{eq:#1}}
\newcommand{\eqnref}[1]{\eqref{eq:#1}}
\newcommand{\seclab}[1]{\label{sec:#1}}
\newcommand{\secref}[1]{\ref{sec:#1}}
\newcommand{\lemlab}[1]{\label{lem:#1}}
\newcommand{\lemref}[1]{\ref{lem:#1}}

\newcommand{\mb}[1]{\mathbf{#1}}
\newcommand{\mbb}[1]{\mathbb{#1}}
\newcommand{\mc}[1]{\mathcal{#1}}
\newcommand{\nor}[1]{\left\| #1 \right\|}
\newcommand{\snor}[1]{\left| #1 \right|}
\newcommand{\LRp}[1]{\left( #1 \right)}
\newcommand{\LRs}[1]{\left[ #1 \right]}
\newcommand{\LRa}[1]{\left\langle #1 \right\rangle}
\newcommand{\LRc}[1]{\left\{ #1 \right\}}
\newcommand{\tanbui}[2]{\textcolor{blue}{\sout{#1}} \textcolor{red}{#2}}
\newcommand{\Grad} {\ensuremath{\nabla}}
\newcommand{\Div} {\ensuremath{\nabla\cdot}}
\newcommand{\Nel} {\ensuremath{{N^\text{el}}}}
\newcommand{\jump}[1] {\ensuremath{\LRs{\![#1]\!}}}
\newcommand{\uh}{\widehat{u}}
\newcommand{\fnh}{\widehat{f}_n}
\renewcommand{\L}{L^2\LRp{\Omega}}
\newcommand{\pO}{\partial\Omega}
\newcommand{\Gh}{\Gamma_h}
\newcommand{\Gm}{\Gamma_{-}}
\newcommand{\Gp}{\Gamma_{+}}
\newcommand{\Go}{\Gamma_0}
\newcommand{\Oh}{\Omega_h}

\newcommand{\eval}[2][\right]{\relax
  \ifx#1\right\relax \left.\fi#2#1\rvert}

\def\etal{{\it et al.~}}


\def\arr#1#2#3#4{\left[
\begin{array}{cc}
#1 & #2\\
#3 & #4\\
\end{array}
\right]}
\def\vecttwo#1#2{\left[
\begin{array}{c}
#1\\
#2\\
\end{array}
\right]}
\def\vectthree#1#2#3{\left[
\begin{array}{c}
#1\\
#2\\
#3\\
\end{array}
\right]}
\def\vectfour#1#2#3#4{\left[
\begin{array}{c}
#1\\
#2\\
#3\\
#4\\
\end{array}
\right]}
\date{}

\newtheorem{proposition}{Proposition}
\newtheorem{corollary}{Corollary}
\newtheorem{theorem}{Theorem}
\newtheorem{lemma}{Lemma}

\newcommand{\G} {\Gamma}
\newcommand{\Gin} {\Gamma_{in}}
\newcommand{\Gout} {\Gamma_{out}}

\renewcommand{\arraystretch}{1.5}

%%%%%%%%%%%%%%% End def of new commands %%%%%%%%%%%%%%%%%%%

\author{Jesse Chan\\Committee: \vspace{-0.4cm}\\Dr. Leszek Demkowicz (co-supervisor), Dr. Robert Moser (co-supervisor),\vspace{-0.4cm}\\ Dr. Tom Hughes}

\title{A DPG method for compressible flow problems}

\begin{document}
\maketitle

Over the last three decades, Computational Fluid Dynamics (CFD) simulations have become commonplace as a tool in the engineering and design of high-speed aircraft.  Wind tunnel experiments are often complemented by computational simulations, and CFD technologies have proved very useful in both the reduction of aircraft development cycles, and in the simulation of conditions difficult to reproduce experimentally.  Great advances have been made in the field since its introduction, especially in areas of meshing, computer architecture, and solution strategies.  Despite this, there still exist many computational limitations in existing CFD methods; in particular, reliable higher order and $hp$-adaptive methods. 

The use of adaptive meshes is crucial to many CFD applications, where the solution often exhibits very localized sharp gradients and shocks.  Good resolution for such problems under uniform meshes is computationally prohibitive and impractical for most physical regimes of interest.  However, the construction of ``good" meshes is a difficult task, usually requiring a-priori knowledge of the form of the solution.  An alternative to such is the construction of automatically adaptive schemes; such methods begin with a coarse mesh and refine based on the minimization of error.  However, this task is difficult, as the convergence of numerical methods for problems in CFD is notoriously sensitive to mesh quality.  Additionally, the use of adaptivity becomes even more difficult in the context of higher order and $hp$ methods \cite{BoeingHigherOrder}.  

Many of these issues are tied to the notion of \emph{robustness}, which we define loosely for CFD applications as the degradation of the quality of numerical solutions with respect to the Reynolds number, or nondimensional viscosity. For typical physical conditions of interest for the compressible Navier-Stokes equations, the Reynolds number is extremely high, on the order of 1e7, yielding solutions with two vastly different scales - inviscid phenomena at an $O(1)$ scale, and $O(1e-7)$ viscous phenomena. 

%From the perspective of the full Navier-Stokes equations, this loss of robustness is doubly problematic.  Not only will any nonlinear solution suffer from similar non-physical oscillations, but nonlinear solvers themselves may fail to yield a solution due to such instabilities --- a nonlinear solution is almost always computed by solving a series of linear problems whose solutions will converge to the nonlinear solution under appropriate assumptions, and the presence of such oscillations in each linear problem can cause the solution convergence to slow significantly or even diverge.  

 The issue of robustness for finite element methods was addressed early on by Brooks and Hughes in the SUPG method \cite{SUPG}, which introduced the idea of residual-based stabilization methods. Such methods have been widely used in industry; however, these methods are often limited to low-order approximations. Additionally, such methods are usually formulated for model problems and then applied to the full Navier-Stokes equations, where the extrapolation step is fairly ad-hoc.

\todo{finish, add DPG reference}
Demkowicz and Gopalakrishnan introduced the DPG method with optimal test functions, applying it very early on to the convection and convection-diffusion problems \cite{DPG1,DPG2,DPG3}. Since then, the DPG method has been applied to a wide range of problems with great success, including elasticity \cite{DPGElasLocking}, thin body shell problems \cite{DPGElas}, the cloaking problem in electromagnetics \cite{DPGcloak}, both Helmholtz and elastic wave propagation problems \cite{DPG4}, and recently, the linear Stokes equation \cite{stokesDPG}. 

\paragraph{Goal:} Our goal is to develop a DPG method and higher order adaptive scheme for the steady compressible laminar Navier-Stokes equations in transonic/supersonic regimes that is robust over a range of Reynolds numbers. In particular, we hope to present a method for which automatic $hp$-adaptivity can be applied to problems in compressible flow.  

\paragraph{Area A: Applicable Mathematics}

Analysis of convection-diffusion problem, development of a provably robust DPG method for the prototypical singular perturbation problem convection-diffusion. 

\paragraph{Area B: Numerical Analysis and Scientific Computation}

Parallel adaptive DPG code, Hessian-based nonlinear DPG, scalable parallel solvers. 

\paragraph{Area C: Mathematical Modeling and Applications}



\bibliographystyle{plain}
\bibliography{CFD_intro,DPG_old,paper,LitRev,NSNotes}

\end{document}
