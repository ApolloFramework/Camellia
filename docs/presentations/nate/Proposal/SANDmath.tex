%%% ====================================================================
%%% @LaTeX-file{
%%%    filename  = "SANDmath.tex",
%%%    version   = "1.2",
%%%    date      = "2005/01/08",
%%%    author    = "Philippe P. Pebay",
%%%    address   = "Sandia National Laboratories, PO Box 969,
%%%                 MS 9051, Livermore, CA 94550, USA",
%%%    email     = "pppebay@ca.sandia.gov",
%%%    supported = "yes",
%%%    abstract  = "This is a LaTeX documentclass that provides 
%%%                 mathematical environments and macros for Sandia 
%%%                 technical reports (SAND reports).",
%%% }
%%% ====================================================================
\typeout{Using SANDmath LaTeX Package: August 1,
2005, v1.2 Philippe P. Pebay}
% ----------------------------------------------------------------------
% -- Packages
% ----------------------------------------------------------------------
\usepackage{latexsym}
\usepackage{amssymb,amsfonts,amsmath,amsthm}
%\usepackage{calrsfs}
%\usepackage{stmaryrd}
%\usepackage[mathscr]{eucal}
% ----------------------------------------------------------------------
% -- Mathematical environment
% ----------------------------------------------------------------------

\DeclareMathAccent{\maxvec}{\mathord}{letters}{"7E}


% -- equation environment
\numberwithin{equation}{section}
% -- theorems & related topics
\theoremstyle{plain}
\newtheorem{theo}{Theorem}[section]
\newtheorem{prop}[theo]{Proposition}
\newtheorem{lemm}[theo]{Lemma}
\newtheorem{coro}[theo]{Corollary}
% -- definitions & axioms
\theoremstyle{definition}
\newtheorem{defi}{Definition}[section]
\newtheorem{axio}{Axiom}[section]
% -- remarks & examples
\theoremstyle{remark}
\newtheorem{rema}{Remark}[section]
\newtheorem{exam}{Example}[section]
% -- algorithms
\newtheoremstyle{algostyle}% name
  {}%      Space above, empty = `usual value' 
  {}%      Space below
  {\sffamily}%         Body font
  {0pt}%         Indent amount (empty = no indent, \parindent = para indent)
  {\bfseries}% Thm head font
  {}%        Punctuation after thm head
  { }% Space after thm head: \newline = linebreak
  %{\thmname{#1}\thmnumber{ \thesection.#2}{ [#3]}}% Thm head spec
  {\thmname{#1}\thmnumber{ #2}{ [#3]}}% Thm head spec
\theoremstyle{algostyle}
\newtheorem{algo}{Algorithm}
% ----------------------------------------------------------------------
% -- Mathematical macros
% ----------------------------------------------------------------------
% -- K, N, Z, R & C
\newcommand{\K}{{\rm I\kern-.16em K}}
\newcommand{\N}{{\rm I\kern-.16em N}}
\newcommand{\Z}{\mathchoice{\sf\textstyle Z\kern-0.4em Z}%
{\sf\textstyle Z\kern-0.4em Z}%
{\sf\scriptstyle Z\kern-0.3em Z}%
{\sf\scriptscriptstyle Z\kern-0.2em Z}}%
\newcommand{\R}{{\rm I\kern-.16em R}}
\newcommand{\C}{\mathchoice{\setbox0=\hbox{$\displaystyle\rm C$}%
\hbox{\hbox to0pt{\kern0.4\wd0\vrule height0.9\ht0\hss}\box0}}%
{\setbox0=\hbox{$\textstyle\rm C$}\hbox{\hbox%
to0pt{\kern0.4\wd0\vrule height0.9\ht0\hss}\box0}}%
{\setbox0=\hbox{$\scriptstyle\rm C$}\hbox{\hbox%
to0pt{\kern0.4\wd0\vrule height0.9\ht0\hss}\box0}}%
{\setbox0=\hbox{$\scriptscriptstyle\rm C$}\hbox{\hbox%
to0pt{\kern0.4\wd0\vrule height0.9\ht0\hss}\box0}}}
% -- discrete intervals
\newcommand{\cdi}[2]{\llbracket{#1},{#2}\rrbracket}
\newcommand{\odi}[2]{\rrbracket{#1},{#2}\llbracket}
\newcommand{\codi}[2]{\llbracket{#1},{#2}\llbracket}
\newcommand{\ocdi}[2]{\rrbracket{#1},{#2}\rrbracket}
% -- functional spaces
%\newcommand{\CC}[1]{{\mathcal{C}^{#1}}}
\newcommand{\CCs}[2]{{\mathcal{C}^{#1}_{#2}}}
\newcommand{\Lsp}[1]{{\mathrm{L}^{\scriptstyle#1}}}
\newcommand{\Lo}{\Lsp{1}}
\newcommand{\Lt}{\Lsp{2}}
\newcommand{\Li}{\Lsp{\infty}}
% -- differential calculus
\newcommand{\dd}[1]{\mathrm{d}{#1}}
\newcommand{\der}[2]{\frac{\dd{#1}}{\dd{#2}}}
\newcommand{\lder}[2]{\frac{\dd{}}{\dd{#2}}{#1}}
\newcommand{\dern}[3]{\frac{\mathrm{d}^{#1}{#2}}{\dd{#3}^{#1}}}
\newcommand{\ldern}[3]{\frac{\mathrm{d}^{#1}}{\dd{#3}^{#1}}{#2}}
\newcommand{\pder}[2]{\frac{\partial{#1}}{\partial{#2}}}
\newcommand{\lpder}[2]{\frac{\partial}{\partial{#2}}{#1}}
\newcommand{\pdern}[3]{\frac{\partial^{#1}{#2}}{\partial{#3}^{#1}}}
\newcommand{\lpdern}[3]{\frac{\partial^{#1}}{\partial{#3}^{#1}}{#2}}
\newcommand{\pxder}[3]{\frac{\partial^2{#1}}{\partial{#2}\partial{#3}}}
\newcommand{\lpxder}[3]{\frac{\partial^2}{\partial{#2}\partial{#3}}{#1}}
\newcommand{\dive}{\operatorname{div}}
\newcommand{\grad}{\operatorname{grad}}
\newcommand{\curl}{\operatorname{curl}}
\newcommand{\Nd}[1]{\nabla\cdot{#1}}
\newcommand{\Ng}[1]{\nabla\,{#1}}
\newcommand{\Nc}[1]{\nabla\times{#1}}
% -- integral calculus
\newcommand{\sint}[4]{\int_{#1}^{#2}{#3}\,\dd{#4}}
\newcommand{\dint}[3]{\iint_{#1}{#2}\,\dd{#3}}
\newcommand{\ddint}[4]{\iint_{#1}{#2}\,\dd{#3}\dd{#4}}
\newcommand{\tint}[3]{\iiint_{#1}{#2}\,\dd{#3}}
\newcommand{\ttint}[5]{\iiint_{#1}{#2}\,\dd{#3}\dd{#4}\dd{#5}}
% -- vector calculus
\newcommand{\norm}[1]{\left\lVert{#1}\right\rVert}
\newcommand{\normone}[1]{\norm{#1}_1}
\newcommand{\normtwo}[1]{\norm{#1}_2}
\newcommand{\normsup}[1]{\norm{#1}_\infty}
\newcommand{\normgen}[2]{\norm{#2}_{#1}}
\newcommand{\normLo}[1]{\norm{#1}_\Lo}
\newcommand{\normLt}[1]{\norm{#1}_\Lt}
\newcommand{\normLi}[1]{\norm{#1}_\Li}
\newcommand{\normL}[2]{\norm{#2}_\Lsp{#1}}
\newcommand{\seminorm}[2]{\left\lvert{#2}\right\rvert_{#1}}
\newcommand{\indnorm}[1]{\left\lVert\!\left\lvert{#1}\right\rVert\!\right\lvert}
\newcommand{\innprod}[2]{\left({#1}|{#2}\right)}
\newcommand{\dualpair}[2]{\left\langle{#1},{#2}\right\rangle}
\newcommand{\mixprod}[3]{\left[{#1},{#2},{#3}\right]}
% -- asymptotic notations
\newcommand{\smallo}[1]{\operatorname{o}\left({#1}\right)}
\newcommand{\bigo}[1]{\operatorname{O}\left({#1}\right)}
\newcommand{\aseq}[2]{\operatornamewithlimits{\sim}_{{#1}\rightarrow{#2}}}
%%%%%%%%%%%%%%%%%%%%%%%%%%%%%%%%%%%%%%%%%%%%%%%%%%%%%%%%%%%%%%%%%%%%%%%

\newcommand{\mini}{{\min}\quad}
\newcommand{\st}{\mbox{\rm s.t.}\quad}
\newcommand{\CC}{\ensuremath{\mathcal{C}} } %
\newcommand{\CG}{\ensuremath{\mathcal{G}} } %
\newcommand{\CH}{\ensuremath{\mathcal{H}} } %
\newcommand{\CI}{\ensuremath{\mathcal{I}} } %
\newcommand{\CJ}{\ensuremath{\mathcal{J}} } %
\newcommand{\CL}{\ensuremath{\mathcal{L}} } %
\newcommand{\CN}{\ensuremath{\mathcal{N}} } %
\newcommand{\CQ}{\ensuremath{\mathcal{Q}} } %
\newcommand{\CR}{\ensuremath{\mathcal{R}} } %
\newcommand{\CT}{\ensuremath{\mathcal{T}} } %
\newcommand{\CS}{\ensuremath{\mathcal{S}} } %
\newcommand{\CU}{\ensuremath{\mathcal{U}} } %
\newcommand{\CV}{\ensuremath{\mathcal{V}} } %
\newcommand{\CW}{\ensuremath{\mathcal{W}} } %
\newcommand{\CX}{\ensuremath{\mathcal{X}} } %
\newcommand{\CY}{\ensuremath{\mathcal{Y}} } %
\newcommand{\CZ}{\ensuremath{\mathcal{Z}} } %

\newcommand{\lp}{\left(}
\newcommand{\rp}{\right)}
\newcommand{\lb}{\left[}
\newcommand{\rb}{\right]}
\newcommand{\lc}{\left\{}
\newcommand{\rc}{\right\}}
\newcommand{\la}{\left\langle}
\newcommand{\ra}{\right\rangle}

\newcommand {\half} {\mbox{$\frac{1}{2}$}}
\newcommand{\deq}{\raisebox{0pt}[1ex][0pt]{$\stackrel{\scriptscriptstyle{\rm def}}{{}={}}$}}
\newcommand{\SQP}{{S\mspace{-1mu}Q\mspace{-1.2mu}P}}
\newcommand{\CGR}{{C\mspace{-1mu}G}}
\newcommand{\LS}{{L\mspace{-1mu}S}}

\newcommand{\eref}[1]{\mbox{\rm(\ref{#1})}}
\newcommand{\set}[2]{\left\{{#1}\,:~{#2}\right\}}

\DeclareMathOperator*{\minimize}{minimize}

\newcommand{\bu}{\mathbf{u}}
\newcommand{\bg}{\mathbf{g}}
\newcommand{\bzero}{\mathbf{0}}
\newcommand{\bx}{\mathbf{x}}
\newcommand{\bs}{\mathbf{s}}
\newcommand{\bn}{\mathbf{n}}
\newcommand{\bb}{\mathbf{b}}
